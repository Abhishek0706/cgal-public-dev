% +------------------------------------------------------------------------+
% | Reference manual page: CombinatorialMap.tex
% +------------------------------------------------------------------------+
% | 04.02.2010   Guillaume Damiand
% | Package: Combinatorial_map
% +------------------------------------------------------------------------+
\ccRefPageBegin
%%RefPage: end of header, begin of main body
% +------------------------------------------------------------------------+
\begin{ccRefConcept}{CombinatorialMap}
%\ccRefLabel{CGAL::CombinatorialMap}

\ccDefinition
  
The concept \ccRefName\ defines a \emph{d}-dimensional combinatorial map.

%\ccParameters
%\ccc{d} an integer for the dimension of the map.\\
%\ccc{Items} must be a model of the \ccc{CombinatorialMapItems} concept. \\
%\ccc{Alloc} must be a standard allocator for \stl\ container classes.

%\ccIsModel

\ccCreation
\ccCreationVariable{cm}
\ccConstructor{CombinatorialMap();}{Default constructor.}

%%%%%%%%%%%%% TYPES
\ccTypes
\ccNestedType{Dart}{Dart type, a model of the \ccc{Dart} concept.}
\ccGlue
\ccNestedType{Dart_handle}{Dart handle type, equal to \ccc{Dart::Dart_handle}.}
\ccGlue
\ccNestedType{Dart_const_handle}{Dart const handle type, equal to \ccc{Dart::Dart_const_handle}.}
\ccGlue
\ccNestedType{size_type}{Size type (an unsigned integral type).}


% +-----------------------------------+
\ccConstants
\ccVariable{static unsigned int dimension;}{The dimension \emph{d} of \ccc{cm}, equal to \ccc{Dart::dimension.}}
\ccGlue
\ccVariable{static size_type NB_MARKS;}{The number of available Boolean marks of \ccc{cm}.}
\ccGlue
\ccVariable{static Dart_handle null_dart_handle;}{The null dart handle constant: a dart \ccc{d0}
  is \emph{i}-free if \ccc{d0.beta(i)==null_dart_handle}. 
  Note that \ccc{*null_dart_handle}\mynotin{}\ccc{cm.darts()}.}

% \ccGlue
% \ccNestedType{Dart_container}{dart container type.}

%\ccNestedType{Items}{the items class provided as template parameter.}
\ccHeading{Types for attributes}

\ccNestedType{Attributes} 
{The tuple of attributes, containing at most
  \emph{dimension+1} types (one for each possible cell of the combinatorial
  map).  Each type of the tuple must be either a model of the
  \ccc{CellAttribute} concept or \ccc{void}.
  The first type corresponds to 0-attributes, 
  the second to 1-attributes and so on. 
  If the \ccTexHtml{$i^{\mbox{th}}$}{i<sup>th</sup>} type in the tuple is \ccc{void}, 
  \emph{(i-1)}-attributes are disabled. Otherwise, \emph{(i-1)}-attributes are enabled and
  have the given type.  If the size of the tuple is \emph{k},
  with \emph{k}\mylt{}\emph{dimension+1}, \myforall{}\emph{i}: \emph{k}\myleq{}\emph{i}\myleq{}\emph{dimension},
  \emph{i}-attributes are disabled.}

\ccNestedType{template<unsigned int i> Attribute_type<i>::type}
   {Type of \emph{i}-attributes, a model of \ccc{CellAttribute} concept.
     \ccc{Attribute_type<i>::type::Dart_handle} is equal to \ccc{Dart_handle}, and
     \ccc{Attribute_type<i>::type::Dart_const_handle} is equal to \ccc{Dart_const_handle}.
  \ccPrecond{0\myleq{}\emph{i}\myleq{}\emph{dimension} and \emph{i}-attributes are non void.}}
\ccGlue	
\ccNestedType{template<unsigned int i> Attribute_handle<i>::type}
    {Handle to \emph{i}-attributes, equal to \ccc{Dart::Attribute_handle<i>::type}.
      \ccPrecond{0\myleq{}\emph{i}\myleq{}\emph{dimension} and \emph{i}-attributes are non void.}}
\ccGlue
\ccNestedType{template<unsigned int i> Attribute_const_handle<i>::type}
    {Const handle to \emph{i}-attributes, equal to \ccc{Dart::Attribute_const_handle<i>::type}.
      \ccPrecond{0\myleq{}\emph{i}\myleq{}\emph{dimension} and \emph{i}-attributes are non void.}}

\ccHeading{Range types}
\ccNestedType{Dart_range}{Range of all the darts of \ccc{cm}.
  This type is a model of \ccc{Range} concept, its iterator type is bidirectional and its value type is \ccc{Dart}.}
\ccGlue
\ccNestedType{Dart_const_range}{Const range of all the darts of \ccc{cm}.
  This type is a model of \ccc{ConstRange} concept, its iterator type is bidirectional and its value type is \ccc{Dart}.}

\ccNestedType{template<unsigned int i> Attribute_range<i>::type}
{Range of all the \emph{i}-attributes (which must be non void), with 0\myleq{}\emph{i}\myleq{}\emph{dimension}.
  This type is a model of \ccc{Range} concept, its iterator type is bidirectional and its value type is \ccc{Attribute_type<i>::type}.}
\ccGlue
\ccNestedType{template<unsigned int i> Attribute_const_range<i>::type}
{Const range of all the \emph{i}-attributes (which must be non void), with 0\myleq{}\emph{i}\myleq{}\emph{dimension}.
  This type is a model of \ccc{ConstRange} concept, its iterator type is bidirectional and its value type is \ccc{Attribute_type<i>::type}.}

%
\ccNestedType{template<unsigned int ... Beta> Dart_of_orbit_range}
{Range of all the darts of the \ccc{<Beta...>} orbit.
  This type is a model of \ccc{Range} concept, its iterator type is forward and its value type is \ccc{Dart}.}
\ccGlue
\ccNestedType{template<unsigned int ... Beta> Dart_of_orbit_const_range}
{Const range of all the darts of the \ccc{<Beta...>} orbit.
  This type is a model of \ccc{ConstRange} concept, its iterator type is forward and its value type is \ccc{Dart}.}

\ccNestedType{template<unsigned int i,unsigned int dim=dimension> Dart_of_cell_range}
{Range of all the darts of an \emph{i}-cell in \emph{dim} dimension
  (with 0\myleq{}\emph{i}\myleq{}\emph{dim+1} and 0\myleq{}\emph{dim}\myleq{}\emph{dimension}). If \emph{i==dim+1},  
  range of all the darts of a connected component.
  This type is a model of \ccc{Range} concept, its iterator type is forward and its value type is \ccc{Dart}.}
\ccGlue
\ccNestedType{template<unsigned int i,unsigned int dim=dimension> Dart_of_cell_const_range}
{Const range of all the darts of the \emph{i}-cell in \emph{dim} dimension 
  (with 0\myleq{}\emph{i}\myleq{}\emph{dim+1} and 0\myleq{}\emph{dim}\myleq{}\emph{dimension}). If \emph{i==dim+1},  
  range of all the darts of a connected component.
  This type is a model of \ccc{ConstRange} concept, its iterator type is forward and its value type is \ccc{Dart}.}

\ccNestedType{template<unsigned int i,unsigned int j,unsigned int dim=dimension> 
  One_dart_per_incident_cell_range}
{Range of one dart of each \emph{i}-cell incident to one \emph{j}-cell.
  Cells are considered in \emph{dim} dimension  
  (with 0\myleq{}\emph{i}\myleq{}\emph{dim+1}, 0\myleq{}\emph{j}\myleq{}\emph{dim+1} and 
   0\myleq{}\emph{dim}\myleq{}\emph{dimension}). If \emph{i}==\emph{dim+1},  
  consider each connected component instead of each \emph{i}-cell. If \emph{j}==\emph{dim+1},
  consider one connected component instead of one \emph{j}-cell.	
  This type is a model of \ccc{Range} concept, its iterator type is forward and its value type is \ccc{Dart}.}
\ccGlue
\ccNestedType{template<unsigned int i,unsigned int j,unsigned int dim=dimension> 
  One_dart_per_incident_cell_const_range}
{Const range of one dart of each \emph{i}-cell incident to one \emph{j}-cell.
  Cells are considered in \emph{dim} dimension  
  (with 0\myleq{}\emph{i}\myleq{}\emph{dim+1}, 0\myleq{}\emph{j}\myleq{}\emph{dim+1} and 
   0\myleq{}\emph{dim}\myleq{}\emph{dimension}). If \emph{i==dim+1},  
  consider each connected component instead of each \emph{i}-cell. If \emph{j==dim+1},
  consider one connected component instead of one \emph{j}-cell.
  This type is a model of \ccc{ConstRange} concept, its iterator type is forward and its value type is \ccc{Dart}.}

\ccNestedType{template<unsigned int i,unsigned int dim=dimension> One_dart_per_cell_range}
{Range of one dart of each \emph{i}-cell of \ccc{cm}.
  Cells are considered in \emph{dim} dimension  
  (with 0\myleq{}\emph{i}\myleq{}\emph{dim+1} and 0\myleq{}\emph{dim}\myleq{}\emph{dimension}). 
  If \emph{i==dim+1}, consider each connected component instead of each \emph{i}-cell. 
  This type is a model of \ccc{Range} concept, its iterator type is forward and its value type is \ccc{Dart}.}
\ccGlue
\ccNestedType{template<unsigned int i,unsigned int dim=dimension> One_dart_per_cell_const_range}
{Const range of one dart of each \emph{i}-cell of \ccc{cm}.
  Cells are considered in \emph{dim} dimension 
  (with 0\myleq{}\emph{i}\myleq{}\emph{dim+1} and 0\myleq{}\emph{dim}\myleq{}\emph{dimension}). 
  If \emph{i==dim+1}, consider each connected component instead of each \emph{i}-cell. 
  This type is a model of \ccc{ConstRange} concept, its iterator type is forward and its value type is \ccc{Dart}.}

% +-----------------------------------+
\ccHeading{Access Member Functions}

\ccMethod{bool is_empty() const;} {Returns true iff \ccc{cm} is empty,
  i.e. it contains no dart.}

\ccMethod{bool is_valid() const;}
         {Returns true iff \ccc{cm} is valid.}
A combinatorial map \ccc{cm} is valid (see Sections~\ref{ssec-combi-map-and-darts} and \ref{ssec-combimap-validity}) 
if for all dart handle \emph{dh} such that 
\ccc{*dh}\myin{}\ccc{cm.darts()}:\\
\begin{itemize}
\item \ccc{dh->is_free(0)}, or \ccc{dh->beta(0)->beta(1)==dh};
\item \ccc{dh->is_free(1)}, or \ccc{dh->beta(1)->beta(0)==dh};
\item \myforall{}\emph{i}, 2\myleq{}\emph{i}\myleq{}\emph{dimension}: 
  \ccc{dh->is_free(i)}, or \ccc{dh->beta(i)->beta(i)==dh};
\item \myforall{}\emph{i},\emph{j}, 0\myleq{}\emph{i}\mylt{}\emph{i}+2\myleq{}\emph{j}\myleq{}\emph{dimension} 
  such that \emph{j}\mygeq{}3: \ccc{dh->beta(i)->beta(j)==null_dart_handle} or
  \ccc{dh->beta(i)->beta(j)->beta(i)->beta(j)==dh};
\item \myforall{}\emph{i}, 0\myleq{}\emph{i}\myleq{}\emph{dimension} such that \emph{i}-attributes are non void:
   \myforall{}\emph{dh2} such that \ccc{*dh2} belong to the same \emph{i}-cell than \ccc{*dh}:
   \ccc{dh2->attribute<i>()==dh->attribute<i>()}.
 \end{itemize}

\ccMethod{bool is_without_boundary(unsigned int i) const;}
         {Returns true iff \ccc{cm} is wihout \emph{i}-boundary
          (i.e. there is no \ccc{i}-free dart).
          \ccPrecond{1\myleq{}\emph{i}\myleq{}\emph{dimension}.}}

\ccMethod{bool is_without_boundary() const;}
         {Returns true iff \ccc{cm} is without boundary in all dimensions.}

\ccMethod{size_type number_of_darts() const;}
    {Returns the number of darts in \ccc{cm}.}

\ccMethod{template <unsigned int i> size_type number_of_attributes() const;}
   {Returns the number of \emph{i}-attributes in \ccc{cm}. 
     \ccPrecond{0\myleq{}\emph{i}\myleq{}\emph{dimension}, and \emph{i}-attributes are non void.}}

\ccMethod{Dart_handle dart_handle(Dart& adart);}
    {Returns the dart handle of \ccc{adart}.
\ccPrecond{\ccc{adart}\myin{}\ccc{cm.darts()}.}}
\ccGlue
\ccMethod{Dart_const_handle dart_handle(const Dart& adart) const;}
    {Returns the dart const handle of \ccc{adart}.
\ccPrecond{\ccc{adart}\myin{}\ccc{cm.darts()}.}}

\ccMethod{Dart_handle beta(Dart_handle dh, int i, int j) const;}
{Returns \betaj{}(\betai{}(\ccc{dh})). 
 Overloads of this member function are defined that take from one to nine integer as arguments.
 For each function, betas are applied in the same order as their indices are given as parameters. 
 For example \ccc{beta(dh,1)}=\betaun{}(\ccc{dh}),
 and \ccc{beta(dh,1,2,3,0)}=\betazero{}(\betatrois{}(\betadeux{}(\betaun{}(\ccc{dh})))).
\ccPrecond{0\myleq{}\emph{i}\myleq{}\emph{dimension}, 0\myleq{}\emph{j}\myleq{}\emph{dimension} and \ccc{*dh}\myin{}\ccc{cm.darts()}.}
}
\ccGlue
\ccMethod{Dart_const_handle beta(Dart_const_handle dh, int i, int j) const;}
{Returns \betaj{}(\betai{}(\ccc{dh})).
  Overloads of this member function are defined that take from one to nine integer as arguments.
\ccPrecond{0\myleq{}\emph{i}\myleq{}\emph{dimension}, 0\myleq{}\emph{j}\myleq{}\emph{dimension} and \ccc{*dh}\myin{}\ccc{cm.darts()}.}
}

\ccMethod{template <unsigned int i> bool is_sewable(Dart_const_handle dh1, Dart_const_handle dh2) const;}
         {Returns true iff \ccc{dh1} can be \emph{i}-sewn with \ccc{dh2} by 
           keeping \ccc{cm} valid, i.e. if there is
           a bijection \emph{f} between all the darts of the orbit
           \emph{D1}=\orbit{\betaun{},\myldots{},\betaimdeux{},\betaipdeux{},\myldots{},\betad{}}(\emph{dh1}) and
           \emph{D2}=\orbit{\betaun{},\myldots{},\betaimdeux{},\betaipdeux{},\myldots{},\betad{}}(\emph{dh2})
           satisfying: \emph{f}(\emph{dh1})=\emph{dh2}, and for all \emph{e}\myin{}\emph{D1}, for all \emph{j}\myin{}\{1,\myldots{},\emph{i}-2,\emph{i}+2,\myldots{},\emph{d}\},
           \emph{f}(\betaj{}(\emph{e}))=\betajinv{}(\emph{f}(\emph{e})).           
           \ccPrecond{0\myleq{}\emph{i}\myleq{}\emph{dimension}, \ccc{*dh1}\myin{}\ccc{cm.darts()}, 
             and \ccc{*dh2}\myin{}\ccc{cm.darts()}.}}	

\ccMethod{std::ostream& display_characteristics(std::ostream & os) const;}
{Displays on \ccc{os} the characteristics of \ccc{cm}: its number of darts, 
  its number of \emph{i}-cells, for each \emph{i}, 0\myleq{}\emph{i}\myleq{}CMap::\emph{dimension}, 
  and its number of connected components.}
Example of output for a 3D combinatorial map containing two disjoint
combinatorial tetrahedra:\\
\texttt{\#Darts=24, \#0-cells=8, \#1-cells=12, \#2-cells=8, \#3-cells=2, \#ccs=2}


% +-----------------------------------+
\ccHeading{Range Access Member Functions}

\ccMethod{Dart_range& darts();}
    {Returns a range of all the darts in \ccc{cm}.}
\ccGlue
\ccMethod{Dart_const_range& darts() const;}
    {Returns a const range of all the darts in \ccc{cm}.}

\ccMethod{template<unsigned int i> Attribute_range<i>::type & attributes();}
   {Returns a range of all the \emph{i}-attributes in \ccc{cm}.
     \ccPrecond{0\myleq{}\emph{i}\myleq{}\emph{dimension}, and \emph{i}-attributes are non void.}}
\ccGlue
\ccMethod{template<unsigned int i> Attribute_const_range<i>::type & attributes() const;}
   {Returns a const range of all the  \emph{i}-attributes in \ccc{cm}.
     \ccPrecond{0\myleq{}\emph{i}\myleq{}\emph{dimension}, and \emph{i}-attributes are non void.}}

% \ccMethod{std::vector<unsigned int> count_cells(const std::vector<unsigned int>& acells) const;}
%     {count the number of $i$-cells of the combinatorial map, for each $i$ in the 
%       \ccc{std::vector} \ccc{acells}, with $0 \leq i \leq dimension+1$ 
%       (for $i=dimension+1$, count the number of connected component).
%       Return a \ccc{std::vector} of size $dimension+2$ containing, for each 
%       $i$ in \ccc{acells}, the number of $i$-cell.}
 
\ccMethod{template<unsigned int ... Beta> Dart_of_orbit_range darts_of_orbit(Dart_handle dh);}
   {Returns a range of all the darts of the orbit \ccc{<Beta...>(dh)}.
     \ccPrecond{\ccc{*dh}\myin{}\ccc{cm.darts()} and \ccc{Beta...} is a
     sequence of integers \ccTexHtml{$i_1$}{i<sub>1</sub>},\myldots{},\ccTexHtml{$i_k$}{i<sub>k</sub>}, such that 
     0\myleq{}\ccTexHtml{$i_1$}{i<sub>1</sub>}\mylt{}\ccTexHtml{$i_2$}{i<sub>2</sub>}\mylt{}\myldots{}\mylt{}\ccTexHtml{$i_k$}{i<sub>k</sub>}\myleq{}\emph{dimension}, and (\ccTexHtml{$i_1$}{i<sub>1</sub>}\myneq{}0 or \ccTexHtml{$i_2$}{i<sub>2</sub>}\myneq{}1).}}
\ccGlue
\ccMethod{template<unsigned int ... Beta> Dart_of_orbit_const_range darts_of_orbit(Dart_const_handle dh) const;}
   {Returns a const range of all the darts of the orbit \ccc{<Beta...>(dh)}.
   \ccPrecond{Same as for the non const version.}}

\ccMethod{template<unsigned int i,unsigned int dim=dimension> Dart_of_cell_range darts_of_cell(Dart_handle dh);}
   {Returns a range of all the darts of the \emph{i}-cell containing \ccc{dh}.
    \emph{i}-cells are considered in \emph{dim} dimension. If \emph{i==dim+1},  
    range of all the darts of the connected component containing \ccc{dh}.
  \ccPrecond{\ccc{*dh}\myin{}\ccc{cm.darts()},
    0\myleq{}\emph{i}\myleq{}\emph{dim+1} and 0\myleq{}\emph{dim}\myleq{}\emph{dimension}.}}
\ccGlue
\ccMethod{template<unsigned int i,unsigned int dim=dimension> Dart_of_cell_const_range darts_of_cell(Dart_const_handle dh) const;}
   {Returns a const range of all the darts of the \emph{i}-cell containing \ccc{dh}.
    \emph{i}-cells are considered in \emph{dim} dimension. If \emph{i==dim+1},  
    const range of all the darts of the connected component containing \ccc{dh}.
   \ccPrecond{Same as for the non const version.}}

\ccMethod{template<unsigned int i,unsigned int j,unsigned int dim=dimension> One_dart_per_incident_cell_range  one_dart_per_incident_cell(Dart_handle dh);}
   {Returns a range of one dart of each \emph{i}-cell incident to the \emph{j}-cell 
     containing \ccc{dh}. Cells are considered in \emph{dim} dimension. If \emph{i==dim+1},  
     consider each connected component instead of each \emph{i}-cell. If \emph{j==dim+1},
     consider the connected component containing \ccc{dh} instead of the \emph{j}-cell.
     \ccPrecond{\ccc{*dh}\myin{}\ccc{cm.darts()}, 0\myleq{}\emph{i}\myleq{}\emph{dim+1},
       0\myleq{}\emph{j}\myleq{}\emph{dim+1} and 0\myleq{}\emph{dim}\myleq{}\emph{dimension}.}}
\ccGlue
\ccMethod{template<unsigned int i,unsigned int j,unsigned int dim=dimension> One_dart_per_incident_cell_const_range one_dart_per_incident_cell(Dart_const_handle dh) const;}
   {Returns a const range of one dart of each \emph{i}-cell incident to the \emph{j}-cell 
     containing \ccc{dh}. Cells are considered in \emph{dim} dimension. If \emph{i==dim+1},  
     consider each connected component instead of each \emph{i}-cell. If \emph{j==dim+1},
     consider the connected component containing \ccc{dh} instead of the \emph{j}-cell.
     \ccPrecond{Same as for the non const version.}}

\ccMethod{template<unsigned int i,unsigned int dim=dimension> One_dart_per_cell_range one_dart_per_cell();}
    {Returns a range of one dart of each \emph{i}-cell in \ccc{cm}.
     Cells are considered in \emph{dim} dimension. If \emph{i==dim+1},  
     range of one dart of each connected component in \ccc{cm}.
  \ccPrecond{0\myleq{}\emph{i}\myleq{}\emph{dim+1} and 0\myleq{}\emph{dim}\myleq{}\emph{dimension}.}}
\ccGlue
\ccMethod{template<unsigned int i,unsigned int dim=dimension> One_dart_per_cell_const_range one_dart_per_cell() const;}
    {Returns a const range of one dart of each \emph{i}-cell in \ccc{cm}.
     Cells are considered in \emph{dim} dimension. If \emph{i==dim+1},  
     const range of one dart of each connected component in \ccc{cm}.
   \ccPrecond{Same as for the non const version.}}

% +-----------------------------------+
\ccHeading{Modifiers}
\ccMethod{template<typename T1> Dart_handle create_dart(T1 t1);} 
  {Creates a new dart in \ccc{cm}, and returns the corresponding handle.
   Calls the constructor of dart having \ccc{T1} as parameter.
   A new dart is initialized to be \emph{i}-free,
   \myforall{}\emph{i}: 0\myleq{}\emph{i}\myleq{}\emph{dimension}, and to have no associated 
   attribute for each non void attribute.
   Overloads of this member function are defined that take from zero to nine arguments.
   With zero argument, \ccc{create_dart()} creates a new dart by using the default constructor.
   }

\ccMethod{void erase_dart(Dart_handle dh);}
   {Removes \ccc{*dh} from \ccc{cm}.
     \ccPrecond{\ccc{*dh}\myin{}\ccc{cm.darts()}.}
}

\ccMethod{template<unsigned int i,typename T1> Attribute_handle<i>::type create_attribute(T1 t1);} 
  {Creates a new \emph{i}-attribute in \ccc{cm}, and returns the corresponding handle.
   Calls the constructor of \emph{i}-attribute having \ccc{T1} as parameter.
   Overloads of this member function are defined that take from zero to nine arguments.
   With zero argument, \ccc{create_attribute<i>()} creates a new \emph{i}-attribute by using the default constructor.
   \ccPrecond{0\myleq{}\emph{i}\myleq{}\emph{dimension}, and \emph{i}-attributes are non void.}}

% \ccMethod{template<unsigned int i> Attribute_handle<i>::type create_attribute();}
% {Creates a new \emph{i}-attribute in \ccc{cm}, and returns the corresponding handle.
%   \ccPrecond{0\myleq{}\emph{i}\myleq{}\emph{dimension}, and \emph{i}-attributes are non void.}}
	
% \ccMethod{template<unsigned int i,class A> Attribute_handle<i>::type 
%   create_attribute(const A&a);}
% {Creates a new \emph{i}-attribute in \ccc{cm} initialized 
%   by the constructor \ccc{A(a)}, and returns the corresponding handle. \ccc{A} must be 
%   a type compatible with the information contained in the \emph{i}-attributes 
%   (compatible means that there is a constructor for the information type 
%   taking an \emph{A} as parameter).
%   \ccPrecond{0\myleq{}\emph{i}\myleq{}\emph{dimension}, and \emph{i}-attributes are non void.}}

\ccMethod{template <unsigned int i> void erase_attribute(Attribute_handle<i>::type ah);}
{Removes the \emph{i}-attribute \ccc{*ah} from \ccc{cm}.
  \ccPrecond{0\myleq{}\emph{i}\myleq{}\emph{dimension}, \emph{i}-attributes are non void,
    and \ccc{*ah}\myin{}\ccc{cm.attributes<i>()}.}}

\ccMethod{template <unsigned int i> void set_attribute(Dart_handle dh, Attribute_handle<i>::type ah);} 
   {Associates the \emph{i}-attribute of all the darts of the \emph{i}-cell 
     containing \ccc{dh} to \ccc{ah}.
  \ccPrecond{\ccc{*dh}\myin{}\ccc{cm.darts()}, 0\myleq{}\emph{i}\myleq{}\emph{dimension}, 
    \emph{i}-attributes are non void, and \ccc{*ah}\myin{}\ccc{cm.attributes<i>()}.}}

\ccMethod{void clear();}
   {Deletes all the darts and all the attributes of \ccc{cm}.}
% +-----------------------------------+
\ccOperations

\ccMethod{template <unsigned int i> void sew(Dart_handle dh1,
  Dart_handle dh2, bool update_attributes=true);} {Links by \betai{}
  two by two all the darts of the orbit
  \emph{D1}=\orbit{\betaun{},\myldots{},\betaimdeux{},\betaipdeux{},\myldots{},\betad{}}(\ccc{dh1}) and
  \emph{D2}=\orbit{\betazero{},\betadeux{},\myldots{},\betaimdeux{},\betaipdeux{},\myldots{},\betad{}}	(\ccc{dh2})
  such that \emph{d2}=\emph{f}(\emph{d1}), \emph{f} being the bijection between \emph{D1} and \emph{D2} 
  satisfying: \emph{f}(\emph{dh1})=\emph{dh2}, and for all \emph{e}\myin{}\emph{D1}, for all 
  \emph{j}\myin{}\{1,\myldots{},\emph{i}-2,\emph{i}+2,\myldots{},\emph{d}\},
  \emph{f}(\betaj{}(\emph{e}))=\betajinv{}(\emph{f}(\emph{e})).

  If \ccc{update_attributes} is \ccc{true}, when necessary, non void
  attributes are updated to ensure the validity of \ccc{cm}: for each
  \emph{j}-cells \emph{c1} and \emph{c2} which are merged into one \emph{j}-cell during
  the sew, the two associated attributes \emph{attr1} and \emph{attr2} are
  considered.  If one attribute is
  NULL and the other not, the non NULL attribute is associated to all
  the darts of the resulting cell.  When the two attributes are non
  NULL, functor \ccc{Attribute_type<i>::type::On_merge} is called on
  the two attributes \emph{attr1} and \emph{attr2}. Then, the attribute
  \emph{attr1} is associated to all darts of the resulting
  \emph{j}-cell. Finally, attribute \emph{attr2} is removed from \ccc{cm}.
  \ccPrecond{\ccc{cm.is_sewable<i>(dh1,dh2)}.}
  \begin{ccAdvanced}
    If \ccc{update_attributes} is \ccc{false}, non void attributes are
    not updated; thus \ccc{cm} can be no more valid after this
    modification.
  \end{ccAdvanced}
}

\ccMethod{template <unsigned int i> void unsew(Dart_handle dh, bool
  update_attributes=true);} {Unlinks by \betai{} all the darts in the
  orbit
  \orbit{\betaun{},\myldots{},\betaimdeux{},\betaipdeux{},\myldots{},\betad{}}(\ccc{dh}).  If
  \ccc{update_attributes} is \ccc{true}, when necessary, non void
  attributes are updated to ensure the validity of \ccc{cm}: for each
  \emph{j}-cell \emph{c} split in two \emph{j}-cells \emph{c1} and \emph{c2} by the
  operation, if \emph{c} is associated to a \emph{j}-attribute \emph{attr1}, then
  this attribute is duplicated into \emph{attr2}, and all the darts
  belonging to \emph{c2} are associated with this new attribute.  Finally,
  the functor \ccc{Attribute_type<i>::type::On_split} is called on the
  two attributes \emph{attr1} and \emph{attr2}.
 \ccPrecond{0\myleq{}\emph{i}\myleq{}\emph{dimension}, \ccc{*dh}\myin{}\ccc{cm.darts()} and
   \ccc{dh} is not \emph{i}-free.}
\begin{ccAdvanced}
  If \ccc{update_attributes} is \ccc{false}, non void attributes are
  not updated thus \ccc{cm} can be no more valid after this
  modification. 
\end{ccAdvanced}}

\begin{ccAdvanced}

\ccMethod{template <unsigned int i> void link_beta(Dart_handle dh1, Dart_handle dh2, bool update_attributes=true);}
{Links \ccc{dh1} and \ccc{dh2}  by \betai{}. 
  \ccc{cm} can be no more valid after this modification.  If
  \ccc{update_attributes} is true, non void attributes of \ccc{dh1} and
  \ccc{dh2} are updated: if one dart has an attribute and the second
  dart not, the non null attribute is associated to the dart having a null attribute. 
  If both darts have an attribute, 
  the attribute of \ccc{dh1} is associated to \ccc{dh2}.
  \ccPrecond{0\myleq{}\emph{i}\myleq{}\emph{dimension}, \ccc{*dh1}\myin{}\ccc{cm.darts()},
    \ccc{*dh2}\myin{}\ccc{cm.darts()} and (\emph{i}\mylt{}2 or \ccc{dh1}\myneq{}\ccc{dh2}).}}


\ccMethod{template <unsigned int i> void unlink_beta(Dart_handle dh);}
  {Unlinks \ccc{dh} and \betai{}(\ccc{dh}) by \betai{}.
    \ccc{cm} can be no more valid after this modification. 
    Attributes of \ccc{dh} and \betai{}(\ccc{dh})
    are not modified.
    \ccPrecond{0\myleq{}\emph{i}\myleq{}\emph{dimension}, \ccc{*dh}\myin{}\ccc{cm.darts()},
      and \ccc{dh} is not \emph{i}-free.}}

\end{ccAdvanced}

% +-----------------------------------+
\begin{ccAdvanced}
\ccHeading{Boolean marks}

\ccMethod{int get_new_mark() const;}{Reserves a new mark. Returns its
  index. Returns -1 if there is no more available free mark.}

\ccMethod{bool is_reserved(int amark) const;}
  {Returns true iff \ccc{amark} is a reserved mark of \ccc{cm}.
  \ccPrecond{0\myleq{}\emph{amark}\mylt{}\ccc{NB_MARKS}.}}

\ccMethod{bool is_marked(Dart_const_handle dh, int amark) const;}
   {Returns true iff \ccc{dh} is marked for \ccc{amark}.
   \ccPrecond{\ccc{is_reserved(amark)} and \ccc{*dh}\myin{}\ccc{cm.darts()}.}}

\ccMethod{void mark(Dart_const_handle dh, int amark) const;}
   {Marks \ccc{dh} for \ccc{amark}.
   \ccPrecond{\ccc{is_reserved(amark)} and \ccc{*dh}\myin{}\ccc{cm.darts()}.}}

\ccMethod{void unmark(Dart_const_handle dh, int amark) const;}
   {Unmarks \ccc{dh} for the mark \ccc{amark}.
   \ccPrecond{\ccc{is_reserved(amark)} and \ccc{*dh}\myin{}\ccc{cm.darts()}.}}

\ccMethod{void negate_mark(int amark) const;}
   {Inverse the mark \ccc{amark} for all the darts of \ccc{cm}.
     All the marked darts become unmarked and all the unmarked darts 
     become marked.
   \ccPrecond{\ccc{is_reserved(amark)}.}}

\ccMethod{void unmark_all(int amark) const;}
   {Unmarks all the darts of \ccc{cm} for \ccc{amark}.
   \ccPrecond{\ccc{is_reserved(amark)}.}}

\ccMethod{size_type number_of_marked_darts(int amark) const;}
   {Returns the number of marked darts for \ccc{amark}.
   \ccPrecond{\ccc{is_reserved(amark)}.}}

\ccMethod{size_type number_of_unmarked_darts(int amark) const;}
   {Return the number of unmarked darts for \ccc{amark}.
   \ccPrecond{\ccc{is_reserved(amark)}.}}

\ccMethod{void free_mark(int amark) const;}
  {Frees \ccc{amark}.
   \ccPrecond{\ccc{is_reserved(amark)}.}}
\end{ccAdvanced}

\ccHasModels
\ccRefIdfierPage{CGAL::Combinatorial_map<d,Items,Alloc>}

% \ccSeeAlso
% \ccRefIdfierPage{CGAL::CombinatorialMapWithPoints}

\end{ccRefConcept}
% +------------------------------------------------------------------------+
%%RefPage: end of main body, begin of footer
\ccRefPageEnd
% EOF
% +------------------------------------------------------------------------+
