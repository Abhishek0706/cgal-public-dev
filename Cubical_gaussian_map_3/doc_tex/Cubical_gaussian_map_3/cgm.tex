\section{Introduction}
\label{CGM_sec:intro}

{\em Cubical Gaussian Map} (CGM) is a basic data dtructure that represents a
Gaussinal map mapped onto a unit cube (an axes parallel cube centered at the
origin whose edges are of length two). {\em Polyhedral CGM} is a data dtructure
that represents a 3D convex polyhedron. It is derived from the CGM data
structure. {\em Spherical CGM} is a data structure that represents an
2D arrangement on the surface of the unit sphere.

\section{CGM}
The baseic CGM data structure consists of six planar maps that
correspond to the six faces of the unit cube. The faces are indexed 0 to 5,
where 0, 1, and 2 refer to the faces that reside in the negative X, Y, and Z
halfspaces respectively. In a similar way 3, 4, and 5 refer to the faces that
reside in the positive X, Y, and Z halfspaces respectively.

\section{Polyhedral CGM}
The Polyhedral CGM can be easily constructed from 2D polygonal cells that
define the 2D surface of a polyhedron, and the 2D polygonal cells that
define the 2D surface of a polyhedron can be easily extracted from this
representation. Each planar map of the basic CGM data structure represents
a portion of the polyhedron surface-boundary with overlaps between the planar
maps. We project the normals of each facet of the polyhedron onto the faces of
the unit cube. A planar map is formed on each face of the unit cube as follows:
Each interesection of a polyhedron facet-normal with a cube face defines a
vertex in the planar map associated with that face. We connect vertices that
are generated by normals of adjacent facets. This forms a planar map bounded by
a unit square on each unit-cube face.

\section{Spherical CGM}

\section{Implementation Details}
When reducing the 3D coordinate system onto a 2D coordinate system attached
to a paerticular unit-cube face, we project the 3D coordinate system as follows:

\begin{tabular}{ll}
Unit-Cube Face & Projection\\
\hline
$0$ & $X,Y,Z \Rightarrow Y,Z$\\
$1$ & $X,Y,Z \Rightarrow Z,X$\\
$2$ & $X,Y,Z \Rightarrow X,Y$\\
$3$ & $X,Y,Z \Rightarrow Z,Y$\\
$4$ & $X,Y,Z \Rightarrow X,Z$\\
$5$ & $X,Y,Z \Rightarrow Y,X$\\
\end{tabular}

As mentioned above each planar map is bounded by a 2-unit square centered at
the origin of a 2D Cartesian coordinate system. Each 2D vertex and 2D edge of
the boundary is indexed as depicted below:

\pspicture[](-2.4,-1.2)(2.4,1.2)
\psset{unit=1cm,subgriddiv=0,arrowsize=3pt 3,linewidth=1pt}
\cnode*(-1,-1){2pt}{0}
\cnode*(-1, 1){2pt}{1}
\cnode*( 1,-1){2pt}{2}
\cnode*( 1, 1){2pt}{3}
\ncline{0}{1}
\ncline{1}{3}
\ncline{3}{2}
\ncline{2}{0}
\uput[  45]{0}( 1, 1){$( 1, 1)$}
\uput[-135]{0}( 1, 1){$3$}
\uput[ -45]{0}( 1,-1){$( 1,-1)$}
\uput[ 135]{0}( 1,-1){$2$}
\uput[ 135]{0}(-1, 1){$(-1, 1)$}
\uput[ -45]{0}(-1, 1){$1$}
\uput[-135]{0}(-1,-1){$(-1,-1)$}
\uput[  45]{0}(-1,-1){$0$}
\endpspicture

\begin{wrapfigure}{l}{8.4cm}
  \vspace{-1.5ex}
  \scalebox{0.5 0.5}{
    \newlength{\displacement}\setlength{\displacement}{0.4cm}
\newlength{\ndisplacement}\setlength{\ndisplacement}{-0.4cm}
\newlength{\hdisplacement}\setlength{\hdisplacement}{0.2cm}
  \begin{tabular}{cc}
    \pspicture[](-3,-2.5)(5,4.5)
    \psset{unit=1cm,subgriddiv=0,arrowsize=3pt 3,linewidth=1pt,
      viewpoint=1 -1 1
    }
    \cnode*(-2,-2){2pt}{0}
    \cnode*(-2, 2){2pt}{1}
    \cnode*( 2, 2){2pt}{2}
    \cnode*( 2,-2){2pt}{3}
    \rput{30}(0,0){
      \rput{-30}(2,0){
	\cnode(-2,-2){2pt}{4}
	\cnode*(-2, 2){2pt}{5}
	\cnode*( 2, 2){2pt}{6}
	\cnode*( 2,-2){2pt}{7}
      }
    }
    \rput{30}(2,-2){
      \rput{0}(2,0){
	\rput{-30}(-\hdisplacement,0){
	  \rput{0}(0,\displacement){
	    \pnode(0,0){a1}
	    \pnode(0,5){b1}
	    \uput[90]{0}(0,5){$X$}
	    \rput{30}(0,0){
	      \pnode(-3,0){c1}
	      \uput[180]{-30}(-3,0){$Y$}
	    }
	  }
	}
      }
    }
    \rput{30}(-2,2){
      \rput{0}(2,0){
	\rput{-30}(-\hdisplacement,0){
	  \rput{0}(\displacement,0){
	    \pnode(0,0){a2}
	    \pnode(5,0){b2}
	    \uput[0]{0}(5,0){$Y$}
	    \rput{30}(0,0){
	      \pnode(-3,0){c2}
	      \uput[180]{-30}(-3,0){$X$}
	    }
	  }
	}
      }
    }
    \ncline{0}{1}
    \ncline{1}{2}
    \ncline{2}{3}
    \ncline{3}{0}
    \ncline[linecolor=gray,linestyle=dashed]{4}{5}
    \ncline{5}{6}
    \ncline{6}{7}
    \ncline[linecolor=gray,linestyle=dashed]{7}{4}
    \ncline[linecolor=gray,linestyle=dashed]{0}{4}
    \ncline{1}{5}
    \ncline{2}{6}
    \ncline{3}{7}
    \ncline[linecolor=blue]{->}{a1}{b1}
    \ncline[linecolor=blue]{->}{a1}{c1}
    \ncline[linecolor=green]{->}{a2}{b2}
    \ncline[linecolor=green]{->}{a2}{c2}
    \rput{0}(\displacement,\displacement){
      \psline[linecolor=red]{->}(-2,-2)(3,-2)
      \psline[linecolor=red]{->}(-2,-2)(-2,3)
      \uput[0]{0}(3,-2){$X$}
      \uput[90]{0}(-2,3){$Y$}
    }
    \rput{30}(0,0){
      \rput{-30}(1,0){
	\my_axis{0}{1}{90}{1}{210}{0.7}
      }
    }
    \rput{0}(1,-0.5){\psframebox*[framearc=.3]{\red \Huge 5}}
    \rput{30}(2,0){
      \rput{-30}(1,0){
	\ThreeDput[normal=1 .0 .0](0,0,0){
	  \psframebox*[framearc=.3]{\blue \Huge 3}
	}
      }
    }
    \rput{0}(2,2){
      \rput{30}(0,-\displacement){
	\rput{0}(2,0){
	  \rput{-30}(-\hdisplacement,0){
	    \ThreeDput[normal=1 .0 .0](0,0,0){
	      \psframebox*[framearc=.3]{\blue \Large 2}
	    }
	  }
	}
      }
    }
    \rput{0}(2,-2){
      \rput{30}(0,\displacement){
	\rput{-30}(\hdisplacement,0){
	  \ThreeDput[normal=1 .0 .0](0,0,0){
	    \psframebox*[framearc=.3]{\blue \Large 1}
	  }
	}
      }
    }
    \rput{0}(2,-2){
      \rput{30}(0,\displacement){
	\rput{0}(2,0){
	  \rput{-30}(-\hdisplacement,0){
	    \ThreeDput[normal=1 .0 .0](0,0,0){
	      \psframebox*[framearc=.3]{\blue \Large 0}
	    }
	  }
	}
      }
    }
    \rput{0}(2,2){
      \rput{30}(0,-\displacement){
	\rput{-30}(\hdisplacement,0){
	  \ThreeDput[normal=1 .0 .0](0,0,0){
	    \psframebox*[framearc=.3]{\blue \Large 3}
	  }
	}
      }
    }
    \rput{30}(0,2){
      \rput{-30}(1,0){
	\ThreeDput[normal=.0 .0 1](0,0,0){
	  \psframebox*[framearc=.3]{\green \Huge 4}
	}
      }
    }
    \rput{0}(2,2){
      \ThreeDput[normal=.0 .0 1](0,0,0){
	\rput{0}(-0.4,0.2){
	  \psframebox*[framearc=.3]{\green \Large 3}
	}
      }
    }
    \rput{0}(-2,2){
      \ThreeDput[normal=.0 .0 1](0,0,0){
	\rput{0}(0.4,1){
	  \psframebox*[framearc=.3]{\green \Large 2}
	}
      }
    }
    \rput{30}(2,2){
      \rput{-30}(2,0){
	\ThreeDput[normal=.0 .0 1](0,0,0){
	  \rput{0}(-0.3,-0.8){
	    \psframebox*[framearc=.3]{\green \Large 1}
	  }
	}
      }
    }
    \rput{30}(-2,2){
      \rput{-30}(2,0){
	\ThreeDput[normal=.0 .0 1](0,0,0){
	  \rput{0}(0.4,-0.2){
	    \psframebox*[framearc=.3]{\green \Large 0}
	  }
	}
      }
    }
    \uput[45]{0}(-2,-2){\psframebox*[framearc=.3]{\red \Large 0}}
    \uput[-45]{0}(-2, 2){\psframebox*[framearc=.3]{\red \Large 1}}
    \uput[-135]{0}( 2, 2){\psframebox*[framearc=.3]{\red \Large 3}}
    \uput[135]{0}( 2,-2){\psframebox*[framearc=.3]{\red \Large 2}}
    \endpspicture &
%
    \pspicture[](-3,-2.5)(5,4.5)
    \psset{unit=1cm,subgriddiv=0,arrowsize=3pt 3,linewidth=1pt,
      viewpoint=-1 -1 -1
    }
    \cnode*(-2,-2){2pt}{0}
    \cnode*(-2, 2){2pt}{1}
    \cnode*( 2, 2){2pt}{2}
    \cnode*( 2,-2){2pt}{3}
    \rput{30}(0,0){
      \rput{-30}(2,0){
	\cnode(-2,-2){2pt}{4}
	\cnode*(-2, 2){2pt}{5}
	\cnode*( 2, 2){2pt}{6}
	\cnode*( 2,-2){2pt}{7}
      }
    }
    \rput{30}(2,-2){
      \rput{-30}(\hdisplacement,0){
	\rput{0}(0,\displacement){
	  \pnode(0,0){a1}
	  \pnode(0,5){b1}
	  \uput[90]{0}(0,5){$Y$}
	  \rput{30}(0,0){
	    \pnode(3,0){c1}
	    \uput[0]{-30}(3,0){$X$}
	  }
	}
      }
    }
    \rput{0}(2,2){
      \rput{30}(-\displacement,0){
	\rput{-30}(\hdisplacement,0){
	  \pnode(0,0){a2}
	  \pnode(-5,0){b2}
	  \uput[180]{0}(-5,0){$Y$}
	  \rput{30}(0,0){
	    \pnode(3,0){c2}
	    \uput[0]{-30}(3,0){$X$}
	  }
	}
      }
    }
    \ncline{0}{1}
    \ncline{1}{2}
    \ncline{2}{3}
    \ncline{3}{0}
    \ncline[linecolor=gray,linestyle=dashed]{4}{5}
    \ncline{5}{6}
    \ncline{6}{7}
    \ncline[linecolor=gray,linestyle=dashed]{7}{4}
    \ncline[linecolor=gray,linestyle=dashed]{0}{4}
    \ncline{1}{5}
    \ncline{2}{6}
    \ncline{3}{7}
    \ncline[linecolor=cyan]{->}{a1}{b1}
    \ncline[linecolor=cyan]{->}{a1}{c1}
    \ncline[linecolor=green]{->}{a2}{b2}
    \ncline[linecolor=green]{->}{a2}{c2}
    \rput{0}(2,-2){
      \rput{0}(-\displacement,\displacement){
	\psline[linecolor=magenta]{->}(0,0)(-5,0)
	\psline[linecolor=magenta]{->}(0,0)(0,5)
	\uput[180]{0}(-5,0){$Y$}
	\uput[90]{0}(0,5){$X$}
      }
    }
    \rput{30}(0,0){
      \rput{-30}(1,0){
	\my_axis{180}{1}{90}{1}{30}{0.7}
      }
    }
    \rput{0}(0,0){\psframebox*[framearc=.3]{\magenta \Huge 2}}
    \rput{30}(2,0){
      \rput{-30}(1,0){
	\ThreeDput[normal=-1 .0 .0](0,0,0){
	  \psframebox*[framearc=.3]{\cyan \Huge 0}
	}
      }
    }
    \rput{0}(2,-2){
      \rput{30}(0,\displacement){
	\rput{-30}(\hdisplacement,0){
	  \ThreeDput[normal=-1 .0 .0](0,0,0){
	    \psframebox*[framearc=.3]{\cyan \Large 0}
	  }
	}
      }
    }
    \rput{0}(2,-2){
      \rput{30}(0,\displacement){
	\rput{0}(2,0){
	  \rput{-30}(-\hdisplacement,0){
	    \ThreeDput[normal=-1 .0 .0](0,0,0){
	      \psframebox*[framearc=.3]{\cyan \Large 2}
	    }
	  }
	}
      }
    }
    \rput{0}(2,2){
      \rput{30}(0,-\displacement){
	\rput{-30}(\hdisplacement,0){
	  \ThreeDput[normal=-1 .0 .0](0,0,0){
	    \psframebox*[framearc=.3]{\cyan \Large 1}
	  }
	}
      }
    }
    \rput{0}(2,2){
      \rput{30}(0,-\displacement){
	\rput{0}(2,0){
	  \rput{-30}(-\hdisplacement,0){
	    \ThreeDput[normal=-1 .0 .0](0,0,0){
	      \psframebox*[framearc=.3]{\cyan \Large 3}
	    }
	  }
	}
      }
    }
    \rput{30}(0,2){
      \rput{-30}(1,0){
	\ThreeDput[normal=.0 .0 -1](0,0,0){
	  \psframebox*[framearc=.3]{\green \Huge 4}
	}
      }
    }
    \uput[45]{0}(-2,-2){\psframebox*[framearc=.3]{\magenta \Large 1}}
    \uput[-45]{0}(-2, 2){\psframebox*[framearc=.3]{\magenta \Large 3}}
    \uput[-135]{0}( 2, 2){\psframebox*[framearc=.3]{\magenta \Large 2}}
    \uput[135]{0}( 2,-2){\psframebox*[framearc=.3]{\magenta \Large 0}}
    \endpspicture \\
%
    \pspicture[](-3,-3.5)(4.5,4)
    \psset{unit=1cm,subgriddiv=0,arrowsize=3pt 3,linewidth=1pt,
      viewpoint=-1 1 -1
    }
    \cnode*(-2,-2){2pt}{0}
    \cnode*(-2, 2){2pt}{1}
    \cnode*( 2, 2){2pt}{2}
    \cnode*( 2,-2){2pt}{3}
    \rput{210}(0,0){
      \rput{-210}(2,0){
	\cnode(-2,-2){2pt}{4}
	\cnode*(-2, 2){2pt}{5}
	\cnode*( 2, 2){2pt}{6}
	\cnode*( 2,-2){2pt}{7}
      }
    }
    \rput{210}(-2,-2){
      \rput{0}(2,0){
	\rput{-210}(-\hdisplacement,0){
	  \rput{0}(0,\displacement){
	    \pnode(0,0){a1}
	    \pnode(0,5){b1}
	    \uput[90]{0}(0,5){$Y$}
	    \rput{210}(0,0){
	      \pnode(-3,0){c1}
	      \uput[180]{-210}(-3,0){$X$}
	    }
	  }
	}
      }
    }
    \rput{0}(-2,-2){
      \rput{210}(\displacement,0){
	\rput{0}(2,0){
	  \rput{-210}(-\hdisplacement,0){
	    \pnode(0,0){a2}
	    \pnode(5,0){b2}
	    \uput[0]{0}(5,0){$X$}
	    \rput{210}(0,0){
	      \pnode(-3,0){c2}
	      \uput[180]{-210}(-3,0){$Y$}
	    }
	  }
	}
      }
    }
    \ncline{0}{1}
    \ncline{1}{2}
    \ncline{2}{3}
    \ncline{3}{0}
    \ncline{4}{5}
    \ncline[linecolor=gray,linestyle=dashed]{5}{6}
    \ncline[linecolor=gray,linestyle=dashed]{6}{7}
    \ncline{7}{4}
    \ncline{0}{4}
    \ncline{1}{5}
    \ncline[linecolor=gray,linestyle=dashed]{2}{6}
    \ncline{3}{7}
    \ncline[linecolor=cyan]{->}{a1}{b1}
    \ncline[linecolor=cyan]{->}{a1}{c1}
    \ncline[linecolor=yellow]{->}{a2}{b2}
    \ncline[linecolor=yellow]{->}{a2}{c2}
    \rput{0}(\displacement,\displacement){
      \psline[linecolor=red]{->}(-2,-2)(3,-2)
      \psline[linecolor=red]{->}(-2,-2)(-2,3)
      \uput[0]{0}(3,-2){$X$}
      \uput[90]{0}(-2,3){$Y$}
    }
    \rput{210}(0,0){
      \rput{-210}(1,0){
	\my_axis{0}{1}{90}{1}{30}{0.7}
      }
    }
    \rput{0}(1,0.5){\psframebox*[framearc=.3]{\red \Huge 5}}
    \rput{210}(-2,0){
      \rput{-210}(1,0){
	\ThreeDput[normal=-1 .0 .0](0,0,0){
	  \psframebox*[framearc=.3]{\cyan \Huge 0}
	}
      }
    }
    \rput{210}(0,-2){
      \rput{-210}(1,0){
	\psframebox*[framearc=.3]{\yellow \Huge 1}
      }
    }
    \uput[45]{0}(-2,-2){\psframebox*[framearc=.3]{\red \Large 0}}
    \uput[-45]{0}(-2, 2){\psframebox*[framearc=.3]{\red \Large 1}}
    \uput[-135]{0}( 2, 2){\psframebox*[framearc=.3]{\red \Large 3}}
    \uput[135]{0}( 2,-2){\psframebox*[framearc=.3]{\red \Large 2}}
    \endpspicture &
%
    \pspicture[](-3,-3.5)(4.5,4)
    \psset{unit=1cm,subgriddiv=0,arrowsize=3pt 3,linewidth=1pt,
      viewpoint=1 1 -1
    }
    \cnode*(-2,-2){2pt}{0}
    \cnode*(-2, 2){2pt}{1}
    \cnode*( 2, 2){2pt}{2}
    \cnode*( 2,-2){2pt}{3}
    \rput{210}(0,0){
      \rput{-210}(2,0){
	\cnode(-2,-2){2pt}{4}
	\cnode*(-2, 2){2pt}{5}
	\cnode*( 2, 2){2pt}{6}
	\cnode*( 2,-2){2pt}{7}
      }
    }
    \rput{0}(-2,-2){
      \rput{210}(0,\displacement){
	\rput{-210}(\hdisplacement,0){
	  \pnode(0,0){a1}
	  \pnode(0,5){b1}
	  \uput[90]{0}(0,5){$X$}
	  \rput{210}(0,0){
	    \pnode(3,0){c1}
	    \uput[0]{-210}(3,0){$Y$}
	  }
	}
      }
    }
    \rput{0}(2,-2){
      \rput{210}(-\displacement,0){
	\rput{-210}(\hdisplacement,0){
	  \pnode(0,0){a2}
	  \pnode(-5,0){b2}
	  \uput[180]{0}(-5,0){$X$}
	  \rput{210}(0,0){
	    \pnode(3,0){c2}
	    \uput[0]{-210}(3,0){$Y$}
	  }
	}
      }
    }
    \ncline{0}{1}
    \ncline{1}{2}
    \ncline{2}{3}
    \ncline{3}{0}
    \ncline{4}{5}
    \ncline[linecolor=gray,linestyle=dashed]{5}{6}
    \ncline[linecolor=gray,linestyle=dashed]{6}{7}
    \ncline{7}{4}
    \ncline{0}{4}
    \ncline{1}{5}
    \ncline[linecolor=gray,linestyle=dashed]{2}{6}
    \ncline{3}{7}
    \ncline[linecolor=blue]{->}{a1}{b1}
    \ncline[linecolor=blue]{->}{a1}{c1}
    \ncline[linecolor=yellow]{->}{a2}{b2}
    \ncline[linecolor=yellow]{->}{a2}{c2}
    \rput{0}(-\displacement,\displacement){
      \psline[linecolor=magenta]{->}(2,-2)(-3,-2)
      \psline[linecolor=magenta]{->}(2,-2)(2,3)
      \uput[180]{0}(-3,-2){$Y$}
      \uput[90]{0}(2,3){$X$}
    }
    \rput{210}(0,0){
      \rput{-210}(1,0){
	\my_axis{180}{1}{90}{1}{210}{0.7}
      }
    }
    \rput{0}(.0,.0){\psframebox*[framearc=.3]{\magenta \Huge 2}}
    \rput{210}(-2,0){
      \rput{-210}(1,0){
	\ThreeDput[normal=1 .0 .0](0,0,0){
	  \psframebox*[framearc=.3]{\blue \Huge 3}
	}
      }
    }
    \rput{210}(0,-2){
      \rput{-210}(1,0){
	\psframebox*[framearc=.3]{\yellow \Huge 1}
      }
    }
   \uput[45]{0}(-2,-2){\psframebox*[framearc=.3]{\magenta \Large 1}}
    \uput[-45]{0}(-2, 2){\psframebox*[framearc=.3]{\magenta \Large 3}}
    \uput[-135]{0}( 2, 2){\psframebox*[framearc=.3]{\magenta \Large 2}}
    \uput[135]{0}( 2,-2){\psframebox*[framearc=.3]{\magenta \Large 0}}
    \endpspicture
  \end{tabular}

  }
  \caption{\sf The data structure.}
  \vspace{-2ex}
  \label{fig:data_struct}
\end{wrapfigure}
