%% Copyright (c) 2005  Foundation for Research and Technology-Hellas (Greece).
%% All rights reserved.
%%
%% This file is part of CGAL (www.cgal.org).
%% You can redistribute it and/or modify it under the terms of the GNU
%% General Public License as published by the Free Software Foundation,
%% either version 3 of the License, or (at your option) any later version.
%%
%% Licensees holding a valid commercial license may use this file in
%% accordance with the commercial license agreement provided with the software.
%%
%% This file is provided AS IS with NO WARRANTY OF ANY KIND, INCLUDING THE
%% WARRANTY OF DESIGN, MERCHANTABILITY AND FITNESS FOR A PARTICULAR PURPOSE.
%%
%% $URL$
%% $Id$
%% 
%%
%% Author(s)     : Menelaos Karavelas <mkaravel@iacm.forth.gr>


\begin{ccRefConcept}{DelaunayGraph_2}

%% \ccHtmlCrossLink{}     %% add further rules for cross referencing links
%% \ccHtmlIndexC[concept]{} %% add further index entries
\ccDefinition

The concept \ccc{DelaunayGraph_2} defines the requirements for the
first template parameter of the \ccc{Voronoi_diagram_2<DG,AT,AP>}
class. The \ccc{DelaunayGraph_2} concept essentially defines the
requirements that a class representing a Delaunay graph must obey so
that the Voronoi diagram adaptor can adapt it.

\ccRefines
\ccc{DefaultConstructible}, \ccc{CopyConstructible}, \ccc{Assignable}

\ccTypes
%\ccTwo{VoronoiDiagram_2::Triangulation_data_structure+}{}
\ccThree{typedef std::pair<Face_handle,int>+++}{Edge+}{}
\ccThreeToTwo
%
\ccNestedType{size_type}{A type for sizes.}
\ccGlue
\ccNestedType{Geom_traits}{A type for the geometric traits associated
  with the Delaunay graph.}
%\ccGlue
%\ccNestedType{Point_2}{A type for a point.}
%\ccGlue
%\ccNestedType{Site_2}{A type for the sites of the Voronoi diagram.}
\ccGlue
\ccNestedType{Triangulation_data_structure}{A type for the underlying
  triangulation data structure. It must be a model of the concept
  \ccc{TriangulationDataStructure_2.}}
\ccGlue
\ccNestedType{Vertex}{A type for the vertices of the Delaunay graph.}
\ccGlue
\ccNestedType{Face}{A type for the faces of the Delaunay graph.}
\ccGlue
\ccTypedef{typedef std::pair<Face_handle,int> Edge;}{The type of the
  edges of the Delaunay graph.}
\ccGlue
\ccNestedType{Vertex_handle}{Handle to the vertices of the Delaunay graph.}
\ccGlue
\ccNestedType{Face_handle}{Handle to the faces of the Delaunay graph.}

The following iterators and circulators must be defined. All iterators
and circulators must be assignable and convertible to their
corresponding handles.

\ccNestedType{All_edges_iterator}{A type for an iterator over all edges of
  the Delaunay graph. Its value type must be \ccc{Edge}.}
\ccGlue
\ccNestedType{Finite_edges_iterator}{A type for an iterator over the finite
  edges of the Delaunay graph. Its value type must be \ccc{Edge}.}
\ccGlue
\ccNestedType{All_faces_iterator}{A type for an iterator over all faces of
  the Delaunay graph. Its value type must be \ccc{Face}.}
\ccGlue
\ccNestedType{Finite_faces_iterator}{A type for an iterator over the finite
  faces of the Delaunay graph. Its value type must be \ccc{Face}.}
\ccGlue
\ccNestedType{All_vertices_iterator}{A type for an iterator over all
  vertices of the Delaunay graph. Its value type must be \ccc{Vertex}.}
\ccGlue
\ccNestedType{Finite_vertices_iterator}{A type for an iterator over
  the finite vertices of the Delaunay graph. Its value type must be
  \ccc{Vertex}.}
\ccGlue
\ccNestedType{Face_circulator}{A type for a circulator over the
  adjacent faces of a vertex of the Delaunay graph. Its value type
  must be \ccc{Face}.}
\ccGlue
\ccNestedType{Vertex_circulator}{A type for a circulator over the
  adjacent vertices of a vertex of the Delaunay graph. Its value type
  must be \ccc{Vertex}.}
\ccGlue
\ccNestedType{Edge_circulator}{A type for a circulator over the
  adjacent edges of a vertex of the Delaunay graph. Its value type
  must be \ccc{Edge}.}


%
%\ccTwo{DelaunayGraph_2}{}
% 

\ccCreationVariable{dg}  %% choose variable name

\ccThree{DelaunayGraph_2}{dg(It begin, It beyond, Geom_traits gt)+}{}
\ccThreeToTwo

\ccCreation
In addition to the default and copy constructors, as well as the assignment
operator, the following constructors are required.

\ccConstructor{DelaunayGraph_2(Geom_traits gt);}
{Constructor that takes an instance of the geometric traits.}
\ccGlue
\ccConstructor{template<class It> DelaunayGraph_2(It first, It beyond);}
{Constructor that takes an iterator range. The value type of the
  iterator must be the type of the sites of the Delaunay graph.}
\ccGlue
\ccConstructor{template<class It> DelaunayGraph_2(It first, It beyond,
  Geom_traits gt);}
{Constructor that takes an iterator range and an instance of the
  geometric traits. The value type of the iterator must be the type of
  the sites of the Delaunay graph.}


\ccHeading{Access methods}
\ccThree{Triangulation_data_structure}{dg.number_of_vertices()+}{}
\ccMethod{Triangulation_data_structure tds();}
{Returns a reference to the underlying triangulation data structure.}
\ccGlue
\ccMethod{Geom_traits geom_traits();}
{Returns a reference to the geometric traits object.}
\ccGlue
\ccMethod{Vertex_handle infinite_vertex();}
{Returns a handle to the infinite vertex.}
\ccGlue
\ccMethod{Vertex_handle finite_vertex();}
{Returns a handle to a finite vertex, provided there exists one.}
\ccGlue
\ccMethod{Face_handle infinite_face();}
{Returns a handle to a face incident to the infinite vertex.}
\ccGlue
\ccMethod{int dimension();}
{Returns the dimension of the Delaunay graph.}
\ccGlue
\ccMethod{size_type number_of_vertices();}
{Returns the number of finite vertices.}
\ccGlue
\ccMethod{size_type number_of_faces();}
{Returns the number of faces (both finite and infinite).}


\ccHeading{Traversal of the Delaunay graph}

A model of the \ccc{DelaunayGraph_2} concept must provide several
iterators and circulators that allow to traverse it (completely or
partially). All iterators and circulators must be convertible to the
corresponding handles.


\ccHeading{Face, Edge and Vertex Iterators}

\ccThree{Finite_vertices_iterator}{dg.finite_vertices_begin()+}{}

The following iterators must allow, respectively, to visit 
finite faces,  finite edges and  finite vertices
of the Delaunay graph. These iterators must be non-mutable,
bidirectional and their value types are respectively
\ccc{Face}, \ccc{Edge} and \ccc{Vertex}.
%They are all invalidated by any change in the Delaunay graph.

\ccMethod{Finite_vertices_iterator finite_vertices_begin();}
{Starts at an arbitrary finite vertex.}
\ccGlue
\ccMethod{Finite_vertices_iterator finite_vertices_end();}
{Past-the-end iterator.}

\ccMethod{Finite_edges_iterator finite_edges_begin();}
{Starts at an arbitrary finite edge.}
\ccGlue
\ccMethod{Finite_edges_iterator finite_edges_end();}
{Past-the-end iterator.}

\ccMethod{Finite_faces_iterator finite_faces_begin();}
{Starts at an arbitrary finite face.}
\ccGlue
\ccMethod{Finite_faces_iterator finite_faces_end()
const;}{Past-the-end iterator.}

The following iterators must allow, respectively, to visit all
(both finite and infinite) faces, edges and vertices
of the Delaunay graph. These iterators are non-mutable, bidirectional
and their value types are respectively
\ccc{Face}, \ccc{Edge} and \ccc{Vertex}. 
%They are all invalidated by any change in the Delaunay graph.

\ccMethod{All_vertices_iterator all_vertices_begin();}
{Starts at an arbitrary  vertex.}
\ccGlue
\ccMethod{All_vertices_iterator all_vertices_end();}
{Past-the-end iterator.}

\ccMethod{All_edges_iterator all_edges_begin();}
{Starts at an arbitrary edge.}
\ccGlue
\ccMethod{All_edges_iterator all_edges_end();}
{Past-the-end iterator.}

\ccMethod{All_faces_iterator all_faces_begin();}
{Starts at an arbitrary face.}
\ccGlue
\ccMethod{All_faces_iterator all_faces_end();}
{Past-the-end iterator.}


\ccThree{Vertex_circulator}
{dg.incident_vertices(Vertex_handle v)+++}{}
\ccThreeToTwo


\ccHeading{Face, Edge and Vertex Circulators}

A model of the \ccc{DelaunayGraph_2} concept must also provide
circulators that allow to visit, respectively, all faces or edges
incident to a given vertex or all vertices adjacent to a given
vertex. These circulators are non-mutable and bidirectional. The
operator \ccc{operator++} must move the circulator counterclockwise
around the vertex while the \ccc{operator--} must move the circulator
clockwise.
%A face circulator is invalidated by any modification of the face
%pointed to.
%An edge circulator is invalidated by any modification
%in one of the two faces incident to the edge pointed to.
%A vertex circulator is invalidated by any modification
%in any of the faces adjacent to the vertex pointed to.

\ccMethod{Face_circulator incident_faces(Vertex_handle v);}
{Starts at an arbitrary face incident to \ccc{v}.}
\ccGlue
\ccMethod{Face_circulator incident_faces(Vertex_handle v, Face_handle f);}
{Starts at face \ccc{f}.
\ccPrecond Face \ccc{f} must be incident to vertex \ccc{v}.}
%\ccGlue
\ccMethod{Edge_circulator incident_edges(Vertex_handle v);}
{Starts at an arbitrary edge incident to \ccc{v}.}
\ccGlue
\ccMethod{Edge_circulator incident_edges(Vertex_handle v, Face_handle f);}
{Starts at the first edge of \ccc{f} incident to \ccc{v}, in
  counterclockwise order around \ccc{v}.
\ccPrecond Face \ccc{f} must be incident to vertex \ccc{v}.}
%\ccGlue
\ccMethod{Vertex_circulator incident_vertices(Vertex_handle v);}
{Starts at an arbitrary  vertex incident to \ccc{v}.}
\ccGlue
\ccMethod{Vertex_circulator incident_vertices(Vertex_handle v, Face_handle f);}
{Starts at the first vertex of \ccc{f} adjacent  to \ccc{v} in
  counterclockwise order around \ccc{v}.
\ccPrecond Face \ccc{f} must be incident to vertex \ccc{v}.}



\ccPredicates
A model of the \ccc{DelaunayGraph_2} concept must provide methods to test
the finite or infinite character of any feature.
\ccThree{bool }{dg.is_infinite(Face_handle f, int i)+}{}

%
\ccMethod{bool is_infinite(Vertex_handle v);}
{\ccc{true}, iff \ccc{v} is the \ccc{infinite_vertex}.}
\ccGlue
\ccMethod{bool is_infinite(Face_handle f);}
{\ccc{true}, iff face \ccc{f} is infinite.}
\ccGlue
\ccMethod{bool is_infinite(Face_handle f, int i);}
{\ccc{true}, iff edge \ccc{(f,i)} is infinite.}
\ccGlue
\ccMethod{bool is_infinite(Edge e);}
{\ccc{true}, iff edge \ccc{e} is infinite.}
\ccGlue
\ccMethod{bool is_infinite(Edge_circulator ec);}
{\ccc{true}, iff edge \ccc{*ec} is infinite.}



%\ccHeading{Insertion}
%\ccThree{Vertex_handle }{dg.insert(Site_2 s)+}{}
%%
%The following insertion methods are in some sense optional. They are
%only required if in the corresponding Voronoi traits the
%\ccc{Has_insert} tag is set to \ccc{CGAL::Tag_true}.
%
%\ccMethod{Vertex_handle  insert(Site_2 s);}{Inserts the
%site \ccc{s} in the Delaunay graph.}
%%
%\ccMethod{template<class Iterator>
%size_type  insert(Iterator first, Iterator beyond);}{Inserts the
%sites in the iterator range \ccc{[first,beyond)} in the Delaunay graph.
%The \ccc{Iterator} type must be a model of \ccc{OutputIterator} and its
%value type must be \ccc{Site_2}}.


%\ccHeading{Nearest neighbor location}
%\ccThree{Vertex_handle }{dg.nearest_neighbor(Point_2 p)+}{}
%%
%\ccMethod{Vertex_handle  nearest_neighbor(Point_2 p);}
%{Finds the nearest neighbor of the point \ccc{p}. In other words it
%finds the site whose Voronoi cell contains \ccc{p}. Ties are broken
%arbitrarily and one of the nearest neighbors of \ccc{p} is
%returned. If the number of vertices of the Delaunay graph is zero
%\ccc{Vertex_handle(NULL)} should be returned.}
%\ccGlue
%\ccMethod{Vertex_handle  nearest_neighbor(Point_2 p,
%Vertex_handle vnear);}{Finds the nearest neighbor of the point
%\ccc{p} using the site associated with \ccc{vnear} as an
%estimate for the nearest neighbor of \ccc{p}. Ties are broken
%arbitrarily and one of the nearest neighbors of \ccc{p} is
%returned. If the number of vertices of the Delaunay graph is zero
%\ccc{Vertex_handle(NULL)} is returned.}


\ccHeading{Validity check}
\ccThree{bool}{dg.is_valid(bool verbose = false)+}{}
%
\ccMethod{bool is_valid(bool verbose = false);}
{Checks the validity of the Delaunay graph. If \ccc{verbose} is
\ccc{true} a short message is sent to \ccc{std::cerr}.}

\ccHeading{Miscellaneous}
\ccThree{bool}{dg.swap(other)+}{}
%
\ccMethod{void clear();}{Clears all contents of the Delaunay graph.}
\ccGlue
\ccMethod{void swap(DelaunayGraph_2 other);}{The Delaunay graphs
  \ccc{other} and \ccVar\ are swapped. \ccVar.\ccc{swap(other)}
  should be preferred to \ccVar\ccc{ = other} or to
  \ccVar\ccc{(other)} if \ccc{other} is deleted afterwards.}


\ccHasModels
\ccc{CGAL::Delaunay_triangulation_2<Traits,Tds>}\\
\ccc{CGAL::Regular_triangulation_2<Traits,Tds>}\\
\ccc{CGAL::Triangulation_hierarchy_2<Tr>} provided that \ccc{Tr} is a
model of \ccc{DelaunayGraph_2}\\
\ccc{CGAL::Segment_Delaunay_graph_2<Gt,DS>}\\
\ccc{CGAL::Segment_Delaunay_graph_hierarchy_2<Gt,STag,DS>}\\
\ccc{CGAL::Apollonius_graph_2<Gt,Agds>}\\
\ccc{CGAL::Apollonius_graph_hierarchy_2<Gt,Agds>}\\

\ccSeeAlso
\ccc{AdaptationTraits_2}\\
\ccc{AdaptationPolicy_2}\\
\ccc{CGAL::Voronoi_diagram_2<DG,AT,AP>}
\end{ccRefConcept}

% +------------------------------------------------------------------------+
%%RefPage: end of main body, begin of footer
% EOF
% +------------------------------------------------------------------------+

