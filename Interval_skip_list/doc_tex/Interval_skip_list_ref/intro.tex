% ======================================================================
%
% Copyright (c) 2003 GeometryFactory
%
% This software and related documentation is part of the
% Computational Geometry Algorithms Library (CGAL).
%
% Every use of CGAL requires a license. Licenses come in three kinds:
%
% - For academic research and teaching purposes, permission to use and
%   copy the software and its documentation is hereby granted free of  
%   charge, provided that
%   (1) it is not a component of a commercial product, and
%   (2) this notice appears in all copies of the software and
%       related documentation.
% - Development licenses grant access to the source code of the library 
%   to develop programs. These programs may be sold to other parties as 
%   executable code. To obtain a development license, please contact
%   the GALIA Consortium (at cgal@cs.uu.nl).
% - Commercialization licenses grant access to the source code and the
%   right to sell development licenses. To obtain a commercialization 
%   license, please contact the GALIA Consortium (at cgal@cs.uu.nl).
%
% This software and documentation is provided "as-is" and without
% warranty of any kind. In no event shall the CGAL Consortium be
% liable for any damage of any kind.
%
%
% ----------------------------------------------------------------------
%
% package       : Interval_skip_list
% author(s)     : Andreas Fabri <Andreas.Fabri@geometryfactory.com>
%
% coordinator   : GeometryFactory (<Andreas.Fabri@geometryfactory.com>)
%
% ======================================================================

\RCSdef{\intervalskiplistRevision}{$Id$}
\RCSdefDate{\intervalskiplistDate}{$Date$}

%----------------------------------------------------------------------

%\clearpage
%\section{Reference Pages for Interval Skip List}

\ccRefChapter{Interval Skip List\label{chap:interval_skip_list_ref}}

\ccChapterAuthor{Andreas Fabri}

This chapter presents the interval skip list introduced by Hanson~\cite{h-islds-91},
and derived from the skip list data structure~\cite{p-slpab-90}.

The data structure stores intervals and allows to perform stabbing queries,
that is to test whether a point is covered by any of the intervals.
It further allows to find all intervals that contain a point.

The interval skip list is, as far as its functionality is concerned,
related to the \ccc{Segment_tree}. Both allow to do stabbing queries
and both allow to find all intervals that contain a given point.  The
implementation of segment trees in \cgal\ works in higher
dimensions, whereas the interval skip list is limited to the 1D
case. However, this interval skip list implementation is fully
dynamic, whereas the segment tree implementation in \cgal\ is
static, that is all intervals must be known in advance.

This package has one concept, namely for the interval with which 
the interval skip list class is parameterized.



\section{Classified Reference Pages}

\subsection*{Concepts}

\ccRefConceptPage{Interval}\\


\subsection*{Classes}

\ccRefIdfierPage{CGAL::Interval_skip_list<Interval>}\\
\ccRefIdfierPage{CGAL::Interval_skip_list_interval<Value>}\\
\ccRefIdfierPage{CGAL::Level_interval<FaceHandle>}\\

