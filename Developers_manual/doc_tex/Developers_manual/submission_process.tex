% =============================================================================
% The CGAL Developers' Manual
% Chapter: Submitting Packages
% -----------------------------------------------------------------------------
% file   : submission_process.tex
% authors: Geert-Jan Giezeman <geert@cs.uu.nl> & Susan Hert <hert@mpi-sb.mpg.de>
% -----------------------------------------------------------------------------
% $Id$
% $Date$
% =============================================================================

\chapter{Editorial Board\label{chap:submission}\label{sec:editorial_board}}
\ccIndexMainItemBegin{editorial board}


The editorial board is in charge of approving the inclusion of new packages
in the library.  This means that they assure that new contributions
\begin{itemize}
   \item are in keeping with the philosophy of \cgal\ (Chapter~\ref{chap:intro});
   \item are generic and fit seamlessly with other parts of the library;
   \item satisfy the coding conventions of \cgal\ (Chapter~\ref{chap:code_format});
   \item carefully and efficiently treat robustness issues
         (Chapter~\ref{chap:robustness});
   \item are designed in a flexible, extensible, and easy-to-use fashion;
   \item and are designed to be technically feasible for the platforms
         supported by \cgal.
\end{itemize}

Software specifications and implementations should be submitted to the
editorial board for approval. The process is described in the 
\ccAnchor{http://www.cgal.org/review_process_rules.html}{submission
  and review rules}. 

% This can be done by sending mail to
% the \ccAnchor{mailto:cgal-editorial-board@lists-sop.inria.fr}{board}
% \lcTex{(\texttt{cgal-editorial-board@lists-sop.inria.fr})} indicating
% where the (PDF) documentation and code can be found.  After
% some reasonable amount of time, you should receive feedback from
% the board about the specification and what, if anything, needs to
% be changed. The usual procedure is that someone from the board is
% assigned to be (or volunteers to be) the primary reviewer and sends
% comments on the submitted package to the board and to the authors of
% the package.  Discussion then proceeds among the board members and the
% authors until a consensus is reached about how the package should be
% modified before being accepted.  When the package has been modified,
% the authors should again notify the editorial board to let them
% know what has changed so a decision about acceptance of the package
% can be taken.

One should write a specification for a new package
and submit it to the editorial board.
% for approval before submitting the package for inclusion in the
% internal releases 
Ideally, this should be done before implementation of the package: This
ensures that time is not wasted in fixing code that may later be changed
due to the recommendations of the board.
% However, since it can take some time for the board to process
% submissions, packages that are to become part of the library
% can be submitted
% before approval.
% Inclusion in an internal release does not ensure inclusion in a public
% release.  
Only after approval by the board will packages be included in new
public releases and then only if they pass the test suite, of course.

The current list of members of the editorial board can be found on the 
\ccAnchor{http://www.cgal.org/people.html}{web site}.
