\begin{ccRefClass}{Points_on_segment_2<Point_2>}

\ccDefinition

The class \ccRefName\ is a generator for points on a segment whose
endpoints are specified upon construction.  The points are equally
spaced

\ccIsModel

\ccRefConceptPage{PointGenerator}

\ccTypes

\ccThree{typedef std::input_iterator_tagxxx}{xxxiterator_category;}{}
\ccTypedef{typedef std::input_iterator_tag iterator_category;}{}
\ccGlue
\ccTypedef{typedef Point_2                 value_type;}{}
\ccGlue
\ccTypedef{typedef std::ptrdiff_t          difference_type;}{}
\ccGlue
\ccTypedef{typedef const Point_2*          pointer;}{}
\ccGlue
\ccTypedef{typedef const Point_2&          reference;}{}
%\ccTypedef{typedef Point_2::FT             FT;}{ ??? needed for concept but 
%                                                not actually there.  
%                                                Uses double instead}

\ccCreationVariable{g}
\ccCreation

\ccConstructor{Points_on_segment_2( const Point_2& p, const Point_2& q, 
                                    std::size_t n, std::size_t i = 0);}{%
  $g$ is an input iterator creating points of type \ccc{P} equally 
  spaced on the segment from $p$ to $q$. $n-i$ points are placed on the
  segment defined by $p$ and $q$. Values of the index parameter $i$ larger 
  than 0 indicate starting points for the sequence further from $p$.
  Point $p$ has index value 0 and $q$ has index value $n-1$.
  \ccRequire The expressions \ccc{to_double(p.x())} and
    \ccc{to_double(p.y())} must  result in the respective
    \ccc{double} representation of the coordinates of $p$, and similarly 
    for $q$.}


\ccOperations
\ccThree{Point_2xxx}{xxxg.source();}{}
\ccMethod{double range();}{returns the range in which the point
  coordinates lie, i.e.~$\forall x: |x| \leq$ \ccc{range()} and
  $\forall y: |y| \leq $\ccc{range()}}.

\ccMethod{const Point_2& source();}{returns the source point of the segment.}%
\ccGlue
\ccMethod{const Point_2& target();}{returns the target point of the segment.}

\ccSeeAlso

\ccRefIdfierPage{CGAL::cpp0x::copy_n} \\
\ccRefIdfierPage{CGAL::Counting_iterator} \\
\ccRefIdfierPage{CGAL::points_on_segment<Point_2>} \\
\ccRefIdfierPage{CGAL::Random_points_in_disc_2<Point_2, Creator>} \\
\ccRefIdfierPage{CGAL::Random_points_in_square_2<Point_2, Creator>} \\
\ccRefIdfierPage{CGAL::Random_points_on_circle_2<Point_2, Creator>} \\
\ccRefIdfierPage{CGAL::Random_points_on_segment_2<Point_2, Creator>} \\
\ccRefIdfierPage{CGAL::Random_points_on_square_2<Point_2, Creator>} \\
\ccRefIdfierPage{CGAL::random_selection} \\
\ccc{std::random_shuffle} \\


\end{ccRefClass}

