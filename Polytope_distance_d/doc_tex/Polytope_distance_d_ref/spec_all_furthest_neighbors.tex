%% ==============================================================
%% Specification: All Furthest Neighbors
%% --------------------------------------------------------------
%% file  : spec_all_furthest_neighbors.awi
%% author: Michael Hoffmann
%% $Id$
%% ==============================================================

\cgalColumnLayout

\begin{ccRefFunction}{all_furthest_neighbors_2}
  \ccIndexSubitem[t]{neighbor}{all furthest}
  \ccIndexMainItem[t]{all furthest neighbors}
  \ccIndexSubitem[t]{furthest}{all neighbors}
  
  \ccDefinition The function \ccRefName\ computes all furthest
  neighbors for the vertices of a convex polygon $P$, i.e. for each
  vertex $v$ of $P$ a vertex $f_v$ of $P$ such that the distance
  between $v$ and $f_v$ is maximized.

  \ccInclude{CGAL/all_furthest_neighbors_2.h}

  \def\ccLongParamLayout{\ccTrue} 
  
  \ccGlobalFunction{ template < class RandomAccessIterator, class
    OutputIterator, class Traits > OutputIterator
    all_furthest_neighbors_2( RandomAccessIterator points_begin,
    RandomAccessIterator points_end, OutputIterator o, Traits t =
    Default_traits);}
  
  computes all furthest neighbors for the vertices of the convex
  polygon described by the range [\ccc{points_begin},
  \ccc{points_end}), writes their indices (relative to
  \ccc{points_begin}) to \ccc{o}\footnote{i.e. the furthest neighbor
    of \ccc{points_begin[}i\ccc{]} is \ccc{points_begin[}$i$-th number
    written to \ccc{o}\ccc{]}} and returns the past-the-end iterator
  of this sequence.
  
  \ccPrecond The points denoted by the non-empty range
  [\ccc{points_begin}, \ccc{points_end}) form the boundary of a convex
  polygon $P$ (oriented clock-- or counterclockwise).
  
  The geometric types and operations to be used for the computation
  are specified by the traits class parameter \ccc{t}. This parameter
  can be omitted if \ccc{RandomAccessIterator} refers to a point type
  from a \ccc{Kernel}. In this case, the kernel is used as default
  traits class.
  
  \ccRequire
  \begin{enumerate}
  \item If \ccc{t} is specified explicitly, \ccc{Traits} is a model
    for \ccc{AllFurthestNeighborsTraits_2}.
  \item Value type of \ccc{RandomAccessIterator} is
    \ccc{Traits::Point_2} or -- if \ccc{t} is not specified explicitly
    -- \ccc{K::Point_2} where \ccc{K} is a model for \ccc{Kernel}.
  \item \ccc{OutputIterator} accepts \ccc{int} as value type.
  \end{enumerate}
  
  \ccSeeAlso
  \ccRefConceptPage{AllFurthestNeighborsTraits_2}\\
  \ccRefIdfierPage{CGAL::monotone_matrix_search}
 
  \ccImplementation The implementation uses monotone matrix
  search\cite{akmsw-gamsa-87}. Its runtime complexity is linear in the
  number of vertices of $P$.
  
  \ccExample The following code generates a random convex polygon
  \ccc{p} with ten vertices, computes all furthest neighbors and
  writes the sequence of their indices (relative to
  \ccc{points_begin}) to \ccc{cout} (e.g. a sequence of
  \ccc{4788911224} means the furthest neighbor of
  \ccc{points_begin[0]} is \ccc{points_begin[4]}, the furthest
  neighbor of \ccc{points_begin[1]} is \ccc{points_begin[7]} etc.).
  
  \ccIncludeExampleCode{Polytope_distance_d/all_furthest_neighbors_2.cpp}
\end{ccRefFunction}

\begin{ccRefConcept}{AllFurthestNeighborsTraits_2}
  \ccCreationVariable{t}\ccTagFullDeclarations
  
  \ccDefinition The concept \ccRefName\ defines types and operations
  needed to compute all furthest neighbors for the vertices of a
  convex polygon using the function \ccc{all_furthest_neighbors_2}.
  
  \ccTypes
  
  \ccNestedType{FT}{model for \ccRefConceptPage{FieldNumberType}.}
  
  \ccNestedType{Point_2}{model for
    \ccRefConceptPage{Kernel::Point_2}.}
  
  \ccNestedType{Compute_squared_distance_2}{model for
    \ccRefConceptPage{Kernel::Compute_squared_distance_2}.}
  
  \ccNestedType{Less_xy_2}{model for
    \ccRefConceptPage{Kernel::Less_xy_2}.}
  
  \ccNestedType{Orientation_2}{model for
    \ccRefConceptPage{Kernel::Orientation_2}.}
  
  \ccOperations The following member functions return function objects
  of the types listed above.
  
  \ccGlue\ccMemberFunction{Compute_squared_distance_2
    compute_squared_distance_2_object();}{}
  
  \ccGlue\ccMemberFunction{Less_xy_2 less_xy_2_object();}{}
  
  \ccGlue\ccMemberFunction{Orientation_2 orientation_2_object();}{}
  
  \ccHasModels 
  \ccRefIdfierPage{Cartesian<FieldNumberType>},
  \ccRefIdfierPage{Homogeneous<RingNumberType>},
  \ccRefIdfierPage{Simple_cartesian<FieldNumberType>},
  \ccRefIdfierPage{Simple_homogeneous<RingNumberType>}.

  \ccSeeAlso
  \ccRefIdfierPage{CGAL::all_furthest_neighbors_2}

  \ccHeading{Notes}
  \begin{itemize}
  \item \ccClassName\ccc{::Less_xy_2} and
    \ccClassName\ccc{::Orientation_2} are used for (expensive)
    precondition checking only. Therefore, they need not to be
    specified, in case that precondition checking is disabled.
  \end{itemize}
\end{ccRefConcept}

%% --------------------------------------------------------------
%% EOF spec_all_furthest_neighbors.awi
%% --------------------------------------------------------------
