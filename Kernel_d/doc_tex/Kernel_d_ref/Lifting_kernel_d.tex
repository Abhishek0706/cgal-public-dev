\begin{ccRefClass}{Lifting_kernel_d<Kernel_d>}

\ccInclude{CGAL/Lifting_kernel_d.h}

\ccDefinition
A model for \ccc{Kernel_d} that improves the computation of sequences of
determinantal predicates The template parameter must be itself a model of
\ccc{Kernel_d}, from which \ccRefName~inherits geometric objects and some
predicates.

\ccIsModel
\ccRefConceptPage{Kernel_d}

\ccHeading{Geometric Objects}

\ccNestedType{Point_d}{A \(d\)-dimensional point, on which the last
                       coordinate is called \ccc{lifting}. We call this
                       point \ccc{lifted point}. The \ccc{non-lifted point}
                       consists in the first \(d-1\) coordinates.}

\ccCreationVariable{kernel}

\ccHeading{Predicates}

This kernel redefines the Orientation predicate and introduces the Volume
predicate. These functors take an iterator range as parameter. This
iterator range should contain \(d\) or \(d+1\) points, where \(d\) is the
dimension of these points. In the first case, they return the Orientation
(respectively, the Volume) of the non-lifted points. In the latter case,
they return the Orientation (respectively, the Volume) of the lifted
points.

\ccNestedType{Orientation_d}{Computes the orientation of \(d\) non-lifted
                             (or \(d+1\) lifted) points, exactly as
                             \ccc{Kernel_d::Orientation_d}.}
\ccGlue
\ccNestedType{Volume_d}{Computes a multiple of the Euclidean volume of
                        \(d\) non-lifted (or \(d+1\) lifted) points,
                        exactly as \ccc{Orientation_d} does. The Euclidean
                        volume can be obtained by dividing by \(d!\) (or
                        \((d+1)!\) in the lifted case) the result.}

\ccOperations

This kernel provides the ability of modifying the liftings of points, via
the following static function.

\ccMemberFunction{void set_lifting(Point_d&,FT&)const;}{}

This kernel also introduces a function that returns the function object for the
above intruduced Volume predicate.

\ccMemberFunction{Volume_d orientation_d_object()const;}{}

\ccImplementation

The kernel takes advantage of the computation of sequences of determinantal
predicates where only one row of the determinant is changed. To this end,
it expands (following Laplace or cofactor expansion) the determinants along
this row, and stores the computed minors in a hash table.

Each point is given an index, and tuples of indices are used as hash keys.
The hash table is implemented using \verb+boost::unordered_map+.

All kernel operations are thread-safe.

\ccSeeAlso
\ccRefIdfierPage{CGAL::Cartesian_d<RingNumberType>}\\
\ccRefIdfierPage{CGAL::Homogeneous_d<RingNumberType>}

\end{ccRefClass}
