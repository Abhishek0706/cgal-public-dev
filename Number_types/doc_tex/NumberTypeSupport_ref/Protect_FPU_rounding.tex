%   $Id$
%   $URL$

\begin{ccRefClass}{Protect_FPU_rounding<Protected>}
\label{protect_fpu_rouding}

\ccDefinition

Floating-point arithmetic, as specified by the IEEE-754 standard, allows to use
so-called directed rounding for the following arithmetic operations: addition,
subtraction, multiplication, division and square root.  The default behavior is
that the result of such an arithmetic operation is the closest floating-point
number to the exact real result of the operation (rounding to the nearest).
The other rounding modes are: round towards plus infinity, round towards minus
infinity, and round towards zero.

Interval arithmetic uses such directed rounding modes to offer guaranteed
enclosures for the evaluation of real functions, such as with \cgal's
\ccc{Interval_nt} class.

In order to efficiently evaluate sequences of interval arithmetic operations,
such as a geometric predicate computing for example a determinant, it is
advised to reduce the number of rounding mode changes, which otherwise are
performed for each arithmetic operation.  \cgal\ exploits the fact that it is
possible to compute a sequence of interval arithmetic operations by doing only
one rounding mode change around the whole function evaluation in order to
benefit from this optimization.

The class \ccc{Protect_FPU_rounding} allows to easily benefit from this.
Its constructor saves the current rounding mode in the object, and then sets
the current rounding mode to the value provided as argument to the constructor.
The destructor sets the rounding mode back to the saved value.
This allows to protect a block of code determined by a C++ scope, and have
the destructor take care of restoring the value automatically.

The related class \ccc{Set_ieee_double_precision} allows to similarly protect
a block of code from excess precision on some machines (x86 typically with
the traditional FPU, not the more recent SSE2).  Note that
\ccc{Protect_FPU_rounding_mode}, when changing rounding modes, also sets the precision
to the correct 64 bit precision, hence providing a similar effect to
\ccc{Set_ieee_double_precision}.  This notably affects the \ccc{Residue} class.

Note for Visual C++ 64-bit users: due to a compiler bug, the stack unwinding
process happenning when an exception is thrown does not correctly execute the
rounding mode restoration when the \ccc{Protect_FPU_rounding} object is
destroyed.  Therefore, for this configuration, some explicit code has to be
added.

\ccInclude{CGAL/FPU.h}

\ccParameters

The template parameter \ccc{Protected} is a Boolean parameter, which defaults
to \ccc{true}.  It follows the same parameter of the \ccc{Interval_nt} class.
When it is \ccc{false}, the constructor and the destructor of the class do
nothing (this is meant to be used in a context where you know that the rounding
mode change has been taken care of at a higher level in the call stack.

What follows describes the behavior when the parameter has its default value,
\ccc{true}.

\ccCreationVariable{P}
\ccCreation

\ccConstructor{Protect_FPU_rounding(FPU_CW_t r = CGAL_FE_UPWARD);}
{The current rounding mode is saved in the object, and rounding mode is set to \ccc{r}
which can be any of \ccc{CGAL_FE_TONEAREST}, \ccc{CGAL_FE_TOWARDZERO},
\ccc{CGAL_FE_UPWARD} (the default) and \ccc{CGAL_FE_DOWNWARD}.}
\ccGlue
\ccFunction{void ~Protect_FPU_rounding();}
{The rounding mode is restored to the saved value.}

\ccSeeAlso
\ccRefIdfierPage{CGAL::Set_ieee_double_precision}

\end{ccRefClass}
