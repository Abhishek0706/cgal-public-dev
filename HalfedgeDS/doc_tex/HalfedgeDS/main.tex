% +------------------------------------------------------------------------+
% | CGAL Reference Manual:  hds.tex
% +------------------------------------------------------------------------+
% | Combinatoric and geometry of polyhedral surfaces in 
% | halfedge representation.
% |
% | 11.10.1996   Lutz Kettner
% |              Start rewriting the whole stuff
% | 
\RCSdef{\hdsRev}{$Id$}
\RCSdefDate{\hdsDate}{$Date$}
% +------------------------------------------------------------------------+

\ccParDims

\ccUserChapter{Halfedge Data Structures \label{chapterHalfedgeDS}}

\ccChapterRelease{\hdsRev. \ \hdsDate}
\ccChapterAuthor{Lutz Kettner}

\begin{ccPkgDescription}{3D Skin Surface Meshing \label{Pkg:SkinSurface3}}
\ccPkgHowToCiteCgal{cgal:k-ssm3-12}
  \ccPkgSummary{ %
    This package allows to build a triangular mesh of a skin surface.
    Skin surfaces are used for modeling large molecules in biological
    computing. The surface is defined by a set of balls, representing
    the atoms of the molecule, and a shrink factor that determines the
    size of the smooth patches gluing the balls together.

    The construction of a triangular mesh of a smooth skin surface is
    often necessary for further analysis and for fast visualization.
    This package provides functions to construct a triangular mesh
    approximating the skin surface from a set of balls and a shrink
    factor. It also contains code to subdivide the mesh efficiently.
    % 
  }
%
  \ccPkgIntroducedInCGAL{3.3}
  \ccPkgDependsOn{\ccRef[3D Triangulation]{Pkg:Triangulation3} and \ccRef[3D Polyhedral Surface]{Pkg:Polyhedron}}
  \ccPkgLicense{\ccLicenseGPL}
  \ccPkgIllustration{Skin_surface_3/small.png}{Skin_surface_3/large.png}
\end{ccPkgDescription}


\begin{ccTexOnly}
    \setlength{\unitlength}{1mm}
    \begin{picture}(0,0)(0.0,0.0)
      \put (91,35){% textwidth = 156mm
          \includegraphics[width=65mm]{HalfedgeDS/fig/halfedge}
      }
    \end{picture}\vspace{-4mm}% compensate for some vspace added by picture
\end{ccTexOnly}

\minitoc

% +------------------------------------------------------------------------+
\section{Introduction}

A halfedge data structure (abbreviated as \ccc{HalfedgeDS}, or
\ccc{HDS} for template parameters) is an edge-centered data structure
capable of maintaining incidence information of vertices, edges and
faces, for example for planar maps, polyhedra, or other orientable,
two-dimensional surfaces embedded in arbitrary dimension. Each edge is
decomposed into two halfedges with opposite orientations. One incident
face and one incident vertex are stored in each halfedge.  For each
face and each vertex, one incident halfedge is stored.  Reduced
variants of the halfedge data structure can omit some of these
information, for example the halfedge pointers in faces or the
storage of faces at all.

% \begin{ccTexOnly}
%     \vspace{-4mm}
%     \begin{center}
%       \parbox{0.4\textwidth}{%
%           \includegraphics[width=0.4\textwidth]{HalfedgeDS/fig/halfedge}%
%       }
%     \end{center}
%     \vspace{-3mm}
% \end{ccTexOnly}

\begin{ccHtmlOnly}
    <CENTER>
    <A HREF="fig/halfedge.gif">
        <img src="fig/halfedge_small.gif" alt="Halfedge Diagram"></A><P>
    </CENTER>
\end{ccHtmlOnly}

The halfedge data structure is a combinatorial data structure,
geometric interpretation is added by classes built on top of the
halfedge data structure.  These classes might be more convenient to
use than the halfedge data structure directly, since the halfedge data
structure is meant as an implementation layer.  See for example the
\ccc{CGAL::Polyhedron_3} class in Chapter~\ref{chapterPolyhedron}.

The data structure provided here is also known as the
FE-structure~\cite{w-ebdss-85}, as
halfedges~\cite{m-ism-88,cgal:bfh-mgedm-95} or as the doubly connected edge
list (DCEL)~\cite{bkos-cgaa-97}, although the original reference for
the DCEL~\cite{mp-fitcp-78} describes a different data structure. The
halfedge data structure can also be seen as one of the variants of the
quad-edge data structure~\cite{gs-pmgsc-85}. In general, the quad-edge
data can represent non-orientable 2-manifolds, but the variant here is
restricted to orientable 2-manifolds only. An overview and comparison
of these different data structures together with a thorough
description of the design implemented here can be found
in~\cite{k-ugpdd-99}.  

The design presented here is a revised and incompatible version of the
previous design~\cite{k-ddsps-98} as used in \cgal\ R2.2 and earlier
releases. Files and identifier names are disjoint with the old design
which allows for both versions to co-exists. However, classes using a
halfedge data structure can only use one design. For example the
polyhedral surface \ccc{Polyhedron_3} uses by default the new
design. See Chapter~\ref{chapterPolyhedron} for how to still select
the old implementation.

% +------------------------------------------------------------------------+
\section{Software Design}


  \begin{figure}
\begin{ccTexOnly}
    \begin{center}
      \parbox{0.7\textwidth}{%
          \includegraphics[width=0.7\textwidth]{HalfedgeDS/fig/hds_design}%
      }
    \end{center}
\end{ccTexOnly}

\begin{ccHtmlOnly}
    <CENTER>
    <A NAME="figureHalfedgeDSDesign">
        <img src="fig/hds_design_col.gif"
         alt="Halfedge Data-Structure Design"><BR>
    <P>
    </CENTER>
\end{ccHtmlOnly}
    \caption{Responsibilities of the different layers in the 
             halfedge data-structure design.}
    \label{figureHalfedgeDSDesign}
  \end{figure}

Figure~\ref{figureHalfedgeDSDesign}
illustrates the responsibilities of the three layers of the software
design, with the \ccc{CGAL::Polyhedron_3} as an example for the top
layer.  The items provide the space for the information that is
actually stored, i.e., with member variables and access member
functions in \ccc{Vertex}, \ccc{Halfedge}, and \ccc{Face}
respectively. Halfedges are required to provide a reference to the
next halfedge and to the opposite halfedge.  Optionally they may
provide a reference to the previous halfedge, to the incident vertex,
and to the incident face. Vertices and faces may be empty. Optionally
they may provide a reference to the incident halfedge. The options
mentioned are supported in the halfedge data structure and the
polyhedron, for example, Euler operations update the optional
references if they are present. Furthermore, the item classes can be
extended with arbitrary attributes and member functions, which will be
promoted by inheritance to the actual classes used for the polyhedron.

Vertices, halfedges, and faces are passed as local types of the
\ccc{Items} class to the halfedge data structure and polyhedron.
Implementations for vertices, halfedges and faces are provided that
fulfill the mandatory part of the requirements. They can be used as
base classes for extensions by the user. Richer implementations are
also provided to serve as defaults; for polyhedra they provide all
optional incidences, a three-dimensional point in the vertex type and
a plane equation in the face type.

The \ccc{Halfedge_data_structure}, concept \ccc{HalfedgeDS}, is
responsible for the storage organization of the items. Currently,
implementations using internally a bidirectional list or a
vector are provided. The \ccc{HalfedgeDS} defines the handles and iterators
belonging to the items. These types are promoted to the declaration of
the items themselves and are used there to provide the references to
the incident items. This promotion of types is done with a template
parameter \ccc{Refs} of the item types.  The halfedge data structure
provides member functions to insert and delete items, to traverse all
items, and it gives access to the items.

There are two different models for the \ccc{HalfedgeDS} concept available,
\ccc{HalfedgeDS_list} and \ccc{HalfedgeDS_vector}, and more might come.
Therefore we have kept their interface small and factored out common
functionality into separate helper classes, \ccc{HalfedgeDS_decorator},
\ccc{HalfedgeDS_const_decorator}, and \ccc{HalfedgeDS_items_decorator},
which are not shown in Figure~\ref{figureHalfedgeDSDesign},
but would be placed at the side of the \ccc{HalfedgeDS}
since they broaden that interface but do not hide it.  These helper
classes contain operations that are useful to implement the operations
in the next layer, for example, the polyhedron. They add, for example,
the Euler operations and partial operations from which further Euler
operations can be built, such as inserting an edge into the ring of
edges at a vertex.  Furthermore, the helper classes contain adaptive
functionality.  For example, if the \ccc{prev()} member function is
not provided for halfedges, the \ccc{find_prev()} member function of
a helper class searches in the positive direction along the face for
the previous halfedge. But if the \ccc{prev()} member function is
provided, the \ccc{find_prev()} member function simply calls it. This
distinction is resolved at compile time with a technique called {\em
compile-time tags}, similar to iterator tags in~\cite{cgal:sl-stl-95}.

The \ccc{Polyhedron_3} as an example for the third layer adds the
geometric interpretation, provides an easy-to-use interface of
high-level functions, and unifies the access to the flexibility
provided underneath.  It renames face to facet, which is more common
for three-dimensional surfaces.  The interface is designed to protect
the integrity of the internal representation, the handles stored in
the items can no longer directly be written by the user.  The
polyhedron adds the convenient and efficient circulators, see
\ccc{Circulator}, for accessing the circular sequence of edges
around a vertex or around a facet. To achieve this, the
\ccc{Polyhedron_3} derives new vertices, halfedges and facets from those
provided in \ccc{Items}.  These new items are those actually used in
the \ccc{HalfedgeDS}, which gives us the coherent type
structure in this design, especially if compared to our previous
design.


% +========================================================================+
\section{Example Programs}
% +========================================================================+
\label{sectionHdsExamples}


% +-------------------------------------------------------------+
\subsection{The Default Halfedge Data Structure}

The following example program uses the default halfedge data structure
and the decorator class. The default halfedge data structure uses a
list-based representation. All incidences of the items and a point
type for vertices are defined. The trivial traits class provides the
type used for the point. The program creates a loop, consisting
of two halfedges, one vertex and two faces, and checks its validity.

\begin{ccTexOnly}
    \vspace{-4mm}
    \begin{center}
      \parbox{0.3\textwidth}{%
          \includegraphics[width=0.3\textwidth]{HalfedgeDS/fig/loop}%
      }
    \end{center}
    \vspace{-3mm}
\end{ccTexOnly}

\begin{ccHtmlOnly}
    <CENTER>
      <img src="fig/loop.gif" alt="Loop Example"><P>
    </CENTER>
\end{ccHtmlOnly}

\ccIncludeExampleCode{HalfedgeDS/hds_prog_default.cpp}


% +-------------------------------------------------------------+
\subsection{A Minimal Halfedge Data Structure}

The following program defines a minimal halfedge data structure using
the minimal items class \ccc{CGAL::HalfedgeDS_min_items} and a
list-based halfedge data structure. The result is a data structure
maintaining only halfedges with next and opposite pointers.  No
vertices or faces are stored. The data structure represents an {\em
  undirected graph}.

\ccIncludeExampleCode{HalfedgeDS/hds_prog_graph.cpp}


% +-------------------------------------------------------------+
\subsection{The Default with a Vector Instead of a List}

The default halfedge data structure uses a list internally and the
maximal base classes. We change the list to a vector representation
here. Again, a trivial traits class provides the type used for the
point.  Note that for the vector storage the size of the halfedge data
structure should be reserved beforehand, either with the constructor
as shown in the example or with the \ccc{reserve()} member function.
One can later resize the data structure with further calls to the
\ccc{reserve()} member function, but only if the data structure is 
in a consistent, i.e., {\em valid}, state.

\ccIncludeExampleCode{HalfedgeDS/hds_prog_vector.cpp}


% +-------------------------------------------------------------+
\subsection{Example Adding Color to Faces}

This example re-uses the base class available for faces and adds a
member variable \ccc{color}.

\ccIncludeExampleCode{HalfedgeDS/hds_prog_color.cpp}

% +-------------------------------------------------------------+
\subsection{Example Defining a More Compact Halfedge}

\begin{ccAdvanced}
  
The halfedge data structure as presented here is slightly less space
efficient as, for example, the winged-edge data
structure~\cite{b-prcv-75}, the DCEL~\cite{mp-fitcp-78} or variants of
the quad-edge data structure~\cite{gs-pmgsc-85}.  On the other hand,
it does not require any search operations during traversals. A
comparison can be found in~\cite{k-ugpdd-99}.

The following example trades traversal time for a compact storage
representation using traditional C techniques (i.e., type casting and
the assumption that pointers, especially those from {\tt malloc} or
{\tt new}, point to even addresses). The idea goes as follows: The
halfedge data structure allocates halfedges pairwise.  Concerning the
vector-based data structure this implies that the absolute value of
the difference between a halfedge and its opposite halfedge is always
one with respect to C pointer arithmetic. We can replace the opposite
pointer by a single bit encoding the sign of this difference.  We will
store this bit as the least significant bit in the next halfedge
handle.  Furthermore, we do not implement a pointer to the previous
halfedge. What remains are three pointers per halfedge. 

We use the static member function \ccc{halfedge_handle()} to convert
from pointers to halfedge handles. The same solution can be applied to
the list-based halfedge data structure \ccc{CGAL::HalfedgeDS_list},
see \texttt{examples/HalfedgeDS/hds\_prog\_compact2.cpp}. Here is the
example for the vector-based data structure.

\ccIncludeExampleCode{HalfedgeDS/hds_prog_compact.cpp}

\end{ccAdvanced}

% +-------------------------------------------------------------+
\subsection{Example Using the Halfedge Iterator}

Two edges are created in the default halfedge data structure.
The halfedge iterator is used to count the halfedges.
%\newpage

\ccIncludeExampleCode{HalfedgeDS/hds_prog_halfedge_iterator.cpp}

% +-------------------------------------------------------------+
\subsection{Example for an Adapter to Build an Edge Iterator}

Three edges are created in the default halfedge data structure.
The adapter \ccc{N_step_adaptor} is used to declare the edge
iterator used in counting the edges.

\ccIncludeExampleCode{HalfedgeDS/hds_prog_edge_iterator.cpp}

% +--------------------------------------------------------+

%% %% Copyright (c) 2005  Foundation for Research and Technology-Hellas (Greece).
%% All rights reserved.
%%
%% This file is part of CGAL (www.cgal.org).
%% You can redistribute it and/or modify it under the terms of the GNU
%% General Public License as published by the Free Software Foundation,
%% either version 3 of the License, or (at your option) any later version.
%%
%% Licensees holding a valid commercial license may use this file in
%% accordance with the commercial license agreement provided with the software.
%%
%% This file is provided AS IS with NO WARRANTY OF ANY KIND, INCLUDING THE
%% WARRANTY OF DESIGN, MERCHANTABILITY AND FITNESS FOR A PARTICULAR PURPOSE.
%%
%% $URL$
%% $Id$
%% 
%%
%% Author(s)     : Menelaos Karavelas <mkaravel@iacm.forth.gr>

\ccUserChapter{2D Voronoi Diagram Adaptor\label{chapter-vda}}
\ccChapterAuthor{Menelaos Karavelas}

\minitoc

\begin{ccPkgDescription}{3D Skin Surface Meshing \label{Pkg:SkinSurface3}}
\ccPkgHowToCiteCgal{cgal:k-ssm3-12}
  \ccPkgSummary{ %
    This package allows to build a triangular mesh of a skin surface.
    Skin surfaces are used for modeling large molecules in biological
    computing. The surface is defined by a set of balls, representing
    the atoms of the molecule, and a shrink factor that determines the
    size of the smooth patches gluing the balls together.

    The construction of a triangular mesh of a smooth skin surface is
    often necessary for further analysis and for fast visualization.
    This package provides functions to construct a triangular mesh
    approximating the skin surface from a set of balls and a shrink
    factor. It also contains code to subdivide the mesh efficiently.
    % 
  }
%
  \ccPkgIntroducedInCGAL{3.3}
  \ccPkgDependsOn{\ccRef[3D Triangulation]{Pkg:Triangulation3} and \ccRef[3D Polyhedral Surface]{Pkg:Polyhedron}}
  \ccPkgLicense{\ccLicenseGPL}
  \ccPkgIllustration{Skin_surface_3/small.png}{Skin_surface_3/large.png}
\end{ccPkgDescription}

%% Copyright (c) 2005  Foundation for Research and Technology-Hellas (Greece).
%% All rights reserved.
%%
%% This file is part of CGAL (www.cgal.org).
%% You can redistribute it and/or modify it under the terms of the GNU
%% General Public License as published by the Free Software Foundation,
%% either version 3 of the License, or (at your option) any later version.
%%
%% Licensees holding a valid commercial license may use this file in
%% accordance with the commercial license agreement provided with the software.
%%
%% This file is provided AS IS with NO WARRANTY OF ANY KIND, INCLUDING THE
%% WARRANTY OF DESIGN, MERCHANTABILITY AND FITNESS FOR A PARTICULAR PURPOSE.
%%
%% $URL$
%% $Id$
%% 
%%
%% Author(s)     : Menelaos Karavelas <mkaravel@iacm.forth.gr>

This chapter describes an adaptor that adapts two-dimensional
triangulated Delaunay graphs to the corresponding Voronoi diagrams.
We start with a few
definitions and a description of the issues that this adaptor
addresses in Section~\ref{sec:vda2-intro}. The software design
of the Voronoi diagram adaptor package is described in
Section~\ref{sec:vda2-design}. In Section~\ref{sec:vda2-traits} we
discuss the traits required for performing the adaptation, and finally
in Section~\ref{sec:vda2-examples} we present a few examples using
this adaptor.

\section{Introduction\label{sec:vda2-intro}}

A Voronoi diagram is typically defined for a set of objects, also
called sites in the sequel, that lie in some space $\Sigma$ and a
distance function that measures the distance of a point $x$ in
$\Sigma$ from an object in the object set. In this package we are
interested in planar Voronoi diagrams, so in the sequel the space
$\Sigma$ will be the space $\mathbb{R}^2$. 
Let $\mathcal{S}=\{S_1,S_2,\ldots,S_n\}$ be our set of sites and let
$\delta(x,S_i)$ denote the distance of a point $x\in\mathbb{R}^2$ from
the site $S_i$. Given two sites $S_i$ and $S_j$, the set $V_{ij}$
of points that are closer to $S_i$ than to $S_j$ with respect to the
distance function $\delta(x,\cdot)$ is simply the set:
\[   V_{ij} = \{x\in\mathbb{R}^2:\, \delta(x,S_i)<\delta(x,S_j)\}. \]
We can then define the set $V_i$ of points on the plane that are closer to
$S_i$ than to any other object in $\mathcal{S}$ as:
\[  V_i = \bigcap_{i\neq j} V_{ij}. \]
The set $V_i$ is said to be the \emph{Voronoi cell} or \emph{Voronoi face} 
of the site $S_i$. The locus of points on the plane that are
equidistant from exactly two sites $S_i$ and $S_j$ is called a
\emph{Voronoi bisector}. A point that is equidistant to three or
more objects in $\mathcal{S}$ is called a \emph{Voronoi vertex}.
A simply connected subset of a Voronoi bisector is called a
\emph{Voronoi edge}.
The collection of Voronoi faces, edges and vertices is called the
\emph{Voronoi diagram} of the set $\mathcal{S}$ with respect to the
distance function $\delta(x,\cdot)$, and it turns out that it is a
subdivision of the plane, i.e., it is a planar graph.

We typically think of faces as 2-dimensional objects, edges as
1-dimensional objects and vertices as 0-dimensional objects. However,
this may not be the case for several combinations of sites and
distance functions (for example points in $\mathbb{R}^2$ under the
$L_1$ or the $L_\infty$ distance can produce 2-dimensional Voronoi
edges). We call a Voronoi diagram \emph{nice} if no such artifacts
exist, i.e., if all vertices edges and faces are 0-, 1- and
2-dimensional, respectively.

Even nice Voronoi diagrams can end up being not so nice. The cell of a
site can in general consist of several disconnected components. Such a
case can happen, for example, when we consider weighted points
$Q_i=(p_i,\lambda_i)$, where $p_i\in\mathbb{R}^2$,
$\lambda_i\in\mathbb{R}$, and the distance function is 
the Euclidean distance multiplied by the weight of each site, i.e.,
$\delta_M(x,Q_i)=\lambda_i\,\|x-p_i\|$, where $\|\cdot\|$ denotes the
Euclidean norm. In this package we are going to restrict ourselves to
nice Voronoi diagrams that have the property that the Voronoi cell of
each site is a simply connected region of the plane. We are going to
call such Voronoi diagrams \emph{simple Voronoi diagrams}. Examples of
simple Voronoi diagrams include the usual Euclidean Voronoi diagram of
points, the Euclidean Voronoi diagram of a set of disks on the plane
(i.e., the Apollonius diagram), the Euclidean Voronoi diagram of a set
of disjoint convex objects on the plane, or the power or (Laguerre)
diagram for a set of circles on the plane. In fact every instance of
an \emph{abstract Voronoi diagram} in the sense of Klein \cite{k-cavd-89} 
is a simple Voronoi diagram in our setting. In the sequel when we
refer to Voronoi diagrams we will refer to simple Voronoi diagrams.

In many cases we are not really interested in computing the
Voronoi diagram itself, but rather its dual graph, called the
\emph{Delaunay graph}. In general the Delaunay graph is a planar
graph, each face of which consists of at least three edges.
Under the non-degeneracy assumption that no point on the plane is
equidistant, under the distance function, to more than three sites, 
the Delaunay graph is a planar graph with triangular faces.
In certain cases this graph can actually be embedded with straight
line segments in which case we talk about a triangulation. This is the
case, for example, for the Euclidean Voronoi diagram of points, or the
power diagram of a set of circles. The dual graphs are, respectively,
the Delaunay triangulation and the regular triangulation of the
corresponding site sets. Graphs of non-constant non-uniform face
complexity can be undesirable in many applications, so typically we
end up triangulating the non-triangular faces of the Delaunay
graph. Intuitively this amounts to imposing an implicit or explicit
perturbation scheme during the construction of the Delaunay graph,
that perturbs the input sites in such a way so as not to have
degenerate configurations.

Choosing between computing the Voronoi diagram or the (triangulated)
Delaunay graph is a major decision while implementing an algorithm. It
heavily affects the design and choice of the different data structures
involved. Although in theory the two approaches are entirely
equivalent, it is not so straightforward to go from one representation
to the other. The objective of this package is to provide a generic
way of going from triangulated Delaunay graphs to planar
subdivisions represented through a DCEL data structure. The goal is to
provide an adaptor that gives the look and feel of a DCEL data structure,
although internally it keeps a graph data structure representing
triangular graphs.

The adaptation might seem straightforward at a first glance, and more
or less this is case; after all one graph is the dual of the
other. The situation becomes complicated whenever we want to treat
artifacts of the representation used. Suppose for example that we have
a set of sites that contains subsets of sites in degenerate
positions. The computed triangulated Delaunay graph has triangular
faces that may be the result of an implicit or explicit perturbation
scheme. The dual of such a triangulated Delaunay graph is a Voronoi
diagram that has all its vertices of degree 3, and for that purpose we
are going to call it a \emph{degree-3 Voronoi diagram} in order to
distinguish it from the true Voronoi diagram of the input sites. A
degree-3 Voronoi diagram can have degenerate features, namely Voronoi
edges of zero length, and/or Voronoi faces of zero area. Although we
can potentially treat such artifacts, they are nonetheless artifacts of
the algorithm we used and do not correspond to the true geometry of
the Voronoi diagram.

The manner that we treat such issues in this package in a generic way
is by defining an \emph{adaptation policy}. The adaptation policy is
responsible for determining which features in the degree-3 Voronoi
diagram are to be rejected and which not. The policy to be used can
vary depending on the application or the intended usage of the
resulting Voronoi diagram. What we care about is that firstly the
policy itself is consistent and, secondly, that the adaptation is also
done in a consistent manner. The latter is the responsibility of the
adaptor provided by this package, whereas the former is the
responsibility of the implementor of a policy.

In this package we currently provide two types of adaptation
policies. The first one is the simplest: we reject no feature of the
degree-3 Voronoi diagram; we call such a policy an
\emph{identity policy} since the Voronoi diagram produced is identical
to the degree-3 Voronoi diagram. The second type of policy eliminates
the degenerate features from the degree-3 Voronoi diagram yielding
the true geometry of the Voronoi diagram of the input sites; we call
such policies \emph{degeneracy removal policies}.

Delaunay graphs can be mutable or non-mutable. By mutable we mean that
sites can be inserted or removed at any time, in an entirely on-line
fashion. By non-mutable we mean that once the Delaunay graph has been
created, no changes, with respect to the set of sites defining it,
are allowed. If the Delaunay graph is a non-mutable one, then the
Voronoi diagram adaptor is a non-mutable adaptor as well.

If the Delaunay graph is mutable then the question of whether the
Voronoi diagram adaptor is also mutable is slightly more complex to
answer. As long as the adaptation policy used does not maintain a
state, the Voronoi diagram adaptor is a mutable one; this is the case,
for example, with our identity policy or the degeneracy removal
policies. If, however, the adaptor maintains a state, then whether it
is mutable or non-mutable really depends on whether its state can be
updated after every change in the Delaunay graph. Such policies are
our caching degeneracy removal policies: some of them result in
mutable adaptors others result in non-mutable ones. In
Section~\ref{sec:vda2-ap} we discuss the issue in more detail.




\section{Software Design\label{sec:vda2-design}}

The \ccc{Voronoi_diagram_2<DG,AT,AP>} class is parameterized by
three template parameters. The first one must be a model of the
\ccc{DelaunayGraph_2} concept. It corresponds to the API required by
an object representing a Delaunay graph. All classes of \cgal{} that
represent Delaunay diagrams are models of this concept, namely,
Delaunay triangulations, regular triangulations, Apollonius 
graphs and segment Delaunay graphs.
%
The second template parameter must be a model of the
\ccc{AdaptationTraits_2} concept. We discuss this concept in detail in
Section~\ref{sec:vda2-traits}.
%
The third template parameter must be model of the
\ccc{AdaptationPolicy_2} concept, which we discuss in detail in
Section~\ref{sec:vda2-ap}.

The \ccc{Voronoi_diagram_2<DG,AT,AP>} class has been
intentionally designed to provide an API similar to the arrangements
class in \cgal: Voronoi diagrams are special cases of arrangements
after all. The API of the two classes, however, could not be
identical. The reason is that arrangements in \cgal{} do not yet support
more than one unbounded faces, or equivalently, cannot handle
unbounded curves. On the contrary, a Voronoi diagram defined over at
least two generating sites, has at least two unbounded faces.

On a more technical level, the \ccc{Voronoi_diagram_2<DG,AT,AP>}
class imitates the representation of the Voronoi diagram (seen as a
planar subdivision) by a DCEL (Doubly Connected Edge List) data
structure. We have vertices (the Voronoi vertices), halfedges
(oriented versions of the Voronoi edges) and faces (the Voronoi
cells). In particular, we can basically perform every operation we can
perform in a standard DCEL data structure:
\begin{itemize}
\item go from a halfedge to its next and previous in the face;
\item go from one face to an adjacent one through a halfedge and its
  twin (opposite) halfedge;
\item walk around the boundary of a face;
\item enumerate/traverse the halfedges incident to a vertex
\item from a halfedge, access the adjacent face;
\item from a face, access an adjacent halfedges;
\item from a halfedges, access its source and target vertices;
\item from a vertex, access an incident halfedge.
\end{itemize}
In addition to the above possibilities for traversal, we can also
traverse the following features through iterators:
\begin{itemize}
\item the vertices of the Voronoi diagram;
\item the edges or halfedges of the Voronoi diagram;
\item the faces of the Voronoi diagram;
\item the bounded faces of the Voronoi diagram;
\item the bounded halfedges of the Voronoi diagram;
\item the unbounded faces of the Voronoi diagram;
\item the unbounded halfedges of the Voronoi diagram;
\item the sites defining the Voronoi diagram.
\end{itemize}

%Finally, depending on the adaptation traits passed to the Voronoi diagram
%adaptor, we can perform certain types of queries, namely:
%\begin{itemize}
%\item given a point $p$ we can determine the feature of the Voronoi
%  diagram (vertex, edge, face) on which $p$ lies, and
%\item given a point $p$ determine all nearest sites of $p$ in the
%  Voronoi diagram, i.e., determine all sites $S_i$ in the Voronoi
%  diagram such that the distance $\delta(S_i,p)$ is minimal.
%\end{itemize}
%% SINCE I DON'T WANT TO SUPPORT THE 2ND QUERY YET, THE TEXT IS FOR
%% NOW THE FOLLOWING:
Finally, depending on the adaptation traits passed to the Voronoi diagram
adaptor, we can perform point location queries, namely given a point
$p$ we can determine the feature of the Voronoi diagram (vertex, edge,
face) on which $p$ lies.

%On a more practical basis, arrangements in \cgal\ require as input a
%set of curves. In the case of Voronoi diagrams these curves lie on
%bisectors of generating sites, and can be bounded arcs, semi-bounded
%arcs or the bisectors themselves. Constructing the bisectors can be
%a complicated task, yielding a complicated outcome, which becomes even
%more complicated when we want to compute their endpoints, which are
%the vertices in the Voronoi diagram. Consider, for example, the
%Voronoi diagram of segments: if the segments' coordinates are represented
%by ring number type, computing the Voronoi bisectors and Voronoi
%vertices requires a field number type that also supports square roots
%on top of the field operations.

%On the other hand, we can fully determine the combinatorial structure
%of the Voronoi diagram using the dual Delaunay graph, modulo of course
%the degenerate features we discussed above. In other words, in order
%to compute the combinatorial representation of the Voronoi diagram we
%do not really need to know the bisecting curves or arcs, or the
%Voronoi vertices. Furthermore, not only can we determine the
%combinatorial structure without computing the Voronoi features, but
%can also determine whether a feature is degenerate or not without
%computing its geometry. More specifically, determining whether the dual
%Voronoi edge of an edge in the Delaunay graph is degenerate (of zero
%length) or not can be done without really computing the curve itself.
%For example, in the case of the Delaunay triangulation, determining
%whether the dual Delaunay edge has zero length amounts to performing
%an incircle test.


\section{The Adaptation Traits\label{sec:vda2-traits}}

The \ccc{AdaptationTraits_2} concept defines the types and
functor required by the adaptor in order to access geometric
information in the Delaunay graph that is needed by the
\ccc{Voronoi_diagram_2<DG,AT,AP>} class.
In particular, it provides functors for accessing sites in the Delaunay
graph and constructing Voronoi vertices from their dual faces in the
Delaunay graph.
Finally, it defines a tag that indicates whether nearest site queries
are to be supported by the Voronoi diagram adaptor. If such queries
are to be supported, a functor is required.

Given a query point, the nearest site functor should return information 
related to how many and which sites of the Voronoi diagram are at
equal and minimal distance from the query point. In particular, if the
query point is closest to a single site, the vertex handle of the
Delaunay graph corresponding to this site is returned. If the
query point is closest to exactly two site, the edge of the
Delaunay graph that is dual to the Voronoi edges on which the query
point lies is returned. If three (or more) sites are closest to
the query point, then the query point coincides with a vertex in the
Voronoi diagram, and the face handle of the face in the Delaunay graph
that is dual to the Voronoi vertex is returned.
This way of abstracting the point location mechanism allows
for multiple different point location strategies, which are passed to
the Voronoi diagram adaptor through different models of the
\ccc{AdaptationTraits_2} concept. The point location and nearest sites
queries of the \ccc{Voronoi_diagram_2<DG,AT,AP>} class use internally
this nearest site query functor.

In this package we provide four adaptation traits classes, all of which
support nearest site queries:
\begin{itemize}
\item
  The \ccc{Apollonius_graph_adaptation_traits_2<AG2>} class: it
  provides the adaptation traits for Apollonius graphs.
\item
  The \ccc{Delaunay_triangulation_adaptation_traits_2<DT2>} class: it
  provides the adaptation traits for Delaunay triangulations.
\item
  The \ccc{Regular_triangulation_adaptation_traits_2<RT2>} class: it
  provides the adaptation traits for regular triangulations.
\item
  The \ccc{Segment_Delaunay_graph_adaptation_traits_2<SDG2>} class: it
  provides the adaptation traits for segment Delaunay graphs.
\end{itemize}

\section{The Adaptation Policy\label{sec:vda2-ap}}

As mentioned above, when we perform the adaptation of a triangulated
Delaunay graph to a Voronoi diagram, a question that arises is whether
we want to eliminate certain features of the Delaunay graph when we
construct its Voronoi diagram representation (such features could be
the Voronoi edges of zero length or, for the Voronoi diagram of a set
of segments forming a polygon, all edges outside the polygon).
The manner that we treat such issues in this package in a generic way
is by defining an adaptation policy. The adaptation policy is
responsible for determining which features in the degree-3 Voronoi
diagram are to be rejected and which not. The policy to be used can
vary depending on the application or the intended usage of the
resulting Voronoi diagram.

The concept \ccc{AdaptationPolicy_2} defines the requirements on
the predicate functors that determine whether a feature of the
triangulated Delaunay graph should be rejected or not. More
specifically it defines an \ccc{Edge_rejector} and a
\ccc{Face_rejector} functor that answer the question: ``should this
edge (face) of the Voronoi diagram be rejected?''. In addition to the
edge and face rejectors the adaptation policy defines a tag, the
\ccc{Has_inserter} tag. This tag is either set to \ccc{CGAL::Tag_true}
or to \ccc{CGAL::Tag_false}. Semantically it determines if the adaptor
is allowed to insert sites in an on-line fashion (on-line removals are
not yet supported). In the former case, i.e., when on-line site
insertions are allowed, an additional functor is required, the
\ccc{Site_inserter} functor. This functor takes a reference to a
Delaunay graph and a site, and inserts the site in the Delaunay
graph. Upon successful insertion, a handle to the vertex representing
the site in the Delaunay graph is returned.

We have implemented two types of policies that provide two different
ways for answering the question of which features of the Voronoi
diagram to keep and which to discard. The first one is called the
\emph{identity policy} and corresponds to the
\ccc{Identity_policy_2<DG,VT>} class. This policy is in some sense the
simplest possible one, since it does not reject any feature of the
Delaunay graph. The Voronoi diagram provided by the adaptor is the
true dual (from the graph-theoretical point of view) of the
triangulated Delaunay graph adapted. This policy assumes that the
Delaunay graph adapted allows for on-line insertions, and the
\ccc{Has_inserter} tag is set to \ccc{CGAL::Tag_true}. A default site
inserter functor is also provided.

The second type of policy we provide is called
\emph{degeneracy removal policy}. If the set of sites defining the
triangulated Delaunay graph contains subsets of sites in degenerate
configurations, the graph-theoretical dual of the triangulated
Delaunay graph has edges and potentially faces that are geometrically
degenerate. By that we mean that the dual of the triangulated Delaunay
graph can have Voronoi edges of zero length or Voronoi faces/cells of
zero area. Such features may not be desirable and ideally we would
like to eliminate them. The degeneracy removal policies eliminate
exactly these features and provide a Voronoi diagram where all edges
have non-zero length and all cells have non-zero area. More
specifically, in these policies the \ccc{Edge_rejector} and
\ccc{Face_rejector} functors reject the edges and vertices of the
Delaunay graph that correspond to dual edges and faces that have zero
length and area, respectively. In this package we provide four
degeneracy removal policies, namely:
\begin{itemize}
\item
  The \ccc{Apollonius_graph_degeneracy_removal_policy_2<AG2>} class: it
  provides an adaptation policy for removing degeneracies when
  adapting an Apollonius graph to an Apollonius diagram.
\item
  The \ccc{Delaunay_triangulation_degeneracy_removal_policy_2<DT2>} class: it
  provides an adaptation policy for removing degeneracies when
  adapting a Delaunay triangulation to a point Voronoi diagram.
\item
  The \ccc{Regular_triangulation_degeneracy_removal_policy_2<RT2>} class: it
  provides an adaptation policy for removing degeneracies when
  adapting a regular triangulation to a power diagram 

\item
  The \ccc{Segment_Delaunay_graph_degeneracy_removal_policy_2<SDG2>} class: it
  provides an adaptation policy for removing degeneracies when
  adapting a segment Delaunay graph to a segment Voronoi diagram.
\end{itemize}

A variation of the degeneracy removal policies are the
\emph{caching degeneracy removal policies}. In these policies we cache
the results of the edge and face rejectors. In particular, every time
we want to determine, for example, if an edge of the Delaunay graph
has, as dual edge in the Voronoi diagram, an edge of zero length, we
check if the result has already been computed. If yes, we simply
return the outcome. If not, we perform the necessary geometric tests,
compute the answer, cache it and return it. Such a policy really pays
off when we have a lot of degenerate data in our input set of
sites. Verifying whether a Voronoi edge is degenerate or not implies
computing the outcome of a predicate in a possibly degenerate or near
degenerate configuration, which is typically very costly (compared to
computing the same predicate in a generic configuration). To avoid this cost
every single time we want to check if a Voronoi edge is degenerate or
not, we compute the result of the geometric predicate the first time
the adaptor asks for it, and simply lookup the answer in the future.
In this package we provide four caching degeneracy removal policies,
one per degeneracy removal policy mentioned above.
Intentionally, we have not indicated the value of the
\ccc{Has_inserter} tag for the degeneracy removal and caching
degeneracy removal policies. The issue is discussed in detail in the
sequel.

We raised the question above, as to whether the adaptor is a mutable
or non-mutable one, in the sense of whether we can add/remove sites in
an on-line fashion. The answer to this question depends on: (1) whether the
Delaunay graph adapted allows for on-line insertions/removals and (2)
whether the adaptation policies maintains a state and whether this
state is easily maintainable when we want to allow for on-line
modifications.

The way we indicate if we allow on-line insertions of sites is via the
\ccc{Has_inserter} tag (as mentioned, on-line removals are currently not
supported). The \ccc{Has_inserter} tag has two possible values,
namely, \ccc{CGAL::Tag_true} and \ccc{CGAL::Tag_false}. The value
\ccc{CGAL::Tag_true} indicates that the Delaunay graph allows for
on-line insertions, whereas the value \ccc{CGAL::Tag_false} indicates
the opposite. Note that these values \emph{do not} indicate if the
Delaunay graph supports on-line insertions, but rather whether the
Voronoi diagram adaptor should be able to perform on-line insertions
or not. This delicate point will be become clearer below.

Let us consider the various scenarios. If the Delaunay graph is
non-mutable, the Voronoi diagram adaptor cannot perform on-line
insertions of sites. In this case not only degeneracy removal
policies, but rather every single adaptation policy for
adapting the Delaunay graph in question should have the
\ccc{Has_inserter} tag set to \ccc{CGAL::Tag_false}.

If the Delaunay graph is mutable, i.e., on-line site insertions as are
allowed, we can choose between two types of adaptation policies, those
that allow these on-line insertions and those that do not. In the
former case the \ccc{Has_inserter} tag should be set to
\ccc{CGAL::Tag_true}, whereas in the latter to
\ccc{CGAL::Tag_false}. In other words, even if the Delaunay graph is
mutable, we can choose (by properly determining the value of the
\ccc{Has_inserter} tag) if the adaptor should be mutable as well. At a
first glance it may seem excessive to restrict existing
functionality. There are situations, however, where such a choice is
necessary.

Consider a caching degeneracy removal policy. If we do not allow for
on-line insertions then the cached quantities are always valid since
the Voronoi diagram never changes. If we allow for on-line insertions
the Voronoi diagram can change, which implies that the results of the edge
and faces degeneracy testers that we have cached are no longer valid
or relevant. In these cases, we need to somehow update these cached
results, and ideally we would like to do this in an efficient manner.
%
The inherent dilemma in the above discussion is whether the Voronoi
diagram adaptor should be able to perform on-line insertions of
sites. The answer to this question in this framework is given by the
\ccc{Has_inserter} tag. If the tag is set to \ccc{CGAL::Tag_false} the
adaptor cannot insert sites on-line, whereas if the tag is set to
\ccc{CGAL::Tag_true} the adaptor can add sites on-line. In other
words, the \ccc{Has_inserter} tag determines how the Voronoi diagram
adaptor should behave, and this is enough from the adaptor's point of
view.

From the point of a view of a policy writer the dilemma is still
there: should the policy allow for on-line insertions or not? The
answer really depends on what are the consequences of such a
choice. For a policy that has no state, such as our degeneracy removal
policies, it is natural to set the \ccc{Has_inserter} tag to
\ccc{CGAL::Tag_true}. For our caching degeneracy removal policies, our
choice was made on the grounds of whether we can update the cached
results efficiently when insertions are performed. For \cgal's
Apollonius graphs, Delaunay triangulation and regular triangulations
it is possible to ask what are the edges and faces of the Delaunay
graph that are to be destroyed when a query site is inserted. This is
done via the \ccc{get_conflicts} method provided by these
classes. Using the outcome of the \ccc{get_conflicts} method the site
inserter can first update the cached results (i.e., indicate which are
invalidated) and then perform the actual insertion. Such a method does
not yet exist for segment Delaunay graphs. We have thus chosen to
support on-line insertions for all non-caching degeneracy removal
policies. The caching degeneracy removal policy for segment Delaunay
graphs does not support on-line insertions, whereas the remaining
three caching degeneracy removal policies support on-line insertions.




\subsection*{Efficiency Considerations\label{subsec:vda2-efficiency}}

One last item that merits some discussion are the different choices
from the point of view of time- and space-efficiency.

As far as the Voronoi diagram adaptor is concerned, only a copy of the
adaptation traits and a copy of the adaptation policy are stored in it.
The various adaptation traits classes we provide are empty
classes (i.e., they do not store anything). The major time and space
efficiency issues arise from the various implementations of the
adaptation policies.
%
Clearly, the identity policy has no dominant effect on neither the
time or space efficiency. The costs when choosing this policy are due
to the underlying Delaunay graph.

The non-caching degeneracy removal policies create a significant time
overhead since every time we want to access a feature of the Voronoi
diagram, we need to perform geometric tests in order to see if this
feature or one of its neighboring ones has been rejected.
Such a policy is acceptable if we know we are away from
degeneracies or for small input sizes. In the case of the segment
Delaunay graph, it is also the only policy we provide that at the same
time removes degeneracies and allows for on-line insertion of sites.
Caching policies seem to be the best choice for moderate to large
input sizes (1000 sites and more). They do not suffer from the problem
of dealing with degenerate configurations, but since they cache the
results, they increase the space requirements by linear additive
factor.
%
To conclude, if the user is interested in getting a Voronoi diagram
without degenerate features and knows all sites in advance, the best course
of action is to insert all sites at construction time and use a caching
degeneracy removal policy. This strategy avoids the updates of the
cached results after each individual insertion, due to the features of
the Voronoi diagram destroyed because of the site inserted.

%% Copyright (c) 2005  Foundation for Research and Technology-Hellas (Greece).
%% All rights reserved.
%%
%% This file is part of CGAL (www.cgal.org); you may redistribute it under
%% the terms of the Q Public License version 1.0.
%% See the file LICENSE.QPL distributed with CGAL.
%%
%% Licensees holding a valid commercial license may use this file in
%% accordance with the commercial license agreement provided with the software.
%%
%% This file is provided AS IS with NO WARRANTY OF ANY KIND, INCLUDING THE
%% WARRANTY OF DESIGN, MERCHANTABILITY AND FITNESS FOR A PARTICULAR PURPOSE.
%%
%% $URL$
%% $Id$
%% 
%%
%% Author(s)     : Menelaos Karavelas <mkaravel@iacm.forth.gr>

\section{Examples\label{sec:vda2-examples}}

In this section we present an example that shows how to perform point
location queries.

%\subsection{Second example}
%
%The following example shows how we can perform point location using
%the Voronoi diagram adaptor.

\ccIncludeExampleCode{Voronoi_diagram_2/vd_2_point_location.cpp}



% EOF
