% +------------------------------------------------------------------------+
% | Reference manual page: RangeSearchTraits.tex
% +------------------------------------------------------------------------+
% | 22.12.2010 Sebastien Loriot
% | Package: ASPAS
% | 
% \RCSdef{\RCSSpatialPointRev}{$Id: RangeSearchTraits.tex -1   $}
% \RCSdefDate{\RCSSpatialPointDate}{$Date: $}
% |
%%RefPage: end of header, begin of main body
% +------------------------------------------------------------------------+


\begin{ccRefConcept}{RangeSearchTraits}

%% \ccHtmlCrossLink{}     %% add further rules for cross referencing links
%% \ccHtmlIndexC[concept]{} %% add further index entries

\ccDefinition
  
The concept \ccClassTemplateName\ defines the requirements for the template 
parameter of the search classes. This concept also defines requirements to
range search queries in a model of \ccc{SpatialTree}.

\ccRefines

\ccc{SearchTraits}

\ccTypes

\ccNestedType{Iso_box_d}{Iso box type. It is only needed for range search queries.}
\ccNestedType{Sphere_d}{Sphere type. It is only needed for range search queries.}


\ccNestedType{Construct_iso_box_d}{Functor with operator to construct the iso box from two points.}

\ccNestedType {Construct_center_d}{Functor with operator to construct 
the center of an object of type \ccc{Sphere_d}.}

\ccNestedType {Construct_squared_radius_d}{Functor with operator to compute 
the squared radius of a an object of type \ccc{Sphere_d}.}

\ccNestedType {Construct_min_vertex_d}{Functor with operator to construct 
the vertex with lexicographically smallest coordinates of an object of type \ccc{Iso_box_d}.}

\ccNestedType {Construct_max_vertex_d}{Functor with operator to construct 
the vertex with lexicographically largest coordinates of an object of type \ccc{Iso_box_d}.}


\ccHasModels
\ccc{CGAL::Cartesian_d<FT>}\\
\ccc{CGAL::Homogeneous_d<RT>}\\
\ccc{CGAL::Search_traits_2<Kernel>}\\
\ccc{CGAL::Search_traits_3<Kernel>}


\ccSeeAlso
\ccc{SearchTraits}
\ccc{CGAL::Search_traits_adapter<Key,PointPropertyMap,BaseTraits>}

\end{ccRefConcept}

% +------------------------------------------------------------------------+
%%RefPage: end of main body, begin of footer
% EOF
% +------------------------------------------------------------------------+

