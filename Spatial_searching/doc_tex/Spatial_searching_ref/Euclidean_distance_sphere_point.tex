% +------------------------------------------------------------------------+
% | Reference manual page: Euclidean_distance_sphere_point.tex
% +------------------------------------------------------------------------+
% | 1.07.2001   Johan W.H. Tangelder
% | Package: ASPAS
% | 
\RCSdef{\RCSEuclideandistancespherepointRev}{$Id$}
\RCSdefDate{\RCSEuclideandistancespherepointDate}{$Date$}
% |
%%RefPage: end of header, begin of main body
% +------------------------------------------------------------------------+


\begin{ccRefClass}{Euclidean_distance_sphere_point<Traits>}  %% add template arg's if necessary

%% \ccHtmlCrossLink{}     %% add further rules for cross referencing links
%% \ccHtmlIndexC[class]{} %% add further index entries

\ccDefinition
  
The class \ccRefName\ provides an implementation of the
\ccc{GeneralDistance} concept for the Euclidean distance ($l_2$
metric) between a $d$-dimensional sphere and a point, and the
Euclidean distance between a $d$-dimensional sphere and a
$d$-dimensional iso-rectangle defined as a $k$-$d$ tree rectangle.


\ccInclude{CGAL/Euclidean_distance_sphere_point.h}

\ccParameters
Expects for the template argument a model of the concept \ccc{SearchTraits}, 
for example \ccc{CGAL::Cartesian_d<double>}.


\ccIsModel

\ccc{GeneralDistance}

\ccTypes

\ccTypedef{Traits::FT FT;}{Number type.}
\ccTypedef{Traits::Point_d Point_d;}{Point type.}
\ccTypedef{Traits::Sphere_d Sphere_d;}{Query item type.}


\ccCreation
\ccCreationVariable{ed}  %% choose variable name


\ccConstructor{Euclidean_distance_sphere_point(Traits t=Traits());}
{Default constructor.}

\ccOperations

\ccMethod{NT transformed_distance(Query_item s, Point_d p);}{Returns the distance between \ccc{s} and \ccc{p}.}

\ccMethod{NT min_distance_to_rectangle(Query_item s, Kd_tree_rectangle<FT> r);}
{Returns the minimal distance between a point from the sphere \ccc{s} and a point from
\ccc{r}.}

\ccMethod{NT max_distance_to_rectangle(Query_item s, Kd_tree_rectangle<FT> r);}
{Returns the maximal distance between the sphere \ccc{s} and
a point from \ccc{r} furthest to \ccc{s}. }

\ccMethod{NT transformed_distance(NT d);} {Returns $d^2$.}

\ccMethod{NT inverse_of_transformed_distance(NT d);} {Returns $d^{1/2}$.}


\ccSeeAlso

\ccc{GeneralDistance}

%% \ccHtmlCrossLink{}     %% add further rules for cross referencing links
%% \ccHtmlIndexC[class]{} %% add further index entries


\end{ccRefClass}

% +------------------------------------------------------------------------+
%%RefPage: end of main body, begin of footer
% EOF
% +------------------------------------------------------------------------+

