% +------------------------------------------------------------------------+
% | Reference manual page: Fair.tex
% +------------------------------------------------------------------------+
% | 1.07.2001   Johan W.H. Tangelder
% | Package: ASPAS
% | 
\RCSdef{\RCSFairRev}{$Id$}
\RCSdefDate{\RCSFairDate}{$Date$}
% |
%%RefPage: end of header, begin of main body
% +------------------------------------------------------------------------+


\begin{ccRefFunctionObjectClass}{Fair<Traits, SpatialSeparator>}  %% add template arg's if necessary

%% \ccHtmlCrossLink{}     %% add further rules for cross referencing links
%% \ccHtmlIndexC[class]{} %% add further index entries

\ccDefinition
Implements the {\em fair} splitting rule.  
This splitting rule is a compromise between the median of rectangle
splitting rule and the midpoint of rectangle splitting rule. This
splitting rule maintains an upper bound on the maximal allowed ratio
of the longest and shortest side of a rectangle (the value of this
upper bound is set in the constructor of the fair splitting
rule). Among the splits that satisfy this bound, it selects the one in
which the points have the largest spread.  It then splits the points
in the most even manner possible, subject to maintaining the bound on
the ratio of the resulting rectangles.

\ccInclude{CGAL/Splitters.h}

\ccParameters

Expects for the first template argument a model of
the concept \ccc{SearchTraits}, 
for example \ccc{Cartesian_d<double>}. 

Expects for the second template argument a model of the concept \ccc{Separator}. 
It has as default value the type, \ccc{CGAL::Plane_separator<Traits::FT>}.



\ccIsModel

Splitter

\ccTypes

\ccTypedef{Traits::FT FT;}{Number type.} 

\ccCreation
\ccCreationVariable{s}  %% choose variable name

\ccConstructor{Fair();}{Default constructor.}
\ccConstructor{ Fair(unsigned int bucket_size, 
                     FT aspect_ratio=FT(3))}{Constructor.}

\ccSeeAlso

\ccc{Splitter},\\
\ccc{SpatialSeparator}

\end{ccRefFunctionObjectClass}

% +------------------------------------------------------------------------+
%%RefPage: end of main body, begin of footer
% EOF
% +------------------------------------------------------------------------+

