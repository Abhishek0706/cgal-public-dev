% +------------------------------------------------------------------------+
% | Reference manual page: Splitter.tex
% +------------------------------------------------------------------------+
% | 1.07.2001   Johan W.H. Tangelder
% | Package: ASPAS
% | 
\RCSdef{\RCSSplitterRev}{$Id$}
\RCSdefDate{\RCSSplitterDate}{$Date$}
% |
%%RefPage: end of header, begin of main body
% +------------------------------------------------------------------------+


\begin{ccRefConcept}{Splitter}

%% \ccHtmlCrossLink{}     %% add further rules for cross referencing links
%% \ccHtmlIndexC[concept]{} %% add further index entries

\begin{ccAdvanced}
\ccDefinition
  
The concept \ccRefName\ defines the requirements for a function object class implementing a splitting rule.

\ccHasModels

\ccc{CGAL::Fair<Traits, SpatialSeparator>}, \\
\ccc{CGAL::Median_of_rectangle<Traits, SpatialSeparator>}, \\
\ccc{CGAL::Median_of_max_spread<Traits, SpatialSeparator>}, \\
\ccc{CGAL::Midpoint_of_rectangle<Traits, SpatialSeparator>}, \\
\ccc{CGAL::Midpoint_of_max_spread<Traits, SpatialSeparator>}, \\
\ccc{CGAL::Sliding_fair<Traits, SpatialSeparator>}, \\
\ccc{CGAL::Sliding_midpoint<Traits, SpatialSeparator>.}

\ccTypes

\ccNestedType{FT}{Number type.}
\ccNestedType{Separator}{Separator.} 
\ccNestedType{Container}{Typedef to an instantiation of \ccc{CGAL::Point_container<Traits>}.} 

\ccCreationVariable{s}  %% choose variable name

The parameters \ccc{aspect_ratio} and \ccc{bucket_size}
define the way in which $k$-$d$ tree is constructed.

\ccOperations


\ccMethod{FT aspect_ratio() const;}{Returns the maximal ratio between the largest and smallest side
of a cell allowed for fair splitting.}

\ccMethod{unsigned int bucket_size() const;} {Returns the bucket size of the leaf nodes.}

\ccMethod{void operator()(Separator& sep, Container& c0, Container& c1) const;}
{
  Sets up \ccc{sep} and splits points of \ccc{c0} into \ccc{c0} and \ccc{c1} using \ccc{sep}.
  Container \ccc{c0} should contain at least two points and \ccc{c1} must be empty.
}

\end{ccAdvanced}

\end{ccRefConcept}

% +------------------------------------------------------------------------+
%%RefPage: end of main body, begin of footer
% EOF
% +------------------------------------------------------------------------+

