% +------------------------------------------------------------------------+
% | Reference manual page: Orthogonal_incremental_neighbor_search.tex
% +------------------------------------------------------------------------+
% | 1.07.2001   Johan W.H. Tangelder
% | Package: ASPAS
% |
\RCSdef{\RCSOrthogonalincrementalneighborsearchRev}{$Id$}
\RCSdefDate{\RCSOrthogonalincrementalneighborsearchDate}{$Date$}
% |
%%RefPage: end of header, begin of main body
% +------------------------------------------------------------------------+


\begin{ccRefClass}{Orthogonal_incremental_neighbor_search<Traits, OrthogonalDistance, Splitter, SpatialTree>}

%% \ccHtmlCrossLink{}     %% add further rules for cross referencing links
%% \ccHtmlIndexC[class]{} %% add further index entries

\ccDefinition

The class \ccRefName\ implements incremental nearest and furthest neighbor searching on a tree.

\ccInclude{CGAL/Orthogonal_incremental_neighbor_search.h}

\ccParameters

Expects for the first template argument a model of the concept \ccc{SearchTraits},
for example \ccc{CGAL::Search_traits_2<CGAL::Cartesian<double> >}.

Expects for the second template argument a model of the
concept \ccc{GeneralDistance}. If \ccc{Traits} is 
\ccc{CGAL::Search_traits_adapter<Key,PointPropertyMap,BaseTraits>}
the default type is \ccc{CGAL::Distance_for_point_adapter<Key,PointPropertyMap,CGAL::Euclidean_distance<Traits> >},
and \ccc{CGAL::Euclidean_distance<Traits>} otherwise.

Expects for third template argument a model of the concept \ccc{Splitter}.
The default type is \ccc{CGAL::Sliding_midpoint<Traits>}.

Expects for fourth template argument a model of the concept \ccc{SpatialTree}.
The default type is \ccc{CGAL::Kd_tree<Traits, Splitter, CGAL::Tag_true>}. The 
template argument must be \ccc{CGAL::Tag_true} because orthogonal search needs extended 
kd tree nodes.

\ccTypes

\ccTypedef{Traits::Point_d Point_d;}{Point type.}
\ccTypedef{Traits::FT FT;}{Number type.}
\ccTypedef{OrthogonalDistance Distance;}{Distance type.}
\ccTypedef{OrthogonalDistance::Query_item Query_item;}{Query item.}
\ccTypedef{std::pair<Point_d,FT> Point_with_transformed_distance;}{Pair of point and transformed distance.}
\ccNestedType{iterator}{const input iterator with value type \ccc{Point_with_transformed_distance}
for enumerating approximate neighbors.}

\ccTypedef{SpatialTree Tree;}{The tree type.}
\ccCreation
\ccCreationVariable{s}  %% choose variable name

\def\ccLongParamLayout{\ccTrue}
\ccConstructor{Orthogonal_incremental_neighbor_search(SpatialTree& tree, Query_item query, FT eps=FT(0.0),
                                          bool search_nearest=true, 
                                          OrthogonalDistance d=OrthogonalDistance());}
{Constructor for incremental neighbor searching of the query item \ccc{query}
in the points stored \ccc{tree} using a distance \ccc{d} and approximation factor \ccc{eps}.}

\ccOperations

\ccMethod{iterator begin() const;}{Returns a const iterator to the approximate nearest or furthest neighbor.}

\ccMethod{iterator end() const;}{Returns the appropriate past-the-end const iterator.}

\
\begin{ccAdvanced}
\ccMethod{std::ostream& statistics(std::ostream& s) const;}
{
Inserts statistics of the search process into the output stream~\ccc{s}.
}
\end{ccAdvanced}

\ccSeeAlso

\ccc{CGAL::Incremental_neighbor_search<Traits, GeneralDistance, SpatialTree>}

\end{ccRefClass}

% +------------------------------------------------------------------------+
%%RefPage: end of main body, begin of footer
% EOF
% +------------------------------------------------------------------------+

