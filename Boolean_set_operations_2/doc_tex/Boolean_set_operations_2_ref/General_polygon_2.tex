\ccRefPageBegin

\begin{ccRefClass}{General_polygon_2<ArrTraits>}

\ccThree{Orientation}{ polygon.reverse_orientation ()~}{}
\ccThreeToTwo

\ccDefinition
%============
The class \ccRefName\ models the concept \ccc{GeneralPolygon_2}.
It represents a simple general-polygon. It is parameterized with the type
\ccc{ArrTraits} that models the concept
\ccc{ArrangementDirectionalXMonotoneTraits}. The latter is a refinement
of the concept \ccc{ArrangementXMonotoneTraits_2}. In addition to the
requirements of the concept \ccc{ArrangementXMonotoneTraits_2}, a
model of the concept \ccc{ArrangementDirectionalXMonotoneTraits} must
support the following functions:
\begin{itemize}
\item Given an $x$-monotone curve, construct its opposite curve.
\item Given an $x$-monotone curve, compare its two endpoints lexicographically.
\end{itemize} 

This class supports a few convenient operations in addition to the 
requirements that the concept \ccc{GeneralPolygon_2} lists.

\ccInclude{CGAL/General_polygon_2.h}
 
\ccTypes
%=======

% \ccNestedType{Vertex_iterator}{a bidirectional iterator over the
%        vertices of the polygonal region. Its value-type is \ccc{Vertex}.}

\ccNestedType{Size}{number of edges size type.}

\ccCreation
\ccCreationVariable{polygon}
%===========================

% \ccAccessFunctions
% %=================
% 
% \ccMethod{Vertex_iterator vertices_begin();} {returns the begin
% iterator of the vertices.}
% \ccGlue
% \ccMethod{Vertex_iterator vertices_end();} {returns the past-the-end
% iterator of the vertices.}
% 

\ccOperations
%============
\ccMethod{Size size();} {returns the number of edges of \ccVar.}

\ccModifiers
% ==========
\ccMethod{void clear();}{clears \ccVar.}

\ccMethod{void reverse_orientation();}{reverses the orientation of the polygon.
\ccPrecond{\ccStyle{is_simple()}.}}

\ccPredicates
% ===========
\ccMethod{bool is_empty();}
{returns \ccc{true} if \ccVar\ is empty, and \ccc{false} otherwise.}

\ccMethod{Orientation orientation();}{returns the orientation of \ccVar.
\ccPrecond{\ccStyle{is_simple()}.}}

% \ccMethod{bool is_convex();}{returns \ccc{true} if the polygonal
% region is convex, and \ccc{false} otherwise.}

\ccIsModel
  \ccc{GeneralPolygon_2}

\end{ccRefClass}
\ccRefPageEnd
