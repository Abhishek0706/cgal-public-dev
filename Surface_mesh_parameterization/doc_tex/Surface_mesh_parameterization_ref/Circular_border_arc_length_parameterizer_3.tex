% +------------------------------------------------------------------------+
% | Reference manual page: Circular_border_arc_length_parameterizer_3.tex
% +------------------------------------------------------------------------+
% | 23.08.2005   Author
% | Package: Surface_mesh_parameterization
% |
\RCSdef{\RCSCircularborderarclengthparameterizerRev}{$Id$}
\RCSdefDate{\RCSCircularborderarclengthparameterizerDate}{$Date$}
% |
\ccRefPageBegin
%%RefPage: end of header, begin of main body
% +------------------------------------------------------------------------+


\begin{ccRefClass}{Circular_border_arc_length_parameterizer_3<ParameterizationMesh_3>}

%% \ccHtmlCrossLink{}     %% add further rules for cross referencing links
%% \ccHtmlIndexC[class]{} %% add further index entries


\ccDefinition

\ccc{Circular_border_arc_length_parameterizer_3} is the default border parameterizer
for fixed border parameterization methods.

% The section below is automatically generated. Do not edit!
%START-AUTO(\ccDefinition)

This class parameterizes the border of a 3D surface onto a circle, with an arc-length parameterization: (u, v) values are proportional to the length of border edges. \ccc{Circular_border_parameterizer_3} implements most of the border parameterization algorithm. This class implements only \ccc{compute_edge_length}() to compute a segment's length.

%END-AUTO(\ccDefinition)

% The section below is automatically generated. Do not edit!
%START-AUTO(\ccInclude)

\ccInclude{CGAL/Circular_border_parameterizer_3.h}

%END-AUTO(\ccInclude)


\ccInheritsFrom

% The section below is automatically generated. Do not edit!
%START-AUTO(\ccInheritsFrom)

\ccc{Circular_border_parameterizer_3<ParameterizationMesh_3>}

%END-AUTO(\ccInheritsFrom)


\ccIsModel

% The section below is automatically generated. Do not edit!
%START-AUTO(\ccIsModel)

Model of the \ccc{BorderParameterizer_3} concept.

%END-AUTO(\ccIsModel)


\ccHeading{Design Pattern}

% The section below is automatically generated. Do not edit!
%START-AUTO(\ccHeading{Design Pattern})

\ccc{BorderParameterizer_3} models are Strategies \cite{cgal:ghjv-dpero-95}: they implement a strategy of border parameterization for models of \ccc{ParameterizationMesh_3}

%END-AUTO(\ccHeading{Design Pattern})


\ccParameters

The full template declaration is:

% The section below is automatically generated. Do not edit!
%START-AUTO(\ccParameters)

template$<$class \ccc{ParameterizationMesh_3}$>$   \\
class \ccc{Circular_border_arc_length_parameterizer_3};

%END-AUTO(\ccParameters)


\ccCreation
\ccCreationVariable{bp}  %% variable name for \ccMethod below

% The section below is automatically generated. Do not edit!
%START-AUTO(\ccCreation)
%END-AUTO(\ccCreation)


\ccOperations

% The section below is automatically generated. Do not edit!
%START-AUTO(\ccOperations)

\ccMethod{virtual double compute_edge_length(const Adaptor& mesh, Vertex_const_handle source, Vertex_const_handle target)[protected, virtual];}
{
Compute the length of an edge.  \\
Arc-length border parameterization: (u, v) values are proportional to the length of border edges.
}
\ccGlue

%END-AUTO(\ccOperations)


\ccSeeAlso

\ccRefIdfierPage{CGAL::Circular_border_uniform_parameterizer_3<ParameterizationMesh_3>}  \\
\ccRefIdfierPage{CGAL::Square_border_arc_length_parameterizer_3<ParameterizationMesh_3>}  \\
\ccRefIdfierPage{CGAL::Square_border_uniform_parameterizer_3<ParameterizationMesh_3>}  \\
\ccRefIdfierPage{CGAL::Two_vertices_parameterizer_3<ParameterizationMesh_3>}  \\


\ccExample

See \ccc{Eigen_parameterization.cpp} example.


\end{ccRefClass}

% +------------------------------------------------------------------------+
%%RefPage: end of main body, begin of footer
\ccRefPageEnd
% EOF
% +------------------------------------------------------------------------+

