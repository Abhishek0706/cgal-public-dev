% +------------------------------------------------------------------------+
% | Reference manual page: Arr_mrg_data_traits.tex
% +------------------------------------------------------------------------+
% | 
% | Package: Arrangement_2
% | 
% +------------------------------------------------------------------------+

\ccRefPageBegin
\begin{ccRefClass}{Arr_curve_data_traits_2<Tr,XData,Mrg,CData,Cnv>}

\ccDefinition
%============

The class \ccRefName\ is a model of the \ccc{ArrangementTraits_2} concept
and serves as a decorator class that allows the extension of the curves
defined by the base traits-class (the \ccc{Tr} parameter), which serves as a
geometric traits-class (a model of the \ccc{ArrangementTraits_2} concept), with
extraneous (non-geometric) data fields.

The traits class inherits its point type from \ccc{Traits::Point_2},
and defines an extended \ccc{Curve_2} and \ccc{X_monotone_curve_2} types,
as detailed below.

Each \ccc{Curve_2} object is associated with a single data field of type
\ccc{CData}, and each \ccc{X_monotone_curve_2} object is associated with
a single data field of type \ccc{XData}. When a curve is
subdivided into $x$-monotone subcurves, its data field is converted using
the conversion functor, which is specified by the \ccc{Cnv} template-parameter,
and the resulting objects is copied to all \ccc{X_monotone_curve_2} objects
induced by this curve. The conversion functor should provide an operator with
the following prototype: \\
\indent \ccc{XData operator() (const CData& d) const;} \\

By default, the two data types are the same, so the conversion operator
is trivial:
\begin{tabbing}
\ccc{CData} = \=\ccc{XData}\\
\ccc{Cnv} = \>\ccc{_Default_convert_functor<CData,XData>}
\end{tabbing}

In case two (or more) $x$-monotone curves overlap, their data fields are
merged to a single field, using the merge functor functor, which is
specified by the \ccc{Mrg} template-parameter. This functor should provide
an operator with the following prototype: \\
\indent \ccc{XData operator() (const XData& d1, const XData& d2) const;} \\
which returns a single data object that represents the merged data field
of \ccc{d1} and \ccc{d2}. The $x$-monotone curve that represents the overlap
is associated with the output of this functor.

The traits-class models the \ccc{ArrangementLandmarkTraits_2} only
if \ccc{Tr} models the concept and
defines \ccc{Has_construct_x_monotone_curve_2_category} to identical
to the one in~\ccc{Tr}.

Both \ccc{Compare_at_x_left_2} and \ccc{Has_left_category} are defined
to be identical as the types in \ccc{Tr}.


\ccInclude{CGAL/Arr_curve_data_traits_2.h}
 
\ccIsModel
    \ccc{ArrangementTraits_2} \\
    \ccc{ArrangementLandmarkTraits_2} (only if \ccc{Tr} models the
    \ccc{ArrangementLandmarkTraits_2}.) \\

\ccTypes
%-------

\ccTypedef{typedef Tr Base_traits_2;}
    {the base traits-class.}
\ccGlue
\ccTypedef{typedef typename Base_traits_2::Curve_2 Base_curve_2;}
    {the base curve.}
\ccGlue
\ccTypedef{typedef typename Base_traits_2::X_monotone_curve_2 Base_x_monotone_curve_2;}
    {the base $x$-monotone curve curve.}

\ccTypedef{typedef Mrg Merge;}{the merge functor.}
\ccGlue
\ccTypedef{typedef Cnv Convert;}{the conversion functor.}

\ccTypedef{typedef CData Curve_data;}
    {the type of data associated with curves.}
\ccGlue
\ccTypedef{typedef XData X_monotone_curve_data;}
    {the type of data associated with $x$-monotone curves.}

\ccInheritsFrom
  \ccc{Base_traits_2}

\subsection*{Class Arr\_curve\_data\_traits\_2$<$Tr,XData,Mrg,CData,Cnv$>$::Curve\_2}
%==================================================================

The \ccc{Curve_2} class nested within the curve-data traits
extends the \ccc{Base_traits_2::Curve_2} type with an extra data field of type
\ccc{Data}.

\begin{ccClass}{Arr_curve_data_traits_2<Tr,XData,Mrg,CData,Cnv>::Curve_2}

\ccInheritsFrom
  \ccc{Base_curve_2}

\ccCreation
\ccCreationVariable{cv}
%---------------------

\ccConstructor{Curve_2 ();}
    {default constructor.}

\ccConstructor{Curve_2 (const Base_curve_2& base);}
    {constructs curve from the given \ccc{base} curve with uninitialized
     data field.}

\ccConstructor{Curve_2 (const Base_curve_2& base, const Data& data);}
    {constructs curve from the given \ccc{base} curve with an attached
     \ccc{data} field.}

\ccAccessFunctions
%-----------------

\ccMethod{const Curve_data& data () const;}
  {returns the data field (a non-const version, which returns a reference
   to the data object, is also available).}

\ccMethod{void set_data (const Curve_data& data);}
  {sets the data field.}

\end{ccClass}

\subsection*{Class Arr\_curve\_data\_traits\_2$<$Tr,XData,Mrg,CData,Cnv$>$::X\_monotone\_curve\_2}
%=============================================================================

The \ccc{X_monotone_curve_2} class nested within the
curve-data traits extends the \ccc{Base_traits_2::X_monotone_curve_2} type
with an extra data field.

\begin{ccClass}{Arr_curve_data_traits_2<Tr,XData,Mrg,CData,Cnv>::X_monotone_curve_2}

\ccInheritsFrom
  \ccc{Base_x_monotone_curve_2}

\ccCreation
\ccCreationVariable{xcv}
%-----------------------

\ccConstructor{X_monotone_curve_2 ();}
    {default constructor.}

\ccConstructor{X_monotone_curve_2 (const Base_x_monotone_curve_2& base);}
    {constructs an $x$-monotone curve from the given \ccc{base} curve with
     uninitialized data field.}

\ccConstructor{X_monotone_curve_2 (const Base_x_monotone_curve_2& base, 
                                   const X_monotone_curve_data& data);}
    {constructs an $x$-monotone curve from the given \ccc{base} $x$-monotone
     curve with an attached \ccc{data} field.}

\ccAccessFunctions
%-----------------

\ccMethod{const X_monotone_curve_data& data () const;}
  {returns the field (a non-const version, which returns a reference
   to the data object, is also available).}

\ccMethod{void set_data (const X_monotone_curve_data& data);}
  {sets the data field.}

\end{ccClass}

\end{ccRefClass}

\ccRefPageEnd
