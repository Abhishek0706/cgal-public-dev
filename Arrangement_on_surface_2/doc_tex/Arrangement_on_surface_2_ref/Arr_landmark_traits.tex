% Reference manual page: ArrangementLandmarkTraits.tex
% Package: Arrangement_2

\ccRefPageBegin
\begin{ccRefConcept}{ArrangementLandmarkTraits_2}

\ccDefinition
% ===========
The concept \ccRefName{} refines the general traits concept by adding
operations needed for the landmarks point-location strategy, namely ---
approximating points and (optionally) connecting points with a simple
$x$-monotone curve.

A model of this concept must define the \ccc{Approximate_number_type}, which
is used to approximate the coordinates of \ccc{Point_2} instances. It is
recommended to define the approximated number type as the built-in
\ccc{double} type. 

\ccRefines
\ccc{ArrangementTraits_2}

\ccTypes
% ======
\ccNestedType{Approximate_number_type}%
  {The number type used to approximate point coordinates.}

\ccHeading{Functor Types}
% =======================
\ccThree{Construct_x_monotone_curve_2}{}{\hspace*{14cm}}
\ccThreeToTwo
\ccNestedType{Approximate_2}%
  {Models the concept \ccc{ArrTraits::Approximate_2}.}

\ccNestedType{Construct_x_monotone_curve_2}%
  {Optional. If this type is defined, the walking strategy uses the
  constructed curve to walk in its direction. If it is not defined
  then the walk is first performed horizontally and then vertically.
  Models the concept \ccc{ArrTraits::ConstructXMonotoneCurve_2}.}

\ccCreationVariable{traits}
% \ccCreation
% ===========

\ccHeading{Accessing Functor Objects}
% ===================================
\ccMethod{Approximate_2 approximate_2_object() const;} {}
\ccGlue
\ccMethod{Construct_x_monotone_curve_2 
          construct_x_monotone_curve_2_object() const;}{}

\ccHasModels
% ==========
\ccc{CGAL::Arr_non_caching_segment_traits_2<Kernel>}\\
\ccc{CGAL::Arr_segment_traits_2<Kernel>}\\
\ccc{CGAL::Arr_linear_traits_2<Kernel>}\\
\ccc{CGAL::Arr_polyline_traits_2<SegmentTraits>}\\
\ccc{CGAL::Arr_conic_traits_2<RatKernel,AlgKernel,NtTraits>}

\ccSeeAlso
%=========
\ccc{ArrangementTraits_2}\lcTex{(\ccRefPage{ArrangementTraits_2})}

\end{ccRefConcept}
\ccRefPageEnd

%%%%%%%% Functors %%%%%%%%
% ========================

%%%%%%%% Approximate_2
\ccRefPageBegin
\begin{ccRefConcept}{ArrTraits::Approximate_2}
\ccRefines{Functor}

\ccHasModels\ccc{ArrangementLandmarkTraits_2::Approximate_2}

\ccCreationVariable{fo}

\ccMethod{Approximate_number_type operator()( ArrTraits::Point_2 p,
                                              int i);}{%
  returns an approximation of \ccc{p}'s $x$-coordinate (if \ccc{i == 0}),
  or of \ccc{p}'s $y$-coordinate (if \ccc{i == 1}).}
\end{ccRefConcept}
\ccRefPageEnd

%%%%%%%% ConstructXMonotoneCurve_2
\ccRefPageBegin
\begin{ccRefConcept}{ArrTraits::ConstructXMonotoneCurve_2}
\ccRefines{Functor}

\ccHasModels\ccc{ArrangementLandmarkTraits_2::Construct_x_monotone_curve_2}

\ccCreationVariable{fo}

\ccMethod{ArrTraits::X_monotone_curve_2 operator() ( ArrTraits::Point_2 p1,
                                                     ArrTraits::Point_2 p2);}{%
  returns an $x$-monotone curve connecting \ccc{p1} and \ccc{p2} (i.e., the
  two input points are its endpoints).}
\end{ccRefConcept}
\ccRefPageEnd

