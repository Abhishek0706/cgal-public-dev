% +------------------------------------------------------------------------+
% | Reference manual page: Arr_trapezoid_ric_point_location.tex
% +------------------------------------------------------------------------+
% | 
% | Package: Arrangement_2
% | 
% +------------------------------------------------------------------------+

\ccRefPageBegin

\begin{ccRefClass}{Arr_trapezoid_ric_point_location<Arrangement>}
\label{arr_ref:trap_pl}

\ccDefinition
%============

The \ccRefName\ class implements the incremental randomized algorithm
introduced by Mulmuley~\cite{m-fppa-90} as presented by
Seidel~\cite{s-sfira-91} (see also~\cite[Chapter~6]{bkos-cgaa-00}).
It subdivides each arrangement face to pseudo-trapezoidal cells, each
of constant complexity, and constructs and maintains a linear-size search
structure on top of these cells, such that each query can be answered
in $O(\log n)$ time, where $n$ is the complexity of the arrangement.

Constructing the search structures takes $O(n \log n)$ expected time 
and may require a small number of rebuilds~\cite{hkh-iiplgtds-12}. Therefore
attaching a trapezoidal point-location object to an existing arrangement
may incur some overhead in running times. In addition, the point-location
object needs to keep its auxiliary data structures up-to-date as the
arrangement goes through structural changes. It is therefore recommended
to use this point-location strategy for static arrangements (or arrangement
that do not alter frequently), and when the number of issued queries
is relatively large.

This strategy supports arbitrary subdivisions, including unbounded ones.

\ccInclude{CGAL/Arr_trapezoid_ric_point_location.h}

\ccIsModel
  \ccc{ArrangementPointLocation_2} \\
  \ccc{ArrangementVerticalRayShoot_2}

\ccCreation
\ccCreationVariable{pl}
%---------------------

\ccConstructor{Arr_trapezoid_ric_point_location (bool with_guarantees = true);}
{If with\_guarantees is set to true, the cunstruction performs rebuilds in order to guarantee a resulting structure with linear size and logarithmic query time. Otherwise the structure has expected linear size and expected logarithmic query time.}
\ccConstructor{Arr_trapezoid_ric_point_location (const Arrangement& arr, bool with_guarantees = true);}
{Constructs a point location search structure for the given arrangement. If with\_guarantees is set to true, the cunstruction performs rebuilds in order to guarantee a resulting structure with linear size and logarithmic query time. Otherwise the structure has expected linear size and expected logarithmic query time.}

\ccModifiers
%===========

\ccMethod{void with_guarantees (bool with_guarantees);}
  {If with\_guarantees is set to true, the structure will guarantee linear size and logarithmic query time, that is, this function may cause a reconstruction of the data structure. }


\end{ccRefClass}

\ccRefPageEnd
