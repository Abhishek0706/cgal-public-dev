%% Copyright (c) 2003,2004,2005  INRIA Sophia-Antipolis (France).
%% All rights reserved.
%%
%% This file is part of CGAL (www.cgal.org).
%% You can redistribute it and/or modify it under the terms of the GNU
%% General Public License as published by the Free Software Foundation,
%% either version 3 of the License, or (at your option) any later version.
%%
%% Licensees holding a valid commercial license may use this file in
%% accordance with the commercial license agreement provided with the software.
%%
%% This file is provided AS IS with NO WARRANTY OF ANY KIND, INCLUDING THE
%% WARRANTY OF DESIGN, MERCHANTABILITY AND FITNESS FOR A PARTICULAR PURPOSE.
%%
%% $URL$
%% $Id$
%% 
%%
%% Author(s)     : Menelaos Karavelas <mkaravel@iacm.forth.gr>



\begin{ccRefConcept}{SegmentDelaunayGraphSite_2} 
%% add template arg's if necessary

%% \ccHtmlCrossLink{}     %% add further rules for cross referencing links
%% \ccHtmlIndexC[class]{} %% add further index entries

\ccDefinition
  
The concept \ccc{SegmentDelaunayGraphSite_2} provides the
requirements for the sites of a segment Delaunay graph.

\ccRefines
\ccc{DefaultConstructible}\\
\ccc{CopyConstructible}\\
\ccc{Assignable}

\ccTwo{SegmentDelaunayGraphSite_2::Segment_2+}{}
\ccTypes
\ccNestedType{Point_2}{The point type.}
\ccGlue
\ccNestedType{Segment_2}{The segment type.}
\ccGlue
\ccNestedType{FT}{The field number type.}
\ccGlue
\ccNestedType{RT}{The ring number type.}




\ccCreation
\ccThree{SegmentDelaunayGraphSite_2+}{construct_site_2(Point_2 p)+}{}


\ccCreationVariable{s}  %% choose variable name

In addition to the default and copy constructors the following static
methods are available for constructing sites:

\ccMethod{SegmentDelaunayGraphSite_2
  construct_site_2(Point_2 p);}{Constructs a site from a point: the site
  represents the point \ccc{p}.}
%
\ccGlue
\ccMethod{SegmentDelaunayGraphSite_2 construct_site_2(Point_2 p1,
  Point_2 p2);}{Constructs a site from two points: the site represents
  the (open) segment \ccc{(p1,p2)}.}
%
\ccGlue
\ccMethod{SegmentDelaunayGraphSite_2 construct_site_2(Point_2 p1,
  Point_2 p2, Point_2 q1, Point_2 q2);}%
{Constructs a site from four points: the site represents the
  point of intersection of the segments \ccc{(p1,p2)} and
  \ccc{(q1,q2)}.}
%
\ccGlue
\ccMethod{SegmentDelaunayGraphSite_2 construct_site_2(Point_2 p1,
  Point_2 p2, Point_2 q1,
  Point_2 q2, bool b);}{Constructs a site from four points and a boolean: the
  site represents a segment. If \ccc{b} is \ccc{true} the endpoints
  are \ccc{p1} and $p_\times$, otherwise $p_\times$ and
  \ccc{p2}. $p_\times$ is the point of intersection of the segments
  \ccc{(p1,p2)},\ccc{(q1,q2)}.}

%
\ccGlue
\ccMethod{SegmentDelaunayGraphSite_2 construct_site_2(Point_2 p1,
  Point_2 p2, Point_2 q1,
  Point_2 q2, Point_2 r1, Point_2 r2);}{Constructs a site from six
  points: the site represents the segment with endpoints the points of
  intersection of the pairs of segments \ccc{(p1,p2)},\ccc{(q1,q2)}
  and \ccc{(p1,p2)},\ccc{(r1,r2)}.}

\ccPredicates
\ccThree{bool}{s.is_input(unsigned int i)+}{}
\ccMethod{bool is_defined();}{Returns \ccc{true} if the site
  represents a valid point or segment.}
%
\ccGlue
\ccMethod{bool is_point();}{Returns \ccc{true} if the site represents
  a point.}
%
\ccGlue
\ccMethod{bool is_segment();}{Returns \ccc{true} if the site
  represents a segment.}
%
\ccGlue
\ccMethod{bool is_input();}{Returns \ccc{true} if the site
  represents an input point or a segment defined by two input
  points. Returns \ccc{false} if it represents a point of intersection
  of two segments, or if it represents a segment, at least one
  endpoint of which is a point of intersection of two segments.}
%
\ccGlue
\ccMethod{bool is_input(unsigned int i);}{Returns \ccc{true} if the
  \ccc{i}-th endpoint of the site is an input point. Returns \ccc{false}
  if the \ccc{i}-th endpoint of the site is the intersection of two
  segments.
  \ccPrecond{\ccc{i} must be at most $1$, and \ccc{s.is_segment()}
    must be \ccc{true}.}}




\ccAccessFunctions
%\ccThree{Segment_2}{s.point(unsigned int i)+}{}
\ccThree{Segment_2}{s.segment()+}{}
%% \ccMethod{Point_2 point(unsigned int i) const;}{
%%   Returns the \ccc{i}-th point of the site's representation. The valid
%%   values for \ccc{i} are $0$ through the number of points in the
%%   \ccc{construct_site_2} used to construct the site.
%%   \ccPrecond{\ccc{i} must be at most $5$. The site must be of the
%%     correct type in order to access the corresponding point.}}
%% %
%% \ccGlue
\ccMethod{Point_2 point() const;}{Returns the point represented by the
  site \ccc{s}.
  \ccPrecond{ \ccc{s.is_point()} must be \ccc{true}.}}
%
\ccGlue
\ccMethod{Segment_2 segment() const;}{Returns the segment represented
  by the site \ccc{s}.
  \ccPrecond{ \ccc{s.is_segment()} must be \ccc{true}.}}
%
\ccGlue
\ccMethod{Point_2 source() const;}{Returns the source endpoint of the
  segment. Note that this method can construct an inexact point if the
  number type used is inexact.
  \ccPrecond{\ccc{s.is_segment()} must be \ccc{true}.}}
%
\ccGlue
\ccMethod{Point_2 target() const;}{Returns the target endpoint of the
  segment. Note that this method can construct an inexact point if the
  number type used is inexact.
  \ccPrecond{\ccc{s.is_segment()} must be \ccc{true}.}}
%
%\ccThree{SegmentDelaunayGraphSite_2}{s.ps}{}
\ccThree{SegmentDelaunayGraphSite_2}{}{}
\ccGlue
\ccMethod{SegmentDelaunayGraphSite_2 supporting_site();}
         {Returns a segment site object representing the segment
           that supports the segment represented by the site. Both
           endpoints of the returned site are input points.
  \ccPrecond{\ccc{s.is_segment()} must be \ccc{true}.}}
%
\ccGlue
\ccMethod{SegmentDelaunayGraphSite_2 supporting_site(unsigned int i);}
         {Returns a segment site object representing the \ccc{i}-th
           segment that supports the point of intersection represented
           by the site. Both endpoints of the returned site are input
           points.
  \ccPrecond{\ccc{i} must be at most $1$, \ccc{s.is_point()} must be
    \ccc{true} and \ccc{s.is_input()} must be \ccc{false}.}}
%
\ccGlue
\ccMethod{SegmentDelaunayGraphSite_2 crossing_site(unsigned int i);}
         {Returns a segment site object representing the \ccc{i}-th
           segment that supports the $i$-th endpoint of the site
           which is not the supporting segment of the site. Both
           endpoints of the returned site are input points.
           \ccPrecond{\ccc{i} must be at most $1$,
             \ccc{s.is_segment()} must be \ccc{true} and
             \ccc{s.is_input(i)} must be \ccc{false}.}}
%
\ccGlue
\ccMethod{SegmentDelaunayGraphSite_2 source_site();}
         {Returns a point site object representing the source point of
           the site.
         \ccPrecond{\ccc{s.is_segment()} must be \ccc{true}.}}
%
\ccGlue
\ccMethod{SegmentDelaunayGraphSite_2 target_site();}
         {Returns a point site object representing the target point of
           the site.
         \ccPrecond{\ccc{s.is_segment()} must be \ccc{true}.}}
%
\ccGlue
\ccMethod{Point_2 source_of_supporting_site();}
{ Returns the source point of the supporting site of the this site.
\ccPrecond{\ccc{is_segment()} must be \ccc{true}.}}
%
\ccGlue
\ccMethod{Point_2 target_of_supporting_site();}
{ Returns the target point of the supporting site of the this site.
\ccPrecond{\ccc{is_segment()} must be \ccc{true}.}}
%
\ccGlue
\ccMethod{Point_2 source_of_supporting_site(unsigned int i);}
{ Returns the source point of the \ccc{i}-th supporting site of the
  this site.
\ccPrecond{\ccc{is_point()} must be \ccc{true}, \ccc{is_input()}
  must be \ccc{false} and \ccc{i} must either be \ccc{0} or \ccc{1}.}}
%
\ccGlue
\ccMethod{Point_2 target_of_supporting_site(unsigned int i);}
{ Returns the target point of the \ccc{i}-th supporting site of the
  this site.
\ccPrecond{\ccc{is_point()} must be \ccc{true}, \ccc{is_input()}
  must be \ccc{false} and \ccc{i} must either be \ccc{0} or \ccc{1}.}}
%
\ccGlue
\ccMethod{Point_2 source_of_crossing_site(unsigned int i);}
{ Returns the source point of the \ccc{i}-th crossing site of the
  this site.
\ccPrecond{\ccc{is_segment()} must be \ccc{true}, \ccc{is_input(i)}
  must be \ccc{false} and \ccc{i} must either be \ccc{0} or \ccc{1}.}}
%
\ccGlue
\ccMethod{Point_2 target_of_crossing_site(unsigned int i);}
{ Returns the target point of the \ccc{i}-th supporting site of the
  this site.
\ccPrecond{\ccc{is_segment()} must be \ccc{true}, \ccc{is_input(i)}
  must be \ccc{false} and \ccc{i} must either be \ccc{0} or \ccc{1}.}}
%% %
%% \ccGlue
%% \ccMethod{SegmentDelaunayGraphSite_2 opposite_site();}
%%          {Returns a segment site object representing the segment site
%%            with its endpoints reversed.
%%          \ccPrecond{\ccc{s.is_segment()} must be \ccc{true}.}}




%% \ccHeading{Set methods}
%% \ccThree{void}{s.set_point(Point_2 p)+}{}
%% %\ccThree{void}{s.set_segment(Point_2 p1, Point_2 p2)+}{}
%% \ccMethod{void set_point(Point_2 p);}
%%          {Sets the site to be a point site with \ccc{p} as the
%%            point.}
%% %
%% %\ccGlue
%% \ccMethod{void set_point(Point_2 p1, Point_2 p2, Point_2 q1, Point_2 q2);}
%%          {Sets the site to be a point site with the point being the
%%            intersection point of the segments \ccc{(p1,p2)} and
%%            \ccc{(q1,q2)}.}
%% %
%% \ccGlue
%% \ccMethod{void set_segment(Point_2 p1, Point_2 p2);}
%%          {Sets the site to be an segment site with \ccc{(p1,p2)}
%%            as the segment.}
%% %
%% \ccGlue
%% \ccMethod{void set_segment(Point_2 p1, Point_2 p2, Point_2 q1, Point_2
%%   q2, bool b);}
%%          {Sets the site to be a segment site. If \ccc{b} is
%%            \ccc{true} the endpoints of the segment are \ccc{p1} and
%%            the point of intersection of the segments \ccc{(p1,p2)} and
%%            \ccc{(q1,q2)}. If \ccc{b} is \ccc{false} the endpoints of
%%            the segment are the point of intersection of the segments
%%            \ccc{(p1,p2)} and \ccc{(q1,q2)} and \ccc{p2}.}
%% %
%% \ccGlue
%% \ccMethod{void set_segment(Point_2 p1, Point_2 p2, Point_2 q1, Point_2
%%   q2, Point_2 r1, Point_2 r2);}
%%          {Sets the site to be a segment site with endpoints the
%%            points of intersection of the pairs of segments
%%            \ccc{(p1,p2)}, \ccc{(q1,q2)} and \ccc{(p1,p2)},
%%            \ccc{(r1,r2)}.}



\ccHasModels
\ccc{CGAL::Segment_Delaunay_graph_site_2<K>}


\ccSeeAlso
\ccc{SegmentDelaunayGraphTraits_2}\\
\ccc{CGAL::Segment_Delaunay_graph_site_2<K>}\\
\ccc{CGAL::Segment_Delaunay_graph_traits_2<K,MTag>}\\
\ccc{CGAL::Segment_Delaunay_graph_traits_without_intersections_2<K,MTag>}\\
\ccc{CGAL::Segment_Delaunay_graph_filtered_traits_2<CK,CM,EK,EM,FK,FM>}\\
\ccc{CGAL::Segment_Delaunay_graph_filtered_traits_without_intersections_2<CK,CM,EK,EM,FK,FM>}

\end{ccRefConcept}

% +------------------------------------------------------------------------+
%%RefPage: end of main body, begin of footer
% EOF
% +------------------------------------------------------------------------+

