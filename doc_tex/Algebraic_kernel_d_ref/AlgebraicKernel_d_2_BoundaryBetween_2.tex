\begin{ccRefFunctionObjectConcept}{AlgebraicKernel_d_2::BoundaryBetweenX_2}

\ccDefinition
Computes a number of type 
\ccc{AlgebraicKernel_d_1::Boundary} in-between the first coordinates of two 
\ccc{AlgebraicKernel_d_2::AlgebraicReal_2}.

\ccRefines 
\ccc{AdaptableBinaryFunction} 

\ccTypes
\ccThree{typedef AlgebraicKernel_d_2::Algebraic_real_2}{second_argument_type+}{}
\ccTypedef{typedef AlgebraicKernel_d_1::Boundary         result_type;}{}
\ccGlue
\ccTypedef{typedef AlgebraicKernel_d_2::Algebraic_real_2 first_argument_type;}{}
\ccGlue
\ccTypedef{typedef AlgebraicKernel_d_2::Algebraic_real_2 second_argument_type;}{}

\ccOperations
\ccCreationVariable{fo}
A model \ccVar\ of this type must provide:

\ccThree{result_type}{fo(first_argument_type,++}{}
\ccMethod{
result_type
operator()(const first_argument_type & a,
           const second_argument_type & b);}{
Computes a number of type \ccc{AlgebraicKernel_d_1::Boundary}
in-between the first coordinates of $a$ and $b$.
} 

\ccSeeAlso
\ccRefIdfierPage{AlgebraicKernel_d_2::AlgebraicReal_2}\\
\ccRefIdfierPage{AlgebraicKernel_d_1::Boundary}\\
\ccRefIdfierPage{AlgebraicKernel_d_2::LowerBoundaryX_2}\\
\ccRefIdfierPage{AlgebraicKernel_d_2::LowerBoundaryY_2}\\
\ccRefIdfierPage{AlgebraicKernel_d_2::UpperBoundaryX_2}\\
\ccRefIdfierPage{AlgebraicKernel_d_2::UpperBoundaryY_2}\\
\ccRefIdfierPage{AlgebraicKernel_d_2::BoundaryBetweenY_2}\\

\end{ccRefFunctionObjectConcept}


\begin{ccRefFunctionObjectConcept}{AlgebraicKernel_d_2::BoundaryBetweenY_2}

\ccDefinition
Computes a number of type 
\ccc{AlgebraicKernel_d_1::Boundary} in-between the second coordinates of two 
\ccc{AlgebraicKernel_d_2::AlgebraicReal_2}.

\ccRefines 
\ccc{AdaptableBinaryFunction} 

\ccTypes
\ccThree{typedef AlgebraicKernel_d_2::Algebraic_real_2}{second_argument_type+}{}
\ccTypedef{typedef AlgebraicKernel_d_1::Boundary         result_type;}{}
\ccGlue
\ccTypedef{typedef AlgebraicKernel_d_2::Algebraic_real_2 first_argument_type;}{}
\ccGlue
\ccTypedef{typedef AlgebraicKernel_d_2::Algebraic_real_2 second_argument_type;}{}

\ccOperations
\ccCreationVariable{fo}
A model \ccVar\ of this type must provide:

\ccThree{result_type}{fo(first_argument_type,++}{}
\ccMethod{
result_type
operator()(const first_argument_type & a,
           const second_argument_type & b);}{
Computes a number of type \ccc{AlgebraicKernel_d_1::Boundary}
in-between the second coordinates of $a$ and $b$.
} 

\ccSeeAlso
\ccRefIdfierPage{AlgebraicKernel_d_2::AlgebraicReal_2}\\
\ccRefIdfierPage{AlgebraicKernel_d_1::Boundary}\\
\ccRefIdfierPage{AlgebraicKernel_d_2::LowerBoundaryX_2}\\
\ccRefIdfierPage{AlgebraicKernel_d_2::LowerBoundaryY_2}\\
\ccRefIdfierPage{AlgebraicKernel_d_2::UpperBoundaryX_2}\\
\ccRefIdfierPage{AlgebraicKernel_d_2::UpperBoundaryY_2}\\
\ccRefIdfierPage{AlgebraicKernel_d_2::BoundaryBetweenX_2}\\

\end{ccRefFunctionObjectConcept}
