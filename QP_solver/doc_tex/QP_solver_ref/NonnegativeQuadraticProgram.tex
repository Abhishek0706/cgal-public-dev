\begin{ccRefConcept}{NonnegativeQuadraticProgram}

\ccDefinition
A model of \ccRefName\ describes a convex quadratic program of the form
\input{QP_solver_ref/_nqp_description}

The description is given by appropriate \emph{random-access} 
iterators over the program data, see below. The program therefore 
comes in \emph{dense} representation which includes zero entries.

\ccHasModels
\ccc{Quadratic_program<NT>}\\
\ccc{Quadratic_program_from_mps<NT>}\\
\ccc{Nonnegative_quadratic_program_from_sparse_iterators<A_s_it, B_it, R_it, FL_it, L_it, FU_it, U_it, D_s_it, C_it>}\\
\ccc{Nonnegative_quadratic_program_from_iterators<A_it, B_it, R_it, FL_it, L_it, FU_it, U_it, D_it, C_it>}\\

\ccTypes

\ccNestedType{A_sparse_iterator}{A random access iterator type to go 
  columnwise over the constraint matrix $A$. The value type
  is an object that provides bidirectional sparse iterators
  for the column in question
  by member calls to \ccc{begin()} and \ccc{end()}.
  Such a column iterator \ccc{it} is sparse, providing \ccc{(index,value)}
  pairs of all non-zero elements of the column. The index is accessed
  by \ccc{it->first} and the value is accessed by \ccc{it->second}.
  }
  
\ccNestedType{A_iterator}{A random access iterator type to go 
  columnwise over the constraint matrix $A$. The value type
  is an random access iterator type for an individual column that
  goes over the entries in that column.}

\ccNestedType{B_iterator}{A random access iterator type to go over 
  the entries of the right-hand side $\qpb$.}

\ccNestedType{R_iterator}{A random access iterator type to go over the
  relations $\qprel$. The value type of \ccc{R_iterator} is
  \ccc{CGAL::Comparison_result}.}


\ccNestedType{D_sparse_iterator}{A random access iterator type to go 
  rowwise over the matrix $2D$. The value type
  is an object that provides bidirectional sparse iterators
  for the row in question
  by member calls to \ccc{begin()} and \ccc{end()}.
  Such a column iterator \ccc{it} is sparse, providing \ccc{(index,value)}
  pairs of all non-zero elements of the row. The index is accessed
  by \ccc{it->first} and the value is accessed by \ccc{it->second}.
  }

\ccNestedType{D_iterator}{A random access iterator type to go rowwise 
  over the matrix $2D$. The value type
  is a random access iterator type for an individual row that
  goes over the entries in that row, up to (and including) the
  entry on the main diagonal.}


\ccNestedType{C_iterator}{A random access iterator type to go over the
  entries of the linear objective function vector $\qpc$.}


\ccOperations

\ccCreationVariable{qp}

\ccMethod{int get_n() const;}{returns the number $n$ of variables (number
  of columns of $A$) in \ccVar.}

\ccMethod{int get_m() const;}{returns the number $m$ of constraints
  (number of rows of $A$) in \ccVar.}

\ccMethod{A_sparse_iterator get_a_sparse() const;}{returns an iterator over the columns
  of $A$. The corresponding past-the-end iterator is \ccc{get_a_sparse()+get_n()}.
  For $j=0,\ldots,n-1$, $\ccc{*(get_a()+j)}$ is an object
  that provides bidirectional sparse
  iterators for column $j$ by calls to \ccc{begin()} and \ccc{end()}.
  These column iterators provide \ccc{(index,value)} pairs for all non-zero
  entries.}

\ccMethod{B_iterator get_b() const;}{returns an iterator over the entries
  of $\qpb$. The corresponding past-the-end iterator is 
        \ccc{get_b()+get_m()}.}

\ccMethod{R_iterator get_r() const;}{returns an iterator over the entries
  of $\qprel$. The corresponding past-the-end iterator is 
        \ccc{get_r()+get_m()}.
  The value \ccc{CGAL::SMALLER} stands
  for $\leq$, \ccc{CGAL::EQUAL} stands for $=$, and \ccc{CGAL::LARGER}
  stands for $\geq$.}


\ccMethod{D_sparse_iterator get_d_sparse() const;}{returns an iterator over the rows of
  $2D$. The corresponding past-the-end iterator is \ccc{get_d_sparse()+get_n()}.
  For $i=0,\ldots,n-1$, $\ccc{*(get_d()+i)}$ is an object
  that provides bidirectional sparse
  iterators for row $i$ by calls to \ccc{begin()} and \ccc{end()}.
  These column iterators provide \ccc{(index,value)} pairs for all non-zero
  entries.}


\input{QP_solver_ref/_c_methods.tex}

\ccRequirements

The value types of all iterator types (nested iterator types
for \ccc{A_iterator} and \ccc{D_iterator},
and the types of \ccc{value} in the nested sparse iterators of
\ccc{A_sparse_iterator} and \ccc{D_sparse_iterator}, respectively) must be
convertible to some common \ccc{IntegralDomain} \ccc{ET}.

\ccSeeAlso
The models

\ccc{Quadratic_program<NT>}\\
\ccc{Quadratic_program_from_mps<NT>}\\
\ccc{Nonnegative_quadratic_program_from_sparse_iterators<A_s_it, B_it, R_it, FL_it, L_it, FU_it, U_it, D_s_it, C_it>}\\
\ccc{Nonnegative_quadratic_program_from_iterators<A_it, B_it, R_it, D_it, C_it>}

and the other concepts

\ccc{QuadraticProgram}\\
\ccc{LinearProgram}\\
\ccc{NonnegativeLinearProgram}
\end{ccRefConcept}
