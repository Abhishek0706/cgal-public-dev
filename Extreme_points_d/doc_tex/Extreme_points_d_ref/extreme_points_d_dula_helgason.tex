% +------------------------------------------------------------------------+
% | Reference manual page: extreme_points_d_dula_helgason.tex
% +------------------------------------------------------------------------+
% | 14.9.2010   Christian Helbling
% | Package: Extreme_points_d
% |
% \RCSdef{\RCSextremepointsddulahelgasonRev}{$Id: $}
% \RCSdefDate{\RCSextremepointsddulahelgasonDate}{$Date: $}
% |
\ccRefPageBegin
%%RefPage: end of header, begin of main body
% +------------------------------------------------------------------------+


\begin{ccRefFunction}{extreme_points_d_dula_helgason}  %% add template arg's if necessary

%% \ccHtmlCrossLink{}     %% add further rules for cross referencing links
%% \ccHtmlIndexC[function]{} %% add further index entries

\ccDefinition
  
The function \ccRefName\ computes the extreme points of the given set of input points.

\ccInclude{CGAL/Extreme_points_d.h}

\ccFunction{template <class InputIterator, class OutputIterator, class ExtremePointsTraits_d>
OutputIterator
extreme_points_d_dula_helgason(InputIterator first, InputIterator beyond, OutputIterator result,ExtremePointsTraits_d ep_traits = Default_traits);}{computes the extreme points of the point set in the range [\ccc{first},\ccc{beyond}). The resulting sequence of extreme points is placed starting at position \ccc{result}, and the past-the-end iterator for the resulting sequence is returned.
}

The default traits class \ccc{Default_traits} is \ccc{Extreme_points_traits_d<Point>} where \ccc{Point} is \ccc{InputIterator::value_type}

\ccRequirements

\ccc{InputIterator::value_type} and \ccc{OutputIterator::value_type} are equivalent to \ccc{ExtremePointsTraits_d::Point}.

\ccSeeAlso

\ccRefIdfierPage{CGAL::Extreme_points_d<Traits>}\\
\ccRefIdfierPage{CGAL::extreme_points_d_simple}\\
\ccRefIdfierPage{CGAL::extreme_points_d}

\ccImplementation
% TODO: where should the bibtex definition of the referenced papers go?
This function implements the extreme points algorithm from Dul\'a and Helgason \cite{cgal:dh-pifch-96} as also described in \cite{cgal:dl-cosfa-08} and \cite{cgal:h-epmhd-10}.

This algorithm requires $O(d n m + n * LP_{d+1,m})$ time in the worst case where $n$ is the number of input points, $m$ the number of extreme points, $d$ the dimension and $LP_{a,b}$ the runtime for solving a linear program with $a$ equality constraints and $b$ nonnegative variables using \cgal's QP solver package.

\ccExample
\ccReferToExampleCode{Extreme_points_d/extreme_points_d_dula_helgason.cpp}

\end{ccRefFunction}

% +------------------------------------------------------------------------+
%%RefPage: end of main body, begin of footer
\ccRefPageEnd
% EOF
% +------------------------------------------------------------------------+

