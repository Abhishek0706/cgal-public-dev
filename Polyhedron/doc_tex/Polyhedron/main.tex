% +------------------------------------------------------------------------+
% | CGAL Reference Manual:  polyhedron.tex
% +------------------------------------------------------------------------+
% | Combinatoric and geometry of polyhedral surfaces in 
% | halfedge representation.
% |
% | 11.10.1996   Lutz Kettner
% |              Start rewriting the whole stuff
% | 
\RCSdef{\polyhedronRev}{$Id$}
\RCSdefDate{\polyhedronDate}{$Date$}
% +------------------------------------------------------------------------+

\ccParDims

\ccUserChapter{3D Polyhedral Surfaces \label{chapterPolyhedron}}

\ccChapterRelease{\polyhedronRev. \ \polyhedronDate}
\ccChapterAuthor{Lutz Kettner}


\begin{ccPkgDescription}{3D Skin Surface Meshing \label{Pkg:SkinSurface3}}
\ccPkgHowToCiteCgal{cgal:k-ssm3-12}
  \ccPkgSummary{ %
    This package allows to build a triangular mesh of a skin surface.
    Skin surfaces are used for modeling large molecules in biological
    computing. The surface is defined by a set of balls, representing
    the atoms of the molecule, and a shrink factor that determines the
    size of the smooth patches gluing the balls together.

    The construction of a triangular mesh of a smooth skin surface is
    often necessary for further analysis and for fast visualization.
    This package provides functions to construct a triangular mesh
    approximating the skin surface from a set of balls and a shrink
    factor. It also contains code to subdivide the mesh efficiently.
    % 
  }
%
  \ccPkgIntroducedInCGAL{3.3}
  \ccPkgDependsOn{\ccRef[3D Triangulation]{Pkg:Triangulation3} and \ccRef[3D Polyhedral Surface]{Pkg:Polyhedron}}
  \ccPkgLicense{\ccLicenseGPL}
  \ccPkgIllustration{Skin_surface_3/small.png}{Skin_surface_3/large.png}
\end{ccPkgDescription}


\begin{ccTexOnly}
    \setlength{\unitlength}{1mm}
    \begin{picture}(0,0)(0.0,0.0)
      \put (78,25){% textwidth = 156mm
          \includegraphics[width=0.5\textwidth]{Polyhedron/fig/shark}
      }
    \end{picture}\vspace{-4mm}% compensate for some vspace added by picture
\end{ccTexOnly}

\minitoc

% +------------------------------------------------------------------------+
\section{Introduction\label{sectionPolyIntro}}

% \begin{ccTexOnly}
%     \vspace*{-50mm}
%     \begin{flushright}~\hspace{1cm}
%       \parbox{0.5\textwidth}{%
%           \includegraphics[width=0.5\textwidth]{Polyhedron/fig/shark}%
%       }%
%     \end{flushright}
% \end{ccTexOnly}

Polyhedral surfaces in three dimensions are composed of vertices,
edges, facets and an incidence relationship on them. The organization
beneath is a halfedge data structure, which restricts the class of
representable surfaces to orientable 2-manifolds -- with and without
boundary. If the surface is closed we call it a {\em polyhedron}, for
example, see the \ccTexHtml{above}{following} model of a hammerhead:

\begin{ccHtmlOnly}
    <CENTER>
        <img src="./fig/shark.gif" alt="Hammerhead"><P>
    </CENTER>
\end{ccHtmlOnly}

The polyhedral surface is realized as a container class that manages
vertices, halfedges, facets with their incidences, and that maintains
the combinatorial integrity of them. It is based on the highly
flexible design of the halfedge data structure, see the introduction
in Chapter~\ref{chapterHalfedgeDS} and~\cite{k-ugpdd-99}. However, the
polyhedral surface can be used and understood without knowing the
underlying design. Some of the examples in this chapter introduce also
gradually into first applications of this flexibility.

% +========================================================================+
\section{Definition}
% +========================================================================+
  
A polyhedral surface \ccc{CGAL::Polyhedron_3<PolyhedronTraits_3>} in
three dimensions consists of vertices $V$, edges $E$, facets $F$ and
an incidence relation on them.  Each edge is represented by two
halfedges with opposite orientations. The incidences stored with a
halfedge are illustrated in the following figure:

\begin{ccTexOnly}
    \vspace{-7mm}
    \begin{center}
      \parbox{0.4\textwidth}{%
          \includegraphics[width=0.4\textwidth]{Polyhedron/fig/halfedge}%
      }
    \end{center}
    \vspace{-5mm}
\end{ccTexOnly}

\begin{ccHtmlOnly}
    <CENTER>
    <A HREF="./fig/halfedge.gif">
        <img src="./fig/halfedge_small.gif" alt="Halfedge Diagram"></A><P>
    </CENTER>
\end{ccHtmlOnly}

Vertices represent points in space. Edges are straight line segments
between two endpoints. Facets are planar polygons without
holes. Facets are defined by the circular sequence of halfedges along
their boundary.  The polyhedral surface itself can have holes (with at
least two facets surrounding it since a single facet cannot have a
hole). The halfedges along the boundary of a hole are called {\em
border halfedges\/} and have no incident facet. An edge is a {\em
border edge\/} if one of its halfedges is a border halfedge.  A
surface is {\em closed\/} if it contains no border halfedges. A closed
surface is a boundary representation for polyhedra in three
dimensions. The convention is that the halfedges are oriented
counterclockwise around facets as seen from the outside of the
polyhedron. An implication is that the halfedges are oriented
clockwise around the vertices. The notion of the solid side of a facet
as defined by the halfedge orientation extends to polyhedral surfaces
with border edges although they do not define a closed object. If
normal vectors are considered for the facets, normals point outwards
(following the right-hand rule).

The strict definition can be found in~\cite{k-ugpdd-99}. One
implication of this definition is that the polyhedral surface is
always an orientable and oriented 2-manifold with border edges, i.e.,
the neighborhood of each point on the polyhedral surface is either
homeomorphic to a disc or to a half disc, except for vertices where
many holes and surfaces with boundary can join. Another implication is
that the smallest representable surface avoiding self intersections is
a triangle (for polyhedral surfaces with border edges) or a
tetrahedron (for polyhedra). Boundary representations of orientable
2-manifolds are closed under Euler operations. They are extended with
operations that create or close holes in the surface.

Other intersections besides the incidence relation are not allowed.
However, this is not automatically verified in the operations, since
self intersections are not easy to check
efficiently. \ccc{CGAL::Polyhedron_3<PolyhedronTraits_3>} does only
maintain the combinatorial integrity of the polyhedral surface (using
Euler operations) and does not consider the coordinates of the points
or any other geometric information.

\ccc{CGAL::Polyhedron_3<PolyhedronTraits_3>} can represent polyhedral
surfaces as well as polyhedra. The interface is designed in such a way
that it is easy to ignore border edges and work only with polyhedra.


% +========================================================================+
\section{Example Programs}
% +========================================================================+
\label{sectionPolyExamples}

The polyhedral surface is based on the highly flexible design of the
halfedge data structure. Examples for this flexibility can be found in
Section~\ref{sectionPolyExtend} and in Section~\ref{sectionHdsExamples}. 
This design is not a prerequisite to understand the following examples.
See also the Section~\ref{sectionPolyAdvanced} below for some advanced 
examples.

% +-------------------------------------------------------------+
\subsection{First Example Using Defaults}

The first example instantiates a polyhedron using a kernel as traits
class. It creates a tetrahedron and stores the reference to one of its
halfedges in a \ccc{Halfedge_handle}. Handles, also know as
{\em trivial iterators}, are used to keep references to halfedges,
vertices, or facets for future use. There is also a \ccc{Halfedge_iterator}
type for enumerating halfedges. Such an iterator type can be used 
wherever a handle is required. Respective \ccc{Halfedge_const_handle} and
\ccc{Halfedge_const_iterator} for a constant polyhedron and similar
handles and iterators with \ccc{Vertex_} and \ccc{Facet_} prefix
are provided too.

The example continues with a test if the halfedge
actually refers to a tetrahedron. This test checks the connected 
component referred to by the halfedge \ccc{h} and not the polyhedral
surface as a whole. This examples works only on the combinatorial
level of a polyhedral surface. The next example adds the geometry.

\ccIncludeExampleCode{Polyhedron/polyhedron_prog_simple.cpp}

% +-------------------------------------------------------------+
\subsection{Example with Geometry in Vertices}

We add geometry to the our construction of a tetrahedron. Four points
are passed as arguments to the construction. This example demonstrates
in addition the use of the vertex iterator and the access to the point
in the vertices. Note the natural access notation \ccc{v->point()}.
Similarly, all information stored in a vertex, halfedge, and facet can
be accessed with a member function given a handle or iterator. For
example, given a halfedge handle \ccc{h} we can write \ccc{h->next()}
to get a halfedge handle to the next halfedge, \ccc{h->opposite()} for
the opposite halfedge, \ccc{h->vertex()} for the incident vertex at
the tip of \ccc{h}, and so on.  The output of the program will be
``\verb|1 0 0\n0 1 0\n0 0 1\n0 0 0\n|''.

%\newpage
\ccIncludeExampleCode{Polyhedron/polyhedron_prog_tetra.cpp}

The polyhedron offers a point iterator for convenience. The above
\texttt{for} loop simplifies to a single statement by using
\ccc{std::copy} and the ostream iterator adaptor.

\begin{ccExampleCode}
std::copy( P.points_begin(), P.points_end(), 
           std::ostream_iterator<Point_3>(std::cout,"\n"));
\end{ccExampleCode}

% +-------------------------------------------------------------+
\subsection{Example for Affine Transformation}

An affine transformation $A$ can act as a functor transforming points
and a point iterator is conveniently defined for polyhedral surfaces.
So, assuming we want only the point coordinates of a polyhedron $P$
transformed, \ccc{std::transform} does the job in a single line.

\begin{ccExampleCode}
std::transform( P.points_begin(), P.points_end(), P.points_begin(), A);
\end{ccExampleCode}


% +-------------------------------------------------------------+
\subsection{Example Computing Plane Equations}

The polyhedral surface has already provisions to store a plane
equation for each facet. However, it does not provide a function to
compute plane equations.

This example computes the plane equations of a polyhedral surface.
The actual computation is implemented in the
\texttt{compute\_plane\_equations} function.  Depending on the arithmetic
(exact/inexact) and the shape of the facets (convex/non-convex)
different methods are useful. We assume here strictly convex facets
and exact arithmetic. In our example a homogeneous representation with
\texttt{int} coordinates is sufficient. The four plane equations of the
tetrahedron are the output of the program.

\ccIncludeExampleCode{Polyhedron/polyhedron_prog_planes.cpp}

% +-------------------------------------------------------------+
\subsection{Example with a Vector Instead of a List Representation\label{sectionPolyVector}}

The polyhedron class template has actually four parameters, where
three of them have default values. Using the default values explicitly
in our examples above for three parameter---ignoring the fourth
parameter, which would be a standard allocator for container class---
the definition of a polyhedron looks like:

\begin{ccExampleCode}
typedef CGAL::Polyhedron_3< Traits, 
                            CGAL::Polyhedron_items_3, 
                            CGAL::HalfedgeDS_default>      Polyhedron;
\end{ccExampleCode}

The \ccc{CGAL::Polyhedron_items_3} class contains the types used for
vertices, edges, and facets. The \ccc{CGAL::HalfedgeDS_default} class
defines the halfedge data structure used, which is a list-based
representation in this case. An alternative is a vector-based
representation. Using a vector provides random
access for the elements in the polyhedral surface and is more space
efficient, but elements cannot be deleted arbitrarily. Using a list
allows arbitrary deletions, but provides only bidirectional iterators
and is less space efficient. The following example creates again a 
tetrahedron with given points, but in a vector-based representation.

The vector-based representation resizes automatically if the reserved
capacity is not sufficient for the new items created. Upon resizing
all handles, iterators, and circulators become invalid. Their correct
update in the halfedge data structure is costly, thus it is advisable
to reserve enough space in advance as indicated with the alternative
constructor in the comment. 

\begin{ccAdvanced}
Note that the polyhedron and not the underlying halfedge data
structure triggers the resize operation, since the resize operation
requires some preconditions, such as valid incidences, to be fulfilled
that only the polyhedron can guarantee.
\end{ccAdvanced}

\ccIncludeExampleCode{Polyhedron/polyhedron_prog_vector.cpp}


% +-------------------------------------------------------------+
\subsection{Example with Circulator Writing Object File Format (OFF)}

We create a tetrahedron and write it to \ccc{std::cout} using the
Object File Format (OFF)~\cite{cgal:p-gmgv16-96}.  This example makes use
of \stl\ algorithms (\ccc{std::copy}, \ccc{std::distance}), \stl\
\ccc{std::ostream_iterator}, and \cgal\ circulators. The polyhedral
surface provides convenient circulators for the counterclockwise
circular sequence of halfedges around a facet and the clockwise
circular sequence of halfedges around a vertex.

However, the computation of the vertex index in the inner loop of the
facet output is not advisable with the \ccc{std::distance} function,
since it takes linear time for non random-access iterators, which
leads to quadratic runtime. For better runtime the vertex index needs
to be stored separately and computed once before writing the
facets. It can be stored, for example, in the vertex itself or in a
hash-structure.  See also the following Section~\ref{sectionPolyIO} for 
file I/O.

\ccIncludeExampleCode{Polyhedron/polyhedron_prog_off.cpp}

% +-------------------------------------------------------------+
\subsection{Example Using Euler Operators to Build a Cube}

Euler operators are the natural way of modifying polyhedral surfaces.
We provide a set of operations for polyhedra: \ccc{split_facet()}, 
\ccc{join_facet()}, \ccc{split_vertex()}, \ccc{join_vertex()},
\ccc{split_loop()}, and \ccc{join_loop()}. We add further convenient
operators, such as \ccc{split_edge()}. However, they could be implemented 
using the six operators above. Furthermore, we provide more operators
to work with polyhedral surfaces with border edges, for example, creating
and deleting holes. We refer to the references manual for the 
definition and illustrative figures of the Euler operators.

The following example implements a function that appends a unit cube
to a polyhedral surface. To keep track of the different steps during
the creation of the cube a sequence of sketches might help with labels
for the different handles that occur in the program code. The following
Figure shows six selected steps from the creation sequence. These steps 
are also marked in the program code.

\begin{ccTexOnly}
    \begin{center}
      \parbox{\textwidth}{%
          \includegraphics[width=\textwidth]{Polyhedron/fig/make_cube}%
      }
    \end{center}
\end{ccTexOnly}

\begin{ccHtmlOnly}
    <CENTER>
        <img src="./fig/make_cube.gif" alt="Steps in making a cube."><P>
    </CENTER>
\end{ccHtmlOnly}


\ccIncludeExampleCode{Polyhedron/polyhedron_prog_cube.cpp}



% +========================================================================+
\section{File I/O}
% +========================================================================+
\label{sectionPolyIO}

Simple file I/O for polyhedral surfaces is already provided in the
library. The file I/O considers so far only the topology of the
surface and its point coordinates. It ignores a possible plane
equation or any user-added attributes, such as color.

The default file format supported in \cgal\ for output as well as for
input is the Object File Format, OFF, with file extension {\tt .off},
which is also understood by Geomview~\cite{cgal:p-gmgv16-96}. For OFF
an ASCII and a binary format exist. The format can be selected with
the \cgal\ modifiers for streams, \ccc{set_ascii_mode} and
\ccc{set_binary_mode} respectively. The modifier \ccc{set_pretty_mode}
can be used to allow for (a few) structuring comments in the
output. Otherwise, the output would be free of comments.  The default
for writing is ASCII without comments. Both, ASCII and binary format,
can be read independent of the stream setting. Since this file format
is the default format, iostream operators are provided for it.

\ccThree{Inventor_ostream&M}{}{.}
\ccInclude{CGAL/IO/Polyhedron_iostream.h}

\ccHtmlNoLinks
\ccGlobalFunction{template <class PolyhedronTraits_3>
    ostream& operator<<( ostream& out, 
                         const CGAL::Polyhedron_3<PolyhedronTraits_3>& P);}

\ccHtmlNoLinks
\ccGlobalFunction{template <class PolyhedronTraits_3>
    istream& operator>>( istream& in, 
                         CGAL::Polyhedron_3<PolyhedronTraits_3>& P);}


Additional formats supported for writing are OpenInventor ({\tt .iv})
\cite{cgal:w-impoo-94}, VRML 1.0 and 2.0 ({\tt .wrl})
\cite{cgal:bpp-vrml-95,cgal:vrmls-97,cgal:hw-vrml2h-96}, and Wavefront Advanced
Visualizer object format ({\tt .obj}). Another convenient output
function writes a polyhedral surface to a Geomview process spawned
from the \cgal\ program.  These output functions are provided as
stream operators, now acting on the stream type of the respective
format.

\ccInclude{CGAL/IO/Polyhedron_inventor_ostream.h}
\\
\ccInclude{CGAL/IO/Polyhedron_VRML_1_ostream.h}
\\
\ccInclude{CGAL/IO/Polyhedron_VRML_2_ostream.h}
\\
\ccInclude{CGAL/IO/Polyhedron_geomview_ostream.h}

\ccHtmlNoLinks
\ccGlobalFunction{template <class PolyhedronTraits_3>
    Inventor_ostream& operator<<( Inventor_ostream& out, 
                           const CGAL::Polyhedron_3<PolyhedronTraits_3>& P);}

\ccHtmlNoLinks
\ccGlobalFunction{template <class PolyhedronTraits_3>
    VRML_1_ostream& operator<<( VRML_1_ostream& out, 
                           const CGAL::Polyhedron_3<PolyhedronTraits_3>& P);}

\ccHtmlNoLinks
\ccGlobalFunction{template <class PolyhedronTraits_3>
    VRML_2_ostream& operator<<( VRML_2_ostream& out, 
                           const CGAL::Polyhedron_3<PolyhedronTraits_3>& P);}

\ccHtmlNoLinks
\ccGlobalFunction{template <class PolyhedronTraits_3>
    Geomview_stream& operator<<( Geomview_stream& out, 
                           const CGAL::Polyhedron_3<PolyhedronTraits_3>& P);}


All these file formats have in common that they represent a surface as
a set of facets. Each facet is a list of indices pointing into a set
of vertices. Vertices are represented as coordinate triples. The
file I/O for polyhedral surfaces \ccc{CGAL::Polyhedron_3} imposes certain 
restrictions on these formats. They must represent a permissible 
polyhedral surface, e.g., a 2-manifold and no isolated vertices, see 
Section~\ref{sectionPolyIntro}.

Some example programs around the different file formats are provided
in the distribution under \texttt{examples/Polyhedron\_IO/} and
\texttt{demo/Polyhedron\_IO/}. We show an example converting OFF input
into VRML 1.0 output.

\begin{ccExampleCode}
// examples/Polyhedron_IO/polyhedron2vrml.cpp
// ----------------------------------------

#include <CGAL/Simple_cartesian.h>
#include <CGAL/Polyhedron_3.h>
#include <CGAL/IO/Polyhedron_iostream.h>
#include <CGAL/IO/Polyhedron_VRML_1_ostream.h> 
#include <iostream>

typedef CGAL::Simple_cartesian<double> Kernel;
typedef CGAL::Polyhedron_3<Kernel>     Polyhedron;

int main() {
    Polyhedron P;
    std::cin >> P;
    CGAL::VRML_1_ostream out( std::cout);
    out << P;
    return ( std::cin && std::cout) ? 0 : 1;
}
\end{ccExampleCode}



% +========================================================================+
\section{Extending Vertices, Halfedges, and Facets}
% +========================================================================+
\label{sectionPolyExtend}

In Section~\ref{sectionPolyVector} we have seen how to change the 
default list representation

\begin{ccExampleCode}
typedef CGAL::Polyhedron_3< Traits, 
                            CGAL::Polyhedron_items_3, 
                            CGAL::HalfedgeDS_default>      Polyhedron;
\end{ccExampleCode}

to a vector based representation of the underlying halfedge data
structure. Now we want to look a bit closer at the second template argument,
\texttt{Polyhedron\_items\_3}, that specifies what kind of vertex, 
halfedge, and facet is used. The implementation of 
\texttt{Polyhedron\_items\_3} looks a bit involved with nested 
wrapper class templates. But ignoring this technicality, what remains
are three local typedefs that define the \texttt{Vertex}, the
\texttt{Halfedge}, and the \texttt{Face} for the polyhedral surface.
Note that we use here \texttt{Face} instead of facet. Face is the term
used for the halfedge data structure. Only the top layer of the
polyhedral surface gives alias names renaming face to facet.

\begin{ccExampleCode}
class Polyhedron_items_3 {
public:
    template < class Refs, class Traits>
    struct Vertex_wrapper {
        typedef typename Traits::Point_3 Point;
        typedef CGAL::HalfedgeDS_vertex_base<Refs, CGAL::Tag_true, Point> Vertex;
    };
    template < class Refs, class Traits>
    struct Halfedge_wrapper {
        typedef CGAL::HalfedgeDS_halfedge_base<Refs>                      Halfedge;
    };
    template < class Refs, class Traits>
    struct Face_wrapper {
        typedef typename Traits::Plane_3 Plane;
        typedef CGAL::HalfedgeDS_face_base<Refs, CGAL::Tag_true, Plane>   Face;
    };
};
\end{ccExampleCode}

If we look up in the reference manual the definitions of the three
classes used in the typedefs, we will see the confirmation that the
default polyhedron uses all supported incidences, a point in the
vertex class, and a plane equation in the face class. Note how the
wrapper class provides two template parameters, \texttt{Refs}, which
we discuss a bit later, and \texttt{Traits}, which is the geometric
traits class used by the polyhedral surface and which provides us here
with the types for the point and the plane equation.

Using this example code we can write our own items class. Instead, we
illustrate an easier way if we only want to exchange one class. We use
a simpler face without the plane equation but with a color attribute
added. To simplify the creation of a vertex, halfedge, or face class,
it is always recommended to derive from one of the given base classes.
Even if the base class would contain no data it would provide
convenient type definitions. So, we derive from the base class, repeat
the mandatory constructors if necessary---which is not the case for
faces but would be for vertices---and add the color attribute.

\begin{ccExampleCode}
template <class Refs>
struct My_face : public CGAL::HalfedgeDS_face_base<Refs> {
    CGAL::Color color;
};
\end{ccExampleCode}

The new items class is derived from the old items class and the
wrapper containing the face typedef gets overridden. Note that the
name of the wrapper and its template parameters are fixed. They cannot
be changed even if, as in this example, a template parameter is not
used.

\begin{ccExampleCode}
struct My_items : public CGAL::Polyhedron_items_3 {
    template <class Refs, class Traits>
    struct Face_wrapper {
        typedef My_face<Refs> Face;
    };
};
\end{ccExampleCode}

When we use our new items class with the polyhedral surface, our new
face class is used in the halfedge data structure and the color
attribute is available in the type \texttt{Polyhedron::Facet}. However,
\texttt{Polyhedron::Facet} is not the same type as our local face 
typedef for \texttt{My\_face}, but it is derived therefrom. Thus,
everything that we put in the local face type except constructors is
then available in the \texttt{Polyhedron::Facet} type. For more
details, see the Chapter~\ref{chapterHalfedgeDS} on the halfedge data
structure design.

Pulling all pieces together, the full example program illustrates how easy
the color attribute can be accessed once it is defined.

\ccIncludeExampleCode{Polyhedron/polyhedron_prog_color.cpp}

We come back to the first template parameter, \texttt{Refs}, of the
wrapper classes. This parameter provides us with local types that
allow us to make further references between vertices, halfedges, and
facets, which have not already been prepared for in the current
design. These local types are \texttt{Vertex\_handle},
\texttt{Halfedge\_handle}, \texttt{Face\_handle}, and there respective
\texttt{\ldots\_const\_handle}. We add now a new vertex reference to a
face class as follows. Encapsulation and access functions could be
added for a more thorough design, but we omit that here for the sake
of brevity. The integration of the face class with the items class
works as illustrated above.

\begin{ccExampleCode}
template <class Refs>
struct My_face : public CGAL::HalfedgeDS_face_base<Refs> {
    typedef typename Refs::Vertex_handle Vertex_handle;
    Vertex_handle vertex_ref;
};
\end{ccExampleCode}

More advanced examples can be found in the Section~\ref{sectionHdsExamples}
illustrating further the design of the halfedge data structure.


% +========================================================================+
\section{Advanced Example Programs}
% +========================================================================+
\label{sectionPolyAdvanced}

% +------------------------------------------------------------------------+
\subsection{Example Creating a Subdivision Surface}

This program reads a polyhedral surface from the standard input and
writes a refined polyhedral surface to the standard output. Input and
output are in the Object File Format, OFF, with the common file
extension {\tt .off}, which is also understood by
Geomview~\cite{cgal:p-gmgv16-96}.

The refinement is a single step of the $\sqrt{3}$-scheme for creating
a subdivision surface~\cite{cgal:k-s-00}. Each step subdivides a facet
into triangles around a new center vertex, smoothes the position of the
old vertices, and flips the old edges. The program is organized along
this outline. In each of these parts, the program efficiently uses the
knowledge that the newly created vertices, edges, and facets have been
added to the end of the sequences. The program needs additional
processing memory only for the smoothing step of the old vertices.

\begin{ccTexOnly}
    \begin{center}
      \parbox{\textwidth}{%
          \includegraphics[width=\textwidth]{Polyhedron/fig/subdiv}%
      }
    \end{center}
\end{ccTexOnly}

\begin{ccHtmlOnly}
    <CENTER>
        <A HREF="./fig/subdiv.gif">
            <img src="./fig/subdiv_small.gif" alt="subdivision examples">
        </A><P>
    </CENTER>
\end{ccHtmlOnly}

The above figure shows three example objects, each 
subdivided four times. The initial object for the left sequence is
the closed surface of three unit cubes glued together to a corner.
The example program shown here can handle only closed surfaces, 
but the extended example
\texttt{examples/Polyhedron/polyhedron\_prog\_subdiv\_with\_boundary.cpp}
handles surfaces with boundary. So, the middle sequence starts with
the same surface where one of the facets has been removed. The boundary
subdivides to a nice circle. The third sequence creates a sharp
edge using a trick in the object presentation. The sharp edge is 
actually a hole whose vertex coordinates pinch the hole shut to form an
edge. The example directory \texttt{examples/Polyhedron/} contains the 
OFF files used here.

\ccIncludeExampleCode{Polyhedron/polyhedron_prog_subdiv.cpp}

% +------------------------------------------------------------------------+
\subsection{Example Using the Incremental Builder and Modifier Mechanism}

A utility class \ccc{CGAL::Polyhedron_incremental_builder_3} helps in
creating polyhedral surfaces from a list of points followed by a list
of facets that are represented as indices into the point list. This is
particularly useful for implementing file reader for common file
formats.  It is used here to create a triangle.

A modifier mechanism allows to access the internal representation of
the polyhedral surface, i.e., the halfedge data structure, in a
controlled manner. A modifier is basically a callback mechanism using
a function object. When called, the function object receives the
internal halfedge data structure as a parameter and can modify it.  On
return, the polyhedron can check the halfedge data structure for
validity. Such a modifier object must always return with a halfedge
data structure that is a valid polyhedral surface. The validity check is
implemented as an expensive postcondition at the end of the \ccc{delegate()}
member function, i.e., it is not called by default, only when expensive
checks are activated.

In this example, \ccc{Build_triangle} is such a function object
derived from \ccc{CGAL::Modifier_base<HalfedgeDS>}. The \ccc{delegate()}
member function of the polyhedron accepts this function object and calls
its \ccc{operator()} with a reference to its internally used halfedge 
data structure. Thus, this member function in \ccc{Build_triangle} can 
create the triangle in the halfedge data structure.

\ccIncludeExampleCode{Polyhedron/polyhedron_prog_incr_builder.cpp}

% +--------------------------------------------------------+

%% %% Copyright (c) 2005  Foundation for Research and Technology-Hellas (Greece).
%% All rights reserved.
%%
%% This file is part of CGAL (www.cgal.org).
%% You can redistribute it and/or modify it under the terms of the GNU
%% General Public License as published by the Free Software Foundation,
%% either version 3 of the License, or (at your option) any later version.
%%
%% Licensees holding a valid commercial license may use this file in
%% accordance with the commercial license agreement provided with the software.
%%
%% This file is provided AS IS with NO WARRANTY OF ANY KIND, INCLUDING THE
%% WARRANTY OF DESIGN, MERCHANTABILITY AND FITNESS FOR A PARTICULAR PURPOSE.
%%
%% $URL$
%% $Id$
%% 
%%
%% Author(s)     : Menelaos Karavelas <mkaravel@iacm.forth.gr>

\ccUserChapter{2D Voronoi Diagram Adaptor\label{chapter-vda}}
\ccChapterAuthor{Menelaos Karavelas}

\minitoc

\begin{ccPkgDescription}{3D Skin Surface Meshing \label{Pkg:SkinSurface3}}
\ccPkgHowToCiteCgal{cgal:k-ssm3-12}
  \ccPkgSummary{ %
    This package allows to build a triangular mesh of a skin surface.
    Skin surfaces are used for modeling large molecules in biological
    computing. The surface is defined by a set of balls, representing
    the atoms of the molecule, and a shrink factor that determines the
    size of the smooth patches gluing the balls together.

    The construction of a triangular mesh of a smooth skin surface is
    often necessary for further analysis and for fast visualization.
    This package provides functions to construct a triangular mesh
    approximating the skin surface from a set of balls and a shrink
    factor. It also contains code to subdivide the mesh efficiently.
    % 
  }
%
  \ccPkgIntroducedInCGAL{3.3}
  \ccPkgDependsOn{\ccRef[3D Triangulation]{Pkg:Triangulation3} and \ccRef[3D Polyhedral Surface]{Pkg:Polyhedron}}
  \ccPkgLicense{\ccLicenseGPL}
  \ccPkgIllustration{Skin_surface_3/small.png}{Skin_surface_3/large.png}
\end{ccPkgDescription}

%% Copyright (c) 2005  Foundation for Research and Technology-Hellas (Greece).
%% All rights reserved.
%%
%% This file is part of CGAL (www.cgal.org).
%% You can redistribute it and/or modify it under the terms of the GNU
%% General Public License as published by the Free Software Foundation,
%% either version 3 of the License, or (at your option) any later version.
%%
%% Licensees holding a valid commercial license may use this file in
%% accordance with the commercial license agreement provided with the software.
%%
%% This file is provided AS IS with NO WARRANTY OF ANY KIND, INCLUDING THE
%% WARRANTY OF DESIGN, MERCHANTABILITY AND FITNESS FOR A PARTICULAR PURPOSE.
%%
%% $URL$
%% $Id$
%% 
%%
%% Author(s)     : Menelaos Karavelas <mkaravel@iacm.forth.gr>

This chapter describes an adaptor that adapts two-dimensional
triangulated Delaunay graphs to the corresponding Voronoi diagrams.
We start with a few
definitions and a description of the issues that this adaptor
addresses in Section~\ref{sec:vda2-intro}. The software design
of the Voronoi diagram adaptor package is described in
Section~\ref{sec:vda2-design}. In Section~\ref{sec:vda2-traits} we
discuss the traits required for performing the adaptation, and finally
in Section~\ref{sec:vda2-examples} we present a few examples using
this adaptor.

\section{Introduction\label{sec:vda2-intro}}

A Voronoi diagram is typically defined for a set of objects, also
called sites in the sequel, that lie in some space $\Sigma$ and a
distance function that measures the distance of a point $x$ in
$\Sigma$ from an object in the object set. In this package we are
interested in planar Voronoi diagrams, so in the sequel the space
$\Sigma$ will be the space $\mathbb{R}^2$. 
Let $\mathcal{S}=\{S_1,S_2,\ldots,S_n\}$ be our set of sites and let
$\delta(x,S_i)$ denote the distance of a point $x\in\mathbb{R}^2$ from
the site $S_i$. Given two sites $S_i$ and $S_j$, the set $V_{ij}$
of points that are closer to $S_i$ than to $S_j$ with respect to the
distance function $\delta(x,\cdot)$ is simply the set:
\[   V_{ij} = \{x\in\mathbb{R}^2:\, \delta(x,S_i)<\delta(x,S_j)\}. \]
We can then define the set $V_i$ of points on the plane that are closer to
$S_i$ than to any other object in $\mathcal{S}$ as:
\[  V_i = \bigcap_{i\neq j} V_{ij}. \]
The set $V_i$ is said to be the \emph{Voronoi cell} or \emph{Voronoi face} 
of the site $S_i$. The locus of points on the plane that are
equidistant from exactly two sites $S_i$ and $S_j$ is called a
\emph{Voronoi bisector}. A point that is equidistant to three or
more objects in $\mathcal{S}$ is called a \emph{Voronoi vertex}.
A simply connected subset of a Voronoi bisector is called a
\emph{Voronoi edge}.
The collection of Voronoi faces, edges and vertices is called the
\emph{Voronoi diagram} of the set $\mathcal{S}$ with respect to the
distance function $\delta(x,\cdot)$, and it turns out that it is a
subdivision of the plane, i.e., it is a planar graph.

We typically think of faces as 2-dimensional objects, edges as
1-dimensional objects and vertices as 0-dimensional objects. However,
this may not be the case for several combinations of sites and
distance functions (for example points in $\mathbb{R}^2$ under the
$L_1$ or the $L_\infty$ distance can produce 2-dimensional Voronoi
edges). We call a Voronoi diagram \emph{nice} if no such artifacts
exist, i.e., if all vertices edges and faces are 0-, 1- and
2-dimensional, respectively.

Even nice Voronoi diagrams can end up being not so nice. The cell of a
site can in general consist of several disconnected components. Such a
case can happen, for example, when we consider weighted points
$Q_i=(p_i,\lambda_i)$, where $p_i\in\mathbb{R}^2$,
$\lambda_i\in\mathbb{R}$, and the distance function is 
the Euclidean distance multiplied by the weight of each site, i.e.,
$\delta_M(x,Q_i)=\lambda_i\,\|x-p_i\|$, where $\|\cdot\|$ denotes the
Euclidean norm. In this package we are going to restrict ourselves to
nice Voronoi diagrams that have the property that the Voronoi cell of
each site is a simply connected region of the plane. We are going to
call such Voronoi diagrams \emph{simple Voronoi diagrams}. Examples of
simple Voronoi diagrams include the usual Euclidean Voronoi diagram of
points, the Euclidean Voronoi diagram of a set of disks on the plane
(i.e., the Apollonius diagram), the Euclidean Voronoi diagram of a set
of disjoint convex objects on the plane, or the power or (Laguerre)
diagram for a set of circles on the plane. In fact every instance of
an \emph{abstract Voronoi diagram} in the sense of Klein \cite{k-cavd-89} 
is a simple Voronoi diagram in our setting. In the sequel when we
refer to Voronoi diagrams we will refer to simple Voronoi diagrams.

In many cases we are not really interested in computing the
Voronoi diagram itself, but rather its dual graph, called the
\emph{Delaunay graph}. In general the Delaunay graph is a planar
graph, each face of which consists of at least three edges.
Under the non-degeneracy assumption that no point on the plane is
equidistant, under the distance function, to more than three sites, 
the Delaunay graph is a planar graph with triangular faces.
In certain cases this graph can actually be embedded with straight
line segments in which case we talk about a triangulation. This is the
case, for example, for the Euclidean Voronoi diagram of points, or the
power diagram of a set of circles. The dual graphs are, respectively,
the Delaunay triangulation and the regular triangulation of the
corresponding site sets. Graphs of non-constant non-uniform face
complexity can be undesirable in many applications, so typically we
end up triangulating the non-triangular faces of the Delaunay
graph. Intuitively this amounts to imposing an implicit or explicit
perturbation scheme during the construction of the Delaunay graph,
that perturbs the input sites in such a way so as not to have
degenerate configurations.

Choosing between computing the Voronoi diagram or the (triangulated)
Delaunay graph is a major decision while implementing an algorithm. It
heavily affects the design and choice of the different data structures
involved. Although in theory the two approaches are entirely
equivalent, it is not so straightforward to go from one representation
to the other. The objective of this package is to provide a generic
way of going from triangulated Delaunay graphs to planar
subdivisions represented through a DCEL data structure. The goal is to
provide an adaptor that gives the look and feel of a DCEL data structure,
although internally it keeps a graph data structure representing
triangular graphs.

The adaptation might seem straightforward at a first glance, and more
or less this is case; after all one graph is the dual of the
other. The situation becomes complicated whenever we want to treat
artifacts of the representation used. Suppose for example that we have
a set of sites that contains subsets of sites in degenerate
positions. The computed triangulated Delaunay graph has triangular
faces that may be the result of an implicit or explicit perturbation
scheme. The dual of such a triangulated Delaunay graph is a Voronoi
diagram that has all its vertices of degree 3, and for that purpose we
are going to call it a \emph{degree-3 Voronoi diagram} in order to
distinguish it from the true Voronoi diagram of the input sites. A
degree-3 Voronoi diagram can have degenerate features, namely Voronoi
edges of zero length, and/or Voronoi faces of zero area. Although we
can potentially treat such artifacts, they are nonetheless artifacts of
the algorithm we used and do not correspond to the true geometry of
the Voronoi diagram.

The manner that we treat such issues in this package in a generic way
is by defining an \emph{adaptation policy}. The adaptation policy is
responsible for determining which features in the degree-3 Voronoi
diagram are to be rejected and which not. The policy to be used can
vary depending on the application or the intended usage of the
resulting Voronoi diagram. What we care about is that firstly the
policy itself is consistent and, secondly, that the adaptation is also
done in a consistent manner. The latter is the responsibility of the
adaptor provided by this package, whereas the former is the
responsibility of the implementor of a policy.

In this package we currently provide two types of adaptation
policies. The first one is the simplest: we reject no feature of the
degree-3 Voronoi diagram; we call such a policy an
\emph{identity policy} since the Voronoi diagram produced is identical
to the degree-3 Voronoi diagram. The second type of policy eliminates
the degenerate features from the degree-3 Voronoi diagram yielding
the true geometry of the Voronoi diagram of the input sites; we call
such policies \emph{degeneracy removal policies}.

Delaunay graphs can be mutable or non-mutable. By mutable we mean that
sites can be inserted or removed at any time, in an entirely on-line
fashion. By non-mutable we mean that once the Delaunay graph has been
created, no changes, with respect to the set of sites defining it,
are allowed. If the Delaunay graph is a non-mutable one, then the
Voronoi diagram adaptor is a non-mutable adaptor as well.

If the Delaunay graph is mutable then the question of whether the
Voronoi diagram adaptor is also mutable is slightly more complex to
answer. As long as the adaptation policy used does not maintain a
state, the Voronoi diagram adaptor is a mutable one; this is the case,
for example, with our identity policy or the degeneracy removal
policies. If, however, the adaptor maintains a state, then whether it
is mutable or non-mutable really depends on whether its state can be
updated after every change in the Delaunay graph. Such policies are
our caching degeneracy removal policies: some of them result in
mutable adaptors others result in non-mutable ones. In
Section~\ref{sec:vda2-ap} we discuss the issue in more detail.




\section{Software Design\label{sec:vda2-design}}

The \ccc{Voronoi_diagram_2<DG,AT,AP>} class is parameterized by
three template parameters. The first one must be a model of the
\ccc{DelaunayGraph_2} concept. It corresponds to the API required by
an object representing a Delaunay graph. All classes of \cgal{} that
represent Delaunay diagrams are models of this concept, namely,
Delaunay triangulations, regular triangulations, Apollonius 
graphs and segment Delaunay graphs.
%
The second template parameter must be a model of the
\ccc{AdaptationTraits_2} concept. We discuss this concept in detail in
Section~\ref{sec:vda2-traits}.
%
The third template parameter must be model of the
\ccc{AdaptationPolicy_2} concept, which we discuss in detail in
Section~\ref{sec:vda2-ap}.

The \ccc{Voronoi_diagram_2<DG,AT,AP>} class has been
intentionally designed to provide an API similar to the arrangements
class in \cgal: Voronoi diagrams are special cases of arrangements
after all. The API of the two classes, however, could not be
identical. The reason is that arrangements in \cgal{} do not yet support
more than one unbounded faces, or equivalently, cannot handle
unbounded curves. On the contrary, a Voronoi diagram defined over at
least two generating sites, has at least two unbounded faces.

On a more technical level, the \ccc{Voronoi_diagram_2<DG,AT,AP>}
class imitates the representation of the Voronoi diagram (seen as a
planar subdivision) by a DCEL (Doubly Connected Edge List) data
structure. We have vertices (the Voronoi vertices), halfedges
(oriented versions of the Voronoi edges) and faces (the Voronoi
cells). In particular, we can basically perform every operation we can
perform in a standard DCEL data structure:
\begin{itemize}
\item go from a halfedge to its next and previous in the face;
\item go from one face to an adjacent one through a halfedge and its
  twin (opposite) halfedge;
\item walk around the boundary of a face;
\item enumerate/traverse the halfedges incident to a vertex
\item from a halfedge, access the adjacent face;
\item from a face, access an adjacent halfedges;
\item from a halfedges, access its source and target vertices;
\item from a vertex, access an incident halfedge.
\end{itemize}
In addition to the above possibilities for traversal, we can also
traverse the following features through iterators:
\begin{itemize}
\item the vertices of the Voronoi diagram;
\item the edges or halfedges of the Voronoi diagram;
\item the faces of the Voronoi diagram;
\item the bounded faces of the Voronoi diagram;
\item the bounded halfedges of the Voronoi diagram;
\item the unbounded faces of the Voronoi diagram;
\item the unbounded halfedges of the Voronoi diagram;
\item the sites defining the Voronoi diagram.
\end{itemize}

%Finally, depending on the adaptation traits passed to the Voronoi diagram
%adaptor, we can perform certain types of queries, namely:
%\begin{itemize}
%\item given a point $p$ we can determine the feature of the Voronoi
%  diagram (vertex, edge, face) on which $p$ lies, and
%\item given a point $p$ determine all nearest sites of $p$ in the
%  Voronoi diagram, i.e., determine all sites $S_i$ in the Voronoi
%  diagram such that the distance $\delta(S_i,p)$ is minimal.
%\end{itemize}
%% SINCE I DON'T WANT TO SUPPORT THE 2ND QUERY YET, THE TEXT IS FOR
%% NOW THE FOLLOWING:
Finally, depending on the adaptation traits passed to the Voronoi diagram
adaptor, we can perform point location queries, namely given a point
$p$ we can determine the feature of the Voronoi diagram (vertex, edge,
face) on which $p$ lies.

%On a more practical basis, arrangements in \cgal\ require as input a
%set of curves. In the case of Voronoi diagrams these curves lie on
%bisectors of generating sites, and can be bounded arcs, semi-bounded
%arcs or the bisectors themselves. Constructing the bisectors can be
%a complicated task, yielding a complicated outcome, which becomes even
%more complicated when we want to compute their endpoints, which are
%the vertices in the Voronoi diagram. Consider, for example, the
%Voronoi diagram of segments: if the segments' coordinates are represented
%by ring number type, computing the Voronoi bisectors and Voronoi
%vertices requires a field number type that also supports square roots
%on top of the field operations.

%On the other hand, we can fully determine the combinatorial structure
%of the Voronoi diagram using the dual Delaunay graph, modulo of course
%the degenerate features we discussed above. In other words, in order
%to compute the combinatorial representation of the Voronoi diagram we
%do not really need to know the bisecting curves or arcs, or the
%Voronoi vertices. Furthermore, not only can we determine the
%combinatorial structure without computing the Voronoi features, but
%can also determine whether a feature is degenerate or not without
%computing its geometry. More specifically, determining whether the dual
%Voronoi edge of an edge in the Delaunay graph is degenerate (of zero
%length) or not can be done without really computing the curve itself.
%For example, in the case of the Delaunay triangulation, determining
%whether the dual Delaunay edge has zero length amounts to performing
%an incircle test.


\section{The Adaptation Traits\label{sec:vda2-traits}}

The \ccc{AdaptationTraits_2} concept defines the types and
functor required by the adaptor in order to access geometric
information in the Delaunay graph that is needed by the
\ccc{Voronoi_diagram_2<DG,AT,AP>} class.
In particular, it provides functors for accessing sites in the Delaunay
graph and constructing Voronoi vertices from their dual faces in the
Delaunay graph.
Finally, it defines a tag that indicates whether nearest site queries
are to be supported by the Voronoi diagram adaptor. If such queries
are to be supported, a functor is required.

Given a query point, the nearest site functor should return information 
related to how many and which sites of the Voronoi diagram are at
equal and minimal distance from the query point. In particular, if the
query point is closest to a single site, the vertex handle of the
Delaunay graph corresponding to this site is returned. If the
query point is closest to exactly two site, the edge of the
Delaunay graph that is dual to the Voronoi edges on which the query
point lies is returned. If three (or more) sites are closest to
the query point, then the query point coincides with a vertex in the
Voronoi diagram, and the face handle of the face in the Delaunay graph
that is dual to the Voronoi vertex is returned.
This way of abstracting the point location mechanism allows
for multiple different point location strategies, which are passed to
the Voronoi diagram adaptor through different models of the
\ccc{AdaptationTraits_2} concept. The point location and nearest sites
queries of the \ccc{Voronoi_diagram_2<DG,AT,AP>} class use internally
this nearest site query functor.

In this package we provide four adaptation traits classes, all of which
support nearest site queries:
\begin{itemize}
\item
  The \ccc{Apollonius_graph_adaptation_traits_2<AG2>} class: it
  provides the adaptation traits for Apollonius graphs.
\item
  The \ccc{Delaunay_triangulation_adaptation_traits_2<DT2>} class: it
  provides the adaptation traits for Delaunay triangulations.
\item
  The \ccc{Regular_triangulation_adaptation_traits_2<RT2>} class: it
  provides the adaptation traits for regular triangulations.
\item
  The \ccc{Segment_Delaunay_graph_adaptation_traits_2<SDG2>} class: it
  provides the adaptation traits for segment Delaunay graphs.
\end{itemize}

\section{The Adaptation Policy\label{sec:vda2-ap}}

As mentioned above, when we perform the adaptation of a triangulated
Delaunay graph to a Voronoi diagram, a question that arises is whether
we want to eliminate certain features of the Delaunay graph when we
construct its Voronoi diagram representation (such features could be
the Voronoi edges of zero length or, for the Voronoi diagram of a set
of segments forming a polygon, all edges outside the polygon).
The manner that we treat such issues in this package in a generic way
is by defining an adaptation policy. The adaptation policy is
responsible for determining which features in the degree-3 Voronoi
diagram are to be rejected and which not. The policy to be used can
vary depending on the application or the intended usage of the
resulting Voronoi diagram.

The concept \ccc{AdaptationPolicy_2} defines the requirements on
the predicate functors that determine whether a feature of the
triangulated Delaunay graph should be rejected or not. More
specifically it defines an \ccc{Edge_rejector} and a
\ccc{Face_rejector} functor that answer the question: ``should this
edge (face) of the Voronoi diagram be rejected?''. In addition to the
edge and face rejectors the adaptation policy defines a tag, the
\ccc{Has_inserter} tag. This tag is either set to \ccc{CGAL::Tag_true}
or to \ccc{CGAL::Tag_false}. Semantically it determines if the adaptor
is allowed to insert sites in an on-line fashion (on-line removals are
not yet supported). In the former case, i.e., when on-line site
insertions are allowed, an additional functor is required, the
\ccc{Site_inserter} functor. This functor takes a reference to a
Delaunay graph and a site, and inserts the site in the Delaunay
graph. Upon successful insertion, a handle to the vertex representing
the site in the Delaunay graph is returned.

We have implemented two types of policies that provide two different
ways for answering the question of which features of the Voronoi
diagram to keep and which to discard. The first one is called the
\emph{identity policy} and corresponds to the
\ccc{Identity_policy_2<DG,VT>} class. This policy is in some sense the
simplest possible one, since it does not reject any feature of the
Delaunay graph. The Voronoi diagram provided by the adaptor is the
true dual (from the graph-theoretical point of view) of the
triangulated Delaunay graph adapted. This policy assumes that the
Delaunay graph adapted allows for on-line insertions, and the
\ccc{Has_inserter} tag is set to \ccc{CGAL::Tag_true}. A default site
inserter functor is also provided.

The second type of policy we provide is called
\emph{degeneracy removal policy}. If the set of sites defining the
triangulated Delaunay graph contains subsets of sites in degenerate
configurations, the graph-theoretical dual of the triangulated
Delaunay graph has edges and potentially faces that are geometrically
degenerate. By that we mean that the dual of the triangulated Delaunay
graph can have Voronoi edges of zero length or Voronoi faces/cells of
zero area. Such features may not be desirable and ideally we would
like to eliminate them. The degeneracy removal policies eliminate
exactly these features and provide a Voronoi diagram where all edges
have non-zero length and all cells have non-zero area. More
specifically, in these policies the \ccc{Edge_rejector} and
\ccc{Face_rejector} functors reject the edges and vertices of the
Delaunay graph that correspond to dual edges and faces that have zero
length and area, respectively. In this package we provide four
degeneracy removal policies, namely:
\begin{itemize}
\item
  The \ccc{Apollonius_graph_degeneracy_removal_policy_2<AG2>} class: it
  provides an adaptation policy for removing degeneracies when
  adapting an Apollonius graph to an Apollonius diagram.
\item
  The \ccc{Delaunay_triangulation_degeneracy_removal_policy_2<DT2>} class: it
  provides an adaptation policy for removing degeneracies when
  adapting a Delaunay triangulation to a point Voronoi diagram.
\item
  The \ccc{Regular_triangulation_degeneracy_removal_policy_2<RT2>} class: it
  provides an adaptation policy for removing degeneracies when
  adapting a regular triangulation to a power diagram 

\item
  The \ccc{Segment_Delaunay_graph_degeneracy_removal_policy_2<SDG2>} class: it
  provides an adaptation policy for removing degeneracies when
  adapting a segment Delaunay graph to a segment Voronoi diagram.
\end{itemize}

A variation of the degeneracy removal policies are the
\emph{caching degeneracy removal policies}. In these policies we cache
the results of the edge and face rejectors. In particular, every time
we want to determine, for example, if an edge of the Delaunay graph
has, as dual edge in the Voronoi diagram, an edge of zero length, we
check if the result has already been computed. If yes, we simply
return the outcome. If not, we perform the necessary geometric tests,
compute the answer, cache it and return it. Such a policy really pays
off when we have a lot of degenerate data in our input set of
sites. Verifying whether a Voronoi edge is degenerate or not implies
computing the outcome of a predicate in a possibly degenerate or near
degenerate configuration, which is typically very costly (compared to
computing the same predicate in a generic configuration). To avoid this cost
every single time we want to check if a Voronoi edge is degenerate or
not, we compute the result of the geometric predicate the first time
the adaptor asks for it, and simply lookup the answer in the future.
In this package we provide four caching degeneracy removal policies,
one per degeneracy removal policy mentioned above.
Intentionally, we have not indicated the value of the
\ccc{Has_inserter} tag for the degeneracy removal and caching
degeneracy removal policies. The issue is discussed in detail in the
sequel.

We raised the question above, as to whether the adaptor is a mutable
or non-mutable one, in the sense of whether we can add/remove sites in
an on-line fashion. The answer to this question depends on: (1) whether the
Delaunay graph adapted allows for on-line insertions/removals and (2)
whether the adaptation policies maintains a state and whether this
state is easily maintainable when we want to allow for on-line
modifications.

The way we indicate if we allow on-line insertions of sites is via the
\ccc{Has_inserter} tag (as mentioned, on-line removals are currently not
supported). The \ccc{Has_inserter} tag has two possible values,
namely, \ccc{CGAL::Tag_true} and \ccc{CGAL::Tag_false}. The value
\ccc{CGAL::Tag_true} indicates that the Delaunay graph allows for
on-line insertions, whereas the value \ccc{CGAL::Tag_false} indicates
the opposite. Note that these values \emph{do not} indicate if the
Delaunay graph supports on-line insertions, but rather whether the
Voronoi diagram adaptor should be able to perform on-line insertions
or not. This delicate point will be become clearer below.

Let us consider the various scenarios. If the Delaunay graph is
non-mutable, the Voronoi diagram adaptor cannot perform on-line
insertions of sites. In this case not only degeneracy removal
policies, but rather every single adaptation policy for
adapting the Delaunay graph in question should have the
\ccc{Has_inserter} tag set to \ccc{CGAL::Tag_false}.

If the Delaunay graph is mutable, i.e., on-line site insertions as are
allowed, we can choose between two types of adaptation policies, those
that allow these on-line insertions and those that do not. In the
former case the \ccc{Has_inserter} tag should be set to
\ccc{CGAL::Tag_true}, whereas in the latter to
\ccc{CGAL::Tag_false}. In other words, even if the Delaunay graph is
mutable, we can choose (by properly determining the value of the
\ccc{Has_inserter} tag) if the adaptor should be mutable as well. At a
first glance it may seem excessive to restrict existing
functionality. There are situations, however, where such a choice is
necessary.

Consider a caching degeneracy removal policy. If we do not allow for
on-line insertions then the cached quantities are always valid since
the Voronoi diagram never changes. If we allow for on-line insertions
the Voronoi diagram can change, which implies that the results of the edge
and faces degeneracy testers that we have cached are no longer valid
or relevant. In these cases, we need to somehow update these cached
results, and ideally we would like to do this in an efficient manner.
%
The inherent dilemma in the above discussion is whether the Voronoi
diagram adaptor should be able to perform on-line insertions of
sites. The answer to this question in this framework is given by the
\ccc{Has_inserter} tag. If the tag is set to \ccc{CGAL::Tag_false} the
adaptor cannot insert sites on-line, whereas if the tag is set to
\ccc{CGAL::Tag_true} the adaptor can add sites on-line. In other
words, the \ccc{Has_inserter} tag determines how the Voronoi diagram
adaptor should behave, and this is enough from the adaptor's point of
view.

From the point of a view of a policy writer the dilemma is still
there: should the policy allow for on-line insertions or not? The
answer really depends on what are the consequences of such a
choice. For a policy that has no state, such as our degeneracy removal
policies, it is natural to set the \ccc{Has_inserter} tag to
\ccc{CGAL::Tag_true}. For our caching degeneracy removal policies, our
choice was made on the grounds of whether we can update the cached
results efficiently when insertions are performed. For \cgal's
Apollonius graphs, Delaunay triangulation and regular triangulations
it is possible to ask what are the edges and faces of the Delaunay
graph that are to be destroyed when a query site is inserted. This is
done via the \ccc{get_conflicts} method provided by these
classes. Using the outcome of the \ccc{get_conflicts} method the site
inserter can first update the cached results (i.e., indicate which are
invalidated) and then perform the actual insertion. Such a method does
not yet exist for segment Delaunay graphs. We have thus chosen to
support on-line insertions for all non-caching degeneracy removal
policies. The caching degeneracy removal policy for segment Delaunay
graphs does not support on-line insertions, whereas the remaining
three caching degeneracy removal policies support on-line insertions.




\subsection*{Efficiency Considerations\label{subsec:vda2-efficiency}}

One last item that merits some discussion are the different choices
from the point of view of time- and space-efficiency.

As far as the Voronoi diagram adaptor is concerned, only a copy of the
adaptation traits and a copy of the adaptation policy are stored in it.
The various adaptation traits classes we provide are empty
classes (i.e., they do not store anything). The major time and space
efficiency issues arise from the various implementations of the
adaptation policies.
%
Clearly, the identity policy has no dominant effect on neither the
time or space efficiency. The costs when choosing this policy are due
to the underlying Delaunay graph.

The non-caching degeneracy removal policies create a significant time
overhead since every time we want to access a feature of the Voronoi
diagram, we need to perform geometric tests in order to see if this
feature or one of its neighboring ones has been rejected.
Such a policy is acceptable if we know we are away from
degeneracies or for small input sizes. In the case of the segment
Delaunay graph, it is also the only policy we provide that at the same
time removes degeneracies and allows for on-line insertion of sites.
Caching policies seem to be the best choice for moderate to large
input sizes (1000 sites and more). They do not suffer from the problem
of dealing with degenerate configurations, but since they cache the
results, they increase the space requirements by linear additive
factor.
%
To conclude, if the user is interested in getting a Voronoi diagram
without degenerate features and knows all sites in advance, the best course
of action is to insert all sites at construction time and use a caching
degeneracy removal policy. This strategy avoids the updates of the
cached results after each individual insertion, due to the features of
the Voronoi diagram destroyed because of the site inserted.

%% Copyright (c) 2005  Foundation for Research and Technology-Hellas (Greece).
%% All rights reserved.
%%
%% This file is part of CGAL (www.cgal.org); you may redistribute it under
%% the terms of the Q Public License version 1.0.
%% See the file LICENSE.QPL distributed with CGAL.
%%
%% Licensees holding a valid commercial license may use this file in
%% accordance with the commercial license agreement provided with the software.
%%
%% This file is provided AS IS with NO WARRANTY OF ANY KIND, INCLUDING THE
%% WARRANTY OF DESIGN, MERCHANTABILITY AND FITNESS FOR A PARTICULAR PURPOSE.
%%
%% $URL$
%% $Id$
%% 
%%
%% Author(s)     : Menelaos Karavelas <mkaravel@iacm.forth.gr>

\section{Examples\label{sec:vda2-examples}}

In this section we present an example that shows how to perform point
location queries.

%\subsection{Second example}
%
%The following example shows how we can perform point location using
%the Voronoi diagram adaptor.

\ccIncludeExampleCode{Voronoi_diagram_2/vd_2_point_location.cpp}



% EOF


