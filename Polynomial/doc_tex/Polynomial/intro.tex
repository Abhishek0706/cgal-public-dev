
\ccChapterAuthor{Michael Hemmer}

\section{Fundamentals}

Note that this is just a very brief introduction to polynomials. 
For a quick reference we refer to the Wikipedia or for a more elaborate 
introduction to any class book on elementary algebra. 

A {\em polynomial} is either zero, or can be written as the sum of one or more 
non-zero {\em terms}. The number of terms is finite. A term consist of a 
constant {\em coefficient} and a {\em monomial}, 
that is, the product of zero or more variables. 
Each variable may have an exponent that is a non-negative integer. 
The exponent on a variable in a term is equal to the {\em degree} of that 
variable in that term. A term with no variables is called a constant term.
The degree of a constant term is 0. 

For example, $-7x^3y$ is a term. The coefficient is $-7$, 
the monomial is $x^3y$, comprised of the variables $x$ and $y$, 
the degree of $x$ is three, and the degree of $y$ is one. 
The {\em total degree} of the entire term is the sum of the 
degrees in each variable. 
In the example above, the degree is $3 + 1 = 4$.

A one-variable (univariate) polynomial $f$ of degree $n$ has the following form:

\[f = a_nx^n + a_{n-1}x^{n-1} + ... + a_2x^2 + a_1x + a_0\]

The coefficient $a_0$ is called the {\em constant coefficient}, 
$a_n$ is called the {\em leading coefficient}.  
If $f$ is not the zero polynomial the leading coefficient is not zero. 
The polynomial is called monic if $a_n = 1$. 
In case the coefficient domain of $f$ possess a greatest common divisor 
(gcd) the {\em content} of $f$ is the gcd of all coefficients of $f$. 
For instance, the content of $12 x^3 + 6$ is $6$. 

A multivariate polynomial is a polynomial in more than one variable. 
According to the number of variables it is possible to 
further classify multivariate polynomials as bivariate, trivariate etc. 
In contrast to univariate polynomials the terms of a multivariate polynomial
are not completely ordered by their total degree. However, given a certain order 
on the variables it is possible to define a lexicographic order on the terms. 
Given this order the leading coefficient of a multivariate polynomial is defined 
as the coefficient of the highest term. For instance the leading coefficient of 
the multivariate polynomial $p = 5 x^3y + 7xy^2$ is $7$, given that $y$ has an 
higher order than $x$. 

However, it is also possible to interpret a multivariate polynomial as a 
univariate polynomial in that variable. For instance the trivariate polynomial 
\[q = x^5 + 7x^2y^1z^2 + 13x^1y^2z^2 \in \Z[x,y,z] \]
may be interpreted as a univariate polynomial in $z$, that is, $q$ is interpreted as an element of $R[z]$, with $R=\Z[x,y]$.  
\[q = (13x^1y^2 + 7x^2y^1)z^2 + x^5z^0 \in R[z] \]
In this case the leading coefficient of $q$ with respect to $z$ is
$13x^1y^2 + 7x^2y^1$ and $x^5$ becomes the 'constant' term. 

A {\em homogeneous polynomial} is a polynomial whose terms do all 
have the same total degree. 
For example, $h = x^5 + 7x^2y^1z^2 + 13x^1y^2z^2$ is a homogeneous polynomial 
of degree $5$, in three variables. 



\section{General Design}

The package introduces a concept \ccc{Polynomial_d}, a concept for multivariate 
polynomials in $d$ variables. Though the concept is written for an arbitrary 
number of variables, the number of variables is considered as fixed for a 
particular model of \ccc{Polynomial_d}. 
The concept also allows univariate polynomials.

First of all a model of \ccc{Polynomial_d} is considered as an algebraic 
structure, that is, the ring operations $\{+, -, \cdot\}$ are 
provided due to the fact that \ccc{Polynomial_d} refines at least the 
concept \ccc{IntegralDomainWithoutDivision}. 
However, a model of \ccc{Polynomial_d} has to be accompanied by a traits class 
\ccc{CGAL::Polynomial_traits_d<Polynomial_d>} being a model of 
\ccc{PolynomialTraits_d}. 
In principal the traits class provides all further functionalities 
on polynomials.

Given a $d$-variate polynomial over some base ring $R$ there are at least 
two different possible views on such a polynomial.
\begin{itemize}
\item The recursive or univariate view:  
In this view, a polynomial is considered as 
an element of $R[x_0,\dots,x_{d-2}][x_{d-1}]$. That is, the polynomial 
is treated as a univariate polynomial over the ring $R[x_0,\dots,x_{d-2}]$. 
\item The symmetric or multivariate view: 
This view is almost symmetric with respect to all variables. It considers 
the polynomial as an element of $R [x_0,\dots,x_{d-1}]$.
\end{itemize}

According to these two different views the traits class is required to provide 
two different coefficient types:
\begin{itemize}
\item \ccc{CGAL::Polynomial_traits_d<Polynomial_d>::Coefficient_type} 
representing $R[x_0,\dots,x_{d-2}]$.
\item \ccc{CGAL::Polynomial_traits_d<Polynomial_d>::Innermost_coefficient_type} 
representing the base ring $R$.
\end{itemize}

Another important type which is introduced by this package is 
\ccc{CGAL::Exponent_vector}. 
It is derived from \ccc{std::vector<int>} and used to identify multivariate 
monomials. For instance the exponent vector containing the sequence 
$[3,2,4]$ corresponds to the trivariate monomial $x_0^3x_1^2x_2^4$. 
Note that a vector with negative exponents is considered as invalid. 
However, we decided to in principal allow negative exponents as they may 
appear as intermediate results, in particular we did {\bf not} derive from 
\ccc{std::vector<unsigned int>}. 


% ###########################
\section{Constructing a multivariate polynomial}

First of all the concept \ccc{Polynomial_d} requires that the model is 
constructible from int. This is due to the fact that \ccc{Polynomial_d} 
refines \ccc{IntegralDomainWithoutDivision} which in turn refines 
\ccc{FromIntConstructible}. Of course this allows only the construction of 
constant polynomials. 

In general a polynomial is constructed using the functor 
\ccc{CGAL::Polynomial_traits_d<Polynomial_d>::Construct_polynomial} a model of 
\ccc{PolynomialTraits_d::ConstructPolynomial}.  Basically there are two options:
\begin{itemize}
\item The polynomial is constructed from an iterator range with value type 
\ccc{CGAL::Polynomial_traits_d<Polynomial_d>::Coefficient_type}, 
where the $begin$ iterator refers to the constant term 
(constant with respect to the outermost variable). 
\item The polynomial is constructed from an iterator range with value type 
\ccc{std::pair< CGAL::Exponent_vector, 
CGAL::Polynomial_traits_d<Polynomial_d>::Innermost_coefficient_type>},
where each pair defines the coefficient for the monomial defined by 
the exponent vector. 
\end{itemize}

However, in some cases it might be more convenient to just construct 
the polynomials representing the different variables and to obtain the 
final polynomial using 
algebraic expressions. The most elegant way to construct a certain variable is 
\ccc{CGAL::Polynomial_traits_d<Polynomial_d>::Shift} being a model of 
\ccc{PolynomialTraits_d::Shift}. 

\subsection{Examples}

The following example illustrates different ways to construct a 
bivariate polynomial:
\ccIncludeExampleCode{Polynomial/construction.cpp}


\section{Polynomial parser}

The purpose of the polynomial parser is to provide an alternative way to input CGAL polynomials in a human-readable format. By default, the class \ccc{CGAL::Polynomial<Coeff>} provides an \ccc{operator >>} to read polynomials from an input stream in an ASCII format:
\begin{ccTexOnly}
\[
\begin{array}{l}
P[8(0,P[8(0,P[8(0,2)(2,-16)(4,80)(6,-128)(8,64)])(2,P[0(0,-16)])(4,P[0(0,80)])\\
\quad {}(6,P[0(0,-128)])(8,P[0(0,64)])])(2,P[0(0,P[0(0,-16)])])(4,P[0(0,P[0(0,80)])])\\
\quad {}(6,P[0(0,P[0(0,-128)])])(8,P[0(0,P[0(0,64)])])]
\end{array}
\]
\end{ccTexOnly}
\begin{ccHtmlOnly}
<span>oops..</span>
\end{ccHtmlOnly}
which corresponds to the following trivariate polynomial:
\begin{ccTexOnly}
\[\begin{array}{l}
64z^8 + (-128)z^6 + 80z^4 + (-16)z^2 + (64y^8 + (-128)y^6 + 80y^4 + (-16)y^2 + \\
\quad {}(64x^8 + (-128)x^6 + 80x^4 + (-16)x^2 + 2)).
\end{array}\]
\end{ccTexOnly}
\begin{ccHtmlOnly}
<span>oops..</span>
\end{ccHtmlOnly}
The parser can handle polynomials in any number of indeterminates (variables) and 
can parse arbitrary algebraic expressions, e.g., those of the form: 
\begin{ccTexOnly}
\[\begin{array}{l}
((y-1)^4 - x*y^3)^5 + (x + z)^2 - (2123234523*x^2 - 2*y*y*x + 3^{10}*x*132123)^3,\\
\quad {}(y + x - z + w - 3)^3 = x + y - 123/12312 + 1.00001*z*x^2.
\end{array}\]
\end{ccTexOnly}
\begin{ccHtmlOnly}
<span>oops..</span>
\end{ccHtmlOnly}
Note that, the multiplication symbol (*) is optional and can be omitted in the above expressions.

The parser is configured via the so-called \ccc{parser policy} provided as a template parameter.
There are three parser policies available by default:

\begin{itemize}
\item \ccc{Default_parser_policy}: parses polynomials "as is". In other words, \ccc{Innermost_coefficient_type} of the target polynomial must agree with the type of coefficients
read from the input stream;
\item \ccc{Mixed_rational_parser_policy}: extends \ccc{Default_parser_policy}, and allows mixing coefficients of integer and rational number types in the same expression;
\item \ccc{Mixed_floating_point_parser_policy}: extends \ccc{Mixed_rational_parser_policy}, and allows mixing the coefficients of integer, rational and floating-point number types in the same expression.
\end{itemize}
For additional information on parser policies, see the corresponding reference pages.

\ccIncludeExampleCode{Polynomial/default_parser_policy.cpp}

\ccIncludeExampleCode{Polynomial/mixed_rational_parser_policy.cpp}

\ccIncludeExampleCode{Polynomial/mixed_floating_point_parser_policy.cpp}


% ###########################
\section{Coefficient Access}
 
In order to obtain a certain coefficient the traits class provides 
several functors. 
Note that the functors do not allow a write access to the coefficients. 
\begin{itemize}
\item \ccc{PolynomialTraits_d::GetCoefficient}: 
a model of this concept provides access to a coefficient in the univariate view,
that is, it returns elements of $R[x_0,\dots,x_{d-2}]$.
\item \ccc{PolynomialTraits_d::GetInnermostCoefficient}: 
a model of this concept provides access to a coefficient in the multivariate 
view, that is, it returns elements of $R$.
\item \ccc{PolynomialTraits_d::LeadingCoefficient}: 
a model of this concept provides access to the leading coefficient in 
the univariate view. 
\item \ccc{PolynomialTraits_d::InnermostLeadingCoefficient}: 
a model of this concept provides access to the leading coefficient in 
the multivariate view, that is,
it returns the (innermost) coefficient of the leading multivariate monomial. 
See also  \ccc{PolynomialTraits_d::DegreeVector}.          
\end{itemize}

\subsection{Examples}
The following example illustrates the application of the functors 
discussed above:
\ccIncludeExampleCode{Polynomial/coefficient_access.cpp}

\section{Degree, total degree and degree vector}

There are three functors in \ccc{PolynomialTraits_d} 
related to the degree of a polynomial. 
\begin{itemize}
\item \ccc{PolynomialTraits_d::Degree}: 
a model of this concept returns the degree of the polynomial in the univariate 
view. By default this is the degree with respect to the outermost variable, 
but it is also possible to select another variable. 
\item \ccc{PolynomialTraits_d::TotalDegree}: 
a model of this concept returns the {\em total degree} of a polynomial. 
The polynomial is considered as a multivariate polynomial. 
The total degree is the maximum over the sums of the exponents of each 
multivariate monomial. 
\item \ccc{PolynomialTraits_d::DegreeVector}:
a model of this concept returns the exponent vector of the leading monomial,
where the monomial order is lexicographic and starts with the outermost 
variable. See also  \ccc{PolynomialTraits_d::InnermostLeadingCoefficient}. 
\end{itemize}

\subsection{Examples}
The following example illustrates the application of the functors discussed 
above:
\ccIncludeExampleCode{Polynomial/degree.cpp}

\section{Changing the order of variables}

Given for instance a bivariate polynomial it is conceivable that one wants to 
interchange the role of $x$ and $y$. That is one wants to interpret the 
$x$ as $y$ and vice versa.
For such a case the polynomial traits provides \ccc{PolynomialTraits_d::Swap}:\\
Given a polynomial $p$ and to two indices $i$ and $j$, 
the functor returns the polynomial in which $x_i$ is substituted by $x_j$ and 
vice versa, that is, the variables swap their positions.
The order of the other variables remains untouched. 

Another scenario is, that a particular variable should be moved to another 
position, for instance, it should become the outermost variable while the 
relative order of the other variables remains unchanged. 
For such a case the polynomial traits provides \ccc{PolynomialTraits_d::Move}.

Of course there is also a general method to interchange the order of 
variables, namely \ccc{PolynomialTraits_d::Permute}.

\subsection{Examples}
The following example illustrates the application of the functors discussed 
above:
\ccIncludeExampleCode{Polynomial/swap_move.cpp}


\section{GCD and More}

Since the concept \ccc{PolynomialTraits_d} refines the concept 
\ccc{AlgebraicStructureTraits} the polynomial traits provides functors for 
integral division, division with remainder, greatest common divisor, etc.
But note that the algebraic structure of a polynomial depends on the algebraic 
structure of the innermost coefficient, for instance, a gcd is available 
if and only if the innermost coefficient is a \ccc{Field} or a 
\ccc{UniqueFactorizationDomain}. Hence, we can not provide a $gcd$ if the 
innermost coefficient is just an \ccc{IntegralDomain} since it is simply 
not well defined\footnote{%
An example for such a number type is the template 
\ccc{CGAL::Sqrt_extension<NT,ROOT>} representing an algebraic extension of 
degree two. This is just an \ccc{IntegralDomain} if NT is not a \ccc{Field}. }.
% 
However, if we would consider the polynomial over the quotient field of the 
integral domain the $gcd$ would be well defined. The only problem is
that the result can not be represented over the ring since it contains 
denominators.   
Therefore, the \ccc{PolynomialTraits_d} requires functors such as 
\ccc{PolynomialTraits_d::GcdUpToConstantFactor}. 
This functor computes the gcd of two polynomials up to a constant factor (utcf).
That is, it returns the correct gcd for polynomials over the quotient field, 
but multiplied by some constant such that the result is representable with 
coefficients in the ring. 

However, note that these 'utcf' functions are usually a bit faster than their 
strict counterparts. This is due to the fact that the 'utcf' functions are allowed 
to skip the computation of the correct constant factor.
Note that in many cases the constant factor is in fact not needed.
In particular if the polynomials are supposed to represent some zero set, 
that is, an algebraic curve or surface.  

The concepts for the related functors are: 
\begin{itemize}
\item \ccc{AlgebraicStructureTraits::Gcd}\\     
      \ccc{PolynomialTraits_d::GcdUpToConstantFactor}

\item \ccc{AlgebraicStructureTraits::IntegralDivision}\\        
      \ccc{PolynomialTraits_d::IntegralDivisionUpToConstantFactor}

\item \ccc{PolynomialTraits_d::UnivariateContent}\\
      \ccc{PolynomialTraits_d::UnivariateContentUpToConstantFactor}
       
\item \ccc{PolynomialTraits_d::SquareFreeFactorize}\\
      \ccc{PolynomialTraits_d::SquareFreeFactorizeUpToConstantFactor}
\end{itemize}

%In case of \ccc{PolynomialTraits_d::MultivariateContent} a counterpart does 
%not make sense, since it is defined as the gcd of all innermost coefficients. 

%There is also no counterpart to \ccc{PolynomialTraits_d::MakeSquareFree}, since 
%the result is canonicalized, which is supposed to remove superfluous constant 
%factors anyway. (See also \ccc{PolynomialTraits_d::Canonicalize}.)


Another analog functionality is the pseudo division. 
The related functors replace the usual division with remainder in case the 
Polynomial is not a \ccc{EuclideanRing}. \\
The concepts for the related functors are: 
\begin{itemize}
\item \ccc{AlgebraicStructureTraits::Div_mod}\\
      \ccc{PolynomialTraits_d::PseudoDivision}
\item \ccc{AlgebraicStructureTraits::Div}\\
      \ccc{PolynomialTraits_d::PseudoDivisionQuotient}
\item \ccc{AlgebraicStructureTraits::Mod}\\
      \ccc{PolynomialTraits_d::PseudoDivisionRemainder}
\end{itemize}


\subsection{Examples}
The following example illustrates the application of some functors 
discussed above:
\ccIncludeExampleCode{Polynomial/gcd_up_to_constant_factor.cpp}

\section{Evaluation and Substitution}

Of course, it should also be possible to evaluate a polynomial 
or substitute its variables. We also require a special functor to 
determine whether a polynomial is zero at a given point. 
In case the inner most coefficient is \ccc{RealEmbeddable} the traits 
also must provide a function to compute the sign at a given point. \\
The concepts for the related functors are:
\begin{itemize}
\item \ccc{PolynomialTraits_d::Substitute}\ccGlue
\item \ccc{PolynomialTraits_d::Evaluate}\ccGlue
\item \ccc{PolynomialTraits_d::IsZeroAt}\ccGlue
\item \ccc{PolynomialTraits_d::SignAt}
\end{itemize}

The traits is also required to provide variants of these functors that 
interpret the polynomial as a homogeneous polynomial by adding a virtual 
homogeneous variable such that each term has the same degree, namely the degree of the polynomial.
Of course there is a difference between the univariate and multivariate view. 
For instance the polynomial \[5x^3 + 7x - 3\] has degree 3, hence it is interpreted as 
the homogeneous polynomial \[5x^3 + 7xw^2 -3w^3\] by adding the homogeneous variable $w$. 
In case of the multivariate view each term is filled up by the homogeneous variable such 
that the degree of each term is equal to the total degree of the polynomial. 
Note that these functors may significantly improve efficiency. 
For instance, it is possible to determine the sign of a polynomial 
over integer coefficients at a rational point without 
changing the coefficient domain of the polynomial. 
For more details have a look at the following concepts:
\begin{itemize}
\item \ccc{PolynomialTraits_d::SubstituteHomogeneous} \ccGlue
\item \ccc{PolynomialTraits_d::EvaluateHomogeneous}\ccGlue
\item \ccc{PolynomialTraits_d::IsZeroAtHomogeneous}\ccGlue
\item \ccc{PolynomialTraits_d::SignAtHomogeneous}
\end{itemize}

Note that substitute allows the substitution of the variables by any type that 
is \ccc{ExplicitInteroperable} with the innermost coefficient type. 
This is a very powerful tool since it allows the substitution of the variables 
by polynomials. However, for some standard manipulations such as translation 
or scaling we require special functors since they are expected to be faster 
than their equivalent implementation using substitution:

\begin{itemize}
\item \ccc{PolynomialTraits_d::Shift}\ccGlue
\item \ccc{PolynomialTraits_d::Negate}\ccGlue
\item \ccc{PolynomialTraits_d::Invert}\ccGlue
\item \ccc{PolynomialTraits_d::Translate}\ccGlue
\item \ccc{PolynomialTraits_d::TranslateHomogeneous}\ccGlue
\item \ccc{PolynomialTraits_d::Scale}\ccGlue
\item \ccc{PolynomialTraits_d::ScaleHomogeneous}
\end{itemize}

\subsection{Examples}   
The following example illustrates the application of some functors 
discussed above:
\ccIncludeExampleCode{Polynomial/substitute.cpp}

\section{Resultants, Subresultants and Sturm-Habicht sequences}

The \ccc{PolynomialTraits_d} concept also provides more sophisticated functors
for computations with polynomials --
computing the resultant of two polynomials, 
their polynomial subresultant sequence, with or without cofactors,
and their principal subresultant coefficients.
%
\begin{itemize}
\item \ccRefConceptPage{PolynomialTraits_d::Resultant}\ccGlue
\item \ccRefConceptPage{PolynomialTraits_d::PolynomialSubresultants}\ccGlue
\item \ccRefConceptPage{PolynomialTraits_d::PolynomialSubresultantsWithCofactors}\ccGlue
\item \ccRefConceptPage{PolynomialTraits_d::PrincipalSubresultants}
\end{itemize}
%
Moreover, functors to compute the Sturm-Habicht sequence, with or without
cofactors, and for the principal Sturm-Habicht coefficients exist.
%
\begin{itemize}
\item \ccRefConceptPage{PolynomialTraits_d::SturmHabichtSequence}\ccGlue
\item \ccRefConceptPage{PolynomialTraits_d::SturmHabichtSequenceWithCofactors}\ccGlue
\item \ccRefConceptPage{PolynomialTraits_d::PrincipalSturmHabichtSequence}
\end{itemize}
%
For a formal definition of all used terms, we refer to the corresponding
reference pages.

The principal Sturm-Habicht sequence allows to count the number of
real roots of a polynomial using the function
\begin{itemize}
\item \ccRefIdfierPage{CGAL::number_of_real_roots}.
\end{itemize}
As input, this function requires an iterator range that represents
the principal Sturm-Habicht coefficients. 
This might look complicated at a first sight,
as one has to store the principal Sturm-Habicht sequence temporarily.
However, we remark an important property of the (principal) Sturm-Habicht
sequence. Having a polynomial $f_t(x)$ that depends on a parameter $t$,
and its (principal) Sturm-Habicht coefficients 
$\mathrm{stha}_0(f_t),\ldots,\mathrm{stha}_n(f_t)$, evaluating 
$\mathrm{stha}_0(f_t)$ for $t=t_0$ yields a valid (principal)
Sturm-Habicht sequence for $f_{t_0}$. The same holds for (principal)
subresultants. Thus, it is enough in such situations to compute
the sequence once for the parameter $t$, and call 
\ccc{CGAL::number_of_real_roots} for each specialized parameter
value.

We finally remark that computing subresultants and Sturm-Habicht sequences
introduces an enormous coefficient blow-up.
An application of the functors therefore does not make sense
for built-in integers except for toy examples.
To avoid overflows, one should use arbitrary size integer types
in real applications.

\subsection{Examples}

The following example illustrates how two compute resultants of two
polynomials, and how to count the number of distinct real roots
of a polynomial using its principal Sturm-Habicht coefficients.

\ccIncludeExampleCode{Polynomial/subresultants.cpp}


\section{Design and Implementation History}

This package is the result of the integration process of the NumeriX library 
of \exacus~\cite{beh+-eeeafcs-05} into \cgal.

The class \ccc{CGAL::Polynomial<Coeff>} had been started by Michael Seel within 
CGAL as part of the Nef\_2 package. As part of the \exacus\ project 
it got significantly improved by Arno Eigenwillig and Michael Hemmer. 

However, due to the recursive definition the class was rather restricted to the 
univariate view. Moreover, it is clear that depending on the context 
other classes that are symmetric in all variables or dedicated 
for sparse polynomials may be more efficient. As a consequence this package 
introduced the \ccc{CGAL::Polynomial_traits_d<Polynomial_d>} giving also
the symmetric view on polynomials and the opportunity to introduce and use
other classes representing polynomials within \cgal. 

 


 
