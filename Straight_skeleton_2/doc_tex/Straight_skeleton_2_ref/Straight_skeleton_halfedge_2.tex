% ======================================================================
%
% Copyright (c) 1999 
% Utrecht University (The Netherlands),
% ETH Zurich (Switzerland),
% INRIA Sophia-Antipolis (France),
% Max-Planck-Institute Saarbruecken (Germany),
% and Tel-Aviv University (Israel).
%
%
% This software and related documentation is part of the
% Computational Geometry Algorithms Library (CGAL).
%
% Every use of CGAL requires a license. Licenses come in three kinds:
%
% - For academic research and teaching purposes, permission to use and
%   copy the software and its documentation is hereby granted free of  
%   charge, provided that
%   (1) it is not a component of a commercial product, and
%   (2) this notice appears in all copies of the software and
%       related documentation.
% - Development licenses grant access to the source code of the library 
%   to develop programs. These programs may be sold to other parties as 
%   executable code. To obtain a development license, please contact
%   the GALIA Consortium (at cgal@cs.uu.nl).
% - Commercialization licenses grant access to the source code and the
%   right to sell development licenses. To obtain a commercialization 
%   license, please contact the GALIA Consortium (at cgal@cs.uu.nl).
%
% This software and documentation is provided "as-is" and without
% warranty of any kind. In no event shall the CGAL Consortium be
% liable for any damage of any kind.
%
%
% ----------------------------------------------------------------------
%
% package       : Straight_skeleton_2
% author(s)     : Fernando Cacciola <fernando_cacciola@hotmail.com>
%
% coordinator   : Scisoft (<fernando_cacciola@hotmail.com>)
%
% ======================================================================

\RCSdef{\alphashapeRevision}{$Id$}
\RCSdefDate{\alphashapeDate}{$Date$}

%----------------------------------------------------------------------

\begin{ccRefClass} {Straight_skeleton_halfedge_2<Halfedge>}

\ccDefinition

The class \ccClassTemplateName\ represents a halfedge of the straight skeleton.
A straight skeleton contains two kinds of halfedges: borders and bisectors. Border halfedges correspond to the edges of the input polygon while Bisector halfedges correspond to the straight-line segments that make the skeleton itself.

Each face of the straight skeleton contains exactly 1 border halfedge and 2 or more bisector halfedges.
\ccInheritsFrom

\ccc{Halfedge}

This class is the underlying halfedge class.

\ccTypes

\ccSetThreeColumns{Oriented_side}{}{\hspace*{10cm}}
\ccThreeToTwo


\ccCreation
\ccCreationVariable{e}

\ccConstructor{Straight_skeleton_halfedge();}
{de}




\ccOperations



\ccMethod{halfedge border_edge();}
{A bisector, represented as two bisector-halfedges, is defined by two polygon edges, each one represented by a border-halfedge.
The method returns:
If this is a border-halfedge: itself.
If this is a bisector-halfedge: the border-halfedge associated with this bisector which is found turning around the face CCW (that is, \ccc{face()->halfedge()}). }

\ccMethod{Segment const & segment() const;}
{Returns the oriented straight-line segment which corresponds to the geometric embedding for the halfedge. The source point of this segment is taken from the source vertex, and the target point from the source vertex of the opposite halfedge.}

\ccMethod{bool is_border() const;}
{Returns \ccc{true} if this is a border halfedge (an edge of the input polygon).}

\ccMethod{bool is_bisector() const;}
{Returns \ccc(true) if this is a bisector halfedge (a segment of the skeleton).
\textbf{Very Important: To determine if a halfedge corresponds to a bisector is\_bisector() can be called on any of the halfedges of the pair (because both halfedges will return the same answer). However, to determine if a halfedge corresponds to a border, it must be tested that it is not a bisector. The reason is that calling is\_border () would return true in only one of the halfedges of the pair.}} 
\end{ccRefClass}

