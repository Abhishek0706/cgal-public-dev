%% Copyright (c) 2004  SciSoft.  All rights reserved.
%%
%% This file is part of CGAL (www.cgal.org).
%% You can redistribute it and/or modify it under the terms of the GNU
%% General Public License as published by the Free Software Foundation,
%% either version 3 of the License, or (at your option) any later version.
%%
%% Licensees holding a valid commercial license may use this file in
%% accordance with the commercial license agreement provided with the software.
%%
%% This file is provided AS IS with NO WARRANTY OF ANY KIND, INCLUDING THE
%% WARRANTY OF DESIGN, MERCHANTABILITY AND FITNESS FOR A PARTICULAR PURPOSE.
%%
%% $Name:  $
%%
%% Author(s)     : Fernando Cacciola <fernando_cacciola@hotmail.com>

\begin{ccRefConcept}{PolygonOffsetBuilderTraits_2}

%% \ccHtmlCrossLink{}     %% add further rules for cross referencing links
%% \ccHtmlIndexC[concept]{} %% add further index entries


\ccDefinition

The concept \ccRefName\ describes the requirements for the geometric traits class required by the algorithm class \ccc{Polygon_offset_builder_2<Ss,Gt,Polygon_2>}.

\ccTypes
  \ccNestedType{Kernel}{A model of the \ccc{Kernel} concept.}{}
  \ccGlue
  \ccNestedType{FT}{A \ccc{SqrtFieldNumberType} provided by the kernel. This type is used to represent the
   coordinates of the input points and to specify the desired offset distance.}{}
  \ccGlue
  \ccNestedType{Point_2}{A 2D point type}{}
  \ccGlue
  \ccTypedef{boost::tuple<FT,FT> Vertex;}{A pair of (x,y) coordinates representing a 2D Cartesian point.}
  \ccGlue
  \ccTypedef{boost::tuple<Vertex,Vertex> Edge;}{A pair of vertices representing an edge}
  \ccGlue
  \ccTypedef{boost::tuple<Edge,Edge,Edge> EdgeTriple;}{A triple of edges representing an event}

\ccTwo{PolygonOffsetBuilderTraits_2}{}  
  
\ccNestedType{Compare_offset_against_event_time_2}
{A predicate object type.\\
Must provide \ccc{Comparison_result operator()( FT d, EdgeTriple const& et) const}, which compares the Euclidean distance \ccc{d} with the event time for \ccc{et}. Such event time is the Euclidean distance at which the \textit{offset lines} intersect in a single point. The source of such offset lines is given by the 3 \textit{oriented} lines defined by the edge-triple \ccc{et}
\ccPrecond{\ccc{et} must be an edge-triple corresponding to an event that actually exist (that is, there must exist an offset distance \ccc{t} $>0$ at which the offset lines do intersect at a single point.}}

\ccNestedType{Construct_offset_point_2}
{A construction object type.\\
Must provide \ccc{boost::optional<Point_2> operator()( FT t, Edge const& x, Edge const& y) const}, which constructs the point of intersection of the lines obtained by offsetting the oriented lines given by \ccc{x} and \ccc{y} an Euclidean distance \ccc{t}.
If the point cannot be computed, not even approximately (because of overflow for instance), an empty optional must be returned.\\ 
\ccPrecond{\ccc{x} and \ccc{y} must intersect in a single point}}

\ccNestedType{Construct_ss_vertex_2}
{A construction object type.\\
Must provide \ccc{Vertex operator()( Point_2 const& p)}, which given a \ccc{Point_2} \ccc{p} returns a Vertex encapsulating the corresponding (x,y) pair of Cartesian coordinates.}

\ccNestedType{Construct_ss_edge_2}
{A construction object type.\\
Must provide \ccc{Edge operator()( Point_2 const& s, Point_2 const& t)}, which given source and target points \ccc{s} and \ccc{t} returns an Edge encapsulating the corresponding input segment (in Cartesian coordinates.)}

\ccNestedType{Construct_ss_triedge_2}
{A construction object type.\\
Must provide \ccc{Triedge operator()( Edge const& e0, Edge const& e1, Edge const& e2)}, which given the 3 edges that define an event, \ccc{e0}, \ccc{e1} and \ccc{e2}, returns a Triedge encapsulating them.}

\ccHasModels

\ccc{CGAL::Polygon_offset_builder_traits_2<K>}.

\ccSeeAlso

\ccc{CGAL::Polygon_offset_builder_2<Ss,Gt,Polygon_2>}\\
\ccc{CGAL::Polygon_offset_builder_traits_2<K>}\\

\end{ccRefConcept}

% +------------------------------------------------------------------------+
%%RefPage: end of main body, begin of footer
% EOF
% +------------------------------------------------------------------------+
