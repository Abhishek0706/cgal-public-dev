% +------------------------------------------------------------------------+
% | Reference manual page: Construct_arc_2.tex
% +------------------------------------------------------------------------+
% | 27.03.2008   Author
% | Package: Curved_kernel_via_analysis_2
% |
\RCSdef{\RCSConstructarcRev}{$Id: header.tex 40270 2007-09-07 15:29:10Z lsaboret $}
\RCSdefDate{\RCSConstructarcDate}{$Date: 2007-09-07 17:29:10 +0200 (Fri, 07 Sep 2007) $}
% |
\ccRefPageBegin
%%RefPage: end of header, begin of main body
% +------------------------------------------------------------------------+


\begin{ccRefFunctionObjectClass}[CGAL::internal::Curved_kernel_via_analysis_2_Functors::]{Construct_arc_2}  %% add template arg's if necessary
\ccRefLabel{Construct_arc_2}

%% \ccHtmlCrossLink{}     %% add further rules for cross referencing links
%% \ccHtmlIndexC[class]{} %% add further index entries

\ccDefinition

%The function object class \ccRefName\ does this and that.

% The section below is automatically generated. Do not edit!
%START-AUTO(\ccInclude)

\ccInclude{CGAL/Curved_kernel_via_analysis_2/Curved_kernel_via_analysis_2_functors.h}

%END-AUTO(\ccInclude)

% The section below is automatically generated. Do not edit!
%START-AUTO(\ccDefinition)

Functor to construct an x-monotone arc.

%END-AUTO(\ccDefinition)


\ccParameters

% The section below is automatically generated. Do not edit!
%START-AUTO(\ccParameters)

template$<$  \\
class \ccc{CurvedKernelViaAnalysis_2}$>$   \\
class \ccc{Construct_arc_2};

%END-AUTO(\ccParameters)


\ccInheritsFrom

% The section below is automatically generated. Do not edit!
%START-AUTO(\ccInheritsFrom)

\ccc{Curved_kernel_via_analysis_2_functor_base<CurvedKernelViaAnalysis_2>}

%END-AUTO(\ccInheritsFrom)


\ccIsModel

\ccTypes

%\ccCreation
%\ccCreationVariable{a}  %% choose variable name

%% \ccIncludeExampleCode{Curved_kernel_via_analysis_2/Construct_arc_2.C}

% The section below is automatically generated. Do not edit!
%START-AUTO(\ccTypes)

\ccNestedType{Curved_kernel_via_analysis_2}
{
this instance' first template parameter
}
\ccGlue
\ccNestedType{Base}
{
the base type
}
\ccGlue
\ccNestedType{Curve_2}
{
the curve type
}
\ccGlue
\ccNestedType{Point_2}
{
the point type
}
\ccGlue
\ccNestedType{Arc_2}
{
the arc type
}
\ccGlue
\ccNestedType{Curve_analysis_2}
{
type of curve analaysis
}
\ccGlue
\ccNestedType{Coordinate_1}
{
the x-coordinate type
}
\ccGlue
\ccNestedType{result_type}
{
the result type
}
\ccGlue
\ccNestedType{Curve_kernel_2}
{
[inherited] \\
type of curve kernel
}
\ccGlue

%END-AUTO(\ccTypes)

\ccHeading{Variables}

% The section below is automatically generated. Do not edit!
%START-AUTO(\ccHeading{Variables})

\ccVariable{Curved_kernel_via_analysis_2* _m_curved_kernel;}
{
[protected, inherited] \\
stores pointer to \ccc{Curved_kernel_via_analysis_2}
}
\ccGlue

%END-AUTO(\ccHeading{Variables})

\ccCreation
\ccCreationVariable{a}  %% choose variable name for \ccMethod below

% The section below is automatically generated. Do not edit!
%START-AUTO(\ccCreation)

\ccConstructor{Construct_arc_2(Curved_kernel_via_analysis_2 * kernel);}
{
Standard constructor.
}
\ccGlue
\begin{description}
\item[Parameters:]
\begin{description}
\item[kernel]The kernel \end{description}
\end{description}
\ccGlue

%END-AUTO(\ccCreation)

\ccOperations

% The section below is automatically generated. Do not edit!
%START-AUTO(\ccOperations)

\ccMethod{Arc_2 operator()(const Point_2& p, const Point_2& q, const Curve_analysis_2& c, int arcno, int arcno_p, int arcno_q);}
{
Constructs a non-vertical arc with two interior end-points (segment).
}
\ccGlue
\begin{description}
\item[Parameters:]
\begin{description}
\item[p]first endpoint \item[q]second endpoint \item[c]The supporting curve \item[arcno]The arcnumber wrt c in the interior of the arc \item[\ccc{arcno_p}]The arcnumber wrt c of the arc at p \item[\ccc{arcno_q}]The arcnumber wrt c of the arc at q \end{description}
\end{description}
\begin{description}
\item[Returns:]The constructed segment\end{description}
\begin{description}
\item[Precondition:]p.x() != q.x() \end{description}
\ccGlue
\ccMethod{Arc_2 operator()(const Point_2& origin, Arr_curve_end inf_end, const Curve_analysis_2& c, int arcno, int arcno_o);}
{
Constructs a non-vertical arc with one interior end-point and whose other end approaches the left or right boundary of the parameter space (ray I).
}
\ccGlue
\begin{description}
\item[Parameters:]
\begin{description}
\item[origin]The interior end-point of the ray \item[\ccc{inf_end}]Defining whether the arc emanates from the left or right boundary \item[c]The supporting curve \item[arcno]The arcnumber wrt c in the interior of the arc \item[\ccc{arcno_o}]The arcnumber wrt c of the arc at origin \end{description}
\end{description}
\begin{description}
\item[Returns:]The constructed ray \end{description}
\ccGlue
\ccMethod{Arc_2 operator()(const Point_2& origin, const Coordinate_1& asympt_x, Arr_curve_end inf_end, const Curve_analysis_2& c, int arcno, int arcno_o);}
{
Constructs a non-vertical arc with one interior end-point and whose other end approaches a vertical asymptote (ray II).
}
\ccGlue
\begin{description}
\item[Parameters:]
\begin{description}
\item[origin]The interior end-point \item[\ccc{asympt_x}]The x-coordinate of the vertical asymptote \item[\ccc{inf_end}]Arc is approaching the bottom or top boundary \item[c]The supporting curve \item[arcno]The arcnumber wrt c in the interior of the arc \item[\ccc{arcno_o}]The arcnumber wrt c of the arc at origin \end{description}
\end{description}
\begin{description}
\item[Returns:]The constructed ray\end{description}
\begin{description}
\item[Precondition:]origin.x() != \ccc{asympt_x} \end{description}
\ccGlue
\ccMethod{Arc_2 operator()(const Curve_analysis_2& c, int arcno);}
{
Constructs a non-vertical arc with two non-interior ends at the left and right boundary (branch I).
}
\ccGlue
\begin{description}
\item[Parameters:]
\begin{description}
\item[c]The supporting curve \item[arcno]The arcnumber wrt to c in the interior of the arc \end{description}
\end{description}
\begin{description}
\item[Returns:]The constructed branch \end{description}
\ccGlue
\ccMethod{Arc_2 operator()(const Coordinate_1& asympt_x1, Arr_curve_end inf_end1, const Coordinate_1& asympt_x2, Arr_curve_end inf_end2, const Curve_analysis_2& c, int arcno);}
{
Constructs a non-vertical arc with two ends approaching vertical asymptotes (branch II).
}
\ccGlue
\begin{description}
\item[Parameters:]
\begin{description}
\item[\ccc{asympt_x1}]The x-coordinate of the first asymptote \item[\ccc{inf_end1}]Arc is approaching the bottom or top boundary at \ccc{asympt_x1} \item[\ccc{asympt_x2}]The x-coordinate of the second asymptote \item[\ccc{inf_end2}]Arc is approaching the bottom or top boundary at \ccc{asympt_x2} \end{description}
\end{description}
\begin{description}
\item[Returns:]The constructed branch\end{description}
\begin{description}
\item[Precondition:]\ccc{asympt_x1} != \ccc{asympt_x2} \end{description}
\ccGlue
\ccMethod{Arc_2 operator()(Arr_curve_end inf_endx, const Coordinate_1& asympt_x, Arr_curve_end inf_endy, const Curve_analysis_2& c, int arcno);}
{
Construct a non-vertical arc with one left- or right-boundary end and one end that approaches a vertical asymptote (branch III).
}
\ccGlue
\begin{description}
\item[Parameters:]
\begin{description}
\item[\ccc{inf_endx}]Defining whether the arc emanates from the left or right boundary \item[\ccc{asympt_x}]The x-coordinate of the asymptote \item[\ccc{inf_endy}]Arc is approaching the bottom or top boundary at \ccc{asympt_x} \end{description}
\end{description}
\begin{description}
\item[Returns:]The constructed branch \end{description}
\ccGlue
\ccMethod{Arc_2 operator()(const Point_2& p, const Point_2& q, const Curve_analysis_2& c);}
{
Constructs a vertical arc with two interior end-points (vertical segment).
}
\ccGlue
\begin{description}
\item[Parameters:]
\begin{description}
\item[p]The first end-point \item[q]The second end-point \item[c]The supporting curve \end{description}
\end{description}
\begin{description}
\item[Returns:]The constructed arc\end{description}
\begin{description}
\item[Precondition:]p != q \&\& p.x() == q.x()
c must have a vertical component at this x \end{description}
\ccGlue
\ccMethod{Arc_2 operator()(const Point_2& origin, Arr_curve_end inf_end, const Curve_analysis_2& c);}
{
Constructs a vertical arc with one interior end-point and one that reaches the bottom or top boundary (vertical ray).
}
\ccGlue
\begin{description}
\item[Parameters:]
\begin{description}
\item[origin]The interior end-point \item[\ccc{inf_end}]Ray emanates from bottom or top boundary \end{description}
\end{description}
\begin{description}
\item[Returns:]The constructed ray\end{description}
\begin{description}
\item[Precondition:]c must have a vertical line component at this x \end{description}
\ccGlue
\ccMethod{Arc_2 operator()(const Coordinate_1& x, const Curve_analysis_2& c);}
{
Constructs a vertical arc that connects bottom with top boundary (vertical branch).
}
\ccGlue
\begin{description}
\item[Parameters:]
\begin{description}
\item[x]The x-coordinate of the branch \end{description}
\end{description}
\begin{description}
\item[Returns:]The constructed branch\end{description}
\begin{description}
\item[Precondition:]c must have a vertical line component at this x \end{description}
\ccGlue
\ccMethod{Curved_kernel_via_analysis_2* _ckva() const;}
{
[protected, inherited] \\
Return pointer to curved kernel.
}
\ccGlue
\begin{description}
\item[Returns:]Pointer to stored kernel \end{description}
\ccGlue

%END-AUTO(\ccOperations)

\end{ccRefFunctionObjectClass}

% +------------------------------------------------------------------------+
%%RefPage: end of main body, begin of footer
\ccRefPageEnd
% EOF
% +------------------------------------------------------------------------+

