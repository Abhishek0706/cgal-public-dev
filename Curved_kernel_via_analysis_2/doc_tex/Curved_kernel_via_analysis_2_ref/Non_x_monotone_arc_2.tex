% +------------------------------------------------------------------------+
% | Reference manual page: Non_x_monotone_arc_2.tex
% +------------------------------------------------------------------------+
% | 27.03.2008   Author
% | Package: Curved_kernel_via_analysis_2
% |
\RCSdef{\RCSNonxmonotonearcRev}{$Id: header.tex 40270 2007-09-07 15:29:10Z lsaboret $}
\RCSdefDate{\RCSNonxmonotonearcDate}{$Date: 2007-09-07 17:29:10 +0200 (Fri, 07 Sep 2007) $}
% |
\ccRefPageBegin
%%RefPage: end of header, begin of main body
% +------------------------------------------------------------------------+


\begin{ccRefClass}[CGAL::internal::]{Non_x_monotone_arc_2}  %% add template arg's if necessary
\ccRefLabel{Non_x_monotone_arc_2}

%% \ccHtmlCrossLink{}     %% add further rules for cross referencing links
%% \ccHtmlIndexC[class]{} %% add further index entries

\ccDefinition

%The class \ccRefName\ does this and that.

% The section below is automatically generated. Do not edit!
%START-AUTO(\ccInclude)

\ccInclude{CGAL/Curved_kernel_via_analysis_2/Non_x_monotone_arc_2.h}

%END-AUTO(\ccInclude)

% The section below is automatically generated. Do not edit!
%START-AUTO(\ccDefinition)

Class representing a (not necessary x-monotone) curve arc.

This class represents a not necessarily x-monotone curve arc. The arc is given as a list of connected x-monotone pieces.

By constructing a new arc, its validity is checked to ensure that its x-monotone pieces form a single chain

%END-AUTO(\ccDefinition)


\ccParameters

% The section below is automatically generated. Do not edit!
%START-AUTO(\ccParameters)

template$<$  \\
class \ccc{CurvedKernelViaAnalysis_2},   \\
class \ccc{Rep_} = \ccc{Non_x_monotone_arc_2_rep<CurvedKernelViaAnalysis_2>}$>$   \\
class \ccc{Non_x_monotone_arc_2};

%END-AUTO(\ccParameters)


\ccInheritsFrom

% The section below is automatically generated. Do not edit!
%START-AUTO(\ccInheritsFrom)

\ccc{Handle_with_policy}

%END-AUTO(\ccInheritsFrom)


\ccIsModel

\ccTypes

%\ccCreation
%\ccCreationVariable{a}  %% choose variable name

%% \ccIncludeExampleCode{Curved_kernel_via_analysis_2/Non_x_monotone_arc_2.C}

% The section below is automatically generated. Do not edit!
%START-AUTO(\ccTypes)

\ccNestedType{Curved_kernel_via_analysis_2}
{
this instance's first template parameter
}
\ccGlue
\ccNestedType{Rep}
{
this instance's second template parameter
}
\ccGlue
\ccNestedType{Self}
{
this instance itself
}
\ccGlue
\ccNestedType{Curve_kernel_2}
{
type of curve kernel
}
\ccGlue
\ccNestedType{Curve_analysis_2}
{
type of analysis of a pair of curves
}
\ccGlue
\ccNestedType{Point_2}
{
type of a point on generic curve
}
\ccGlue
\ccNestedType{Arc_2}
{
type of an x-monotone arc on generic curve
}
\ccGlue
\ccNestedType{Arc_const_iterator}
{
iterator type to range through the list of x-monotone arcs
}
\ccGlue
\ccNestedType{Base}
{
the handle superclass
}
\ccGlue

%END-AUTO(\ccTypes)

\ccCreation
\ccCreationVariable{a}  %% choose variable name for \ccMethod below

% The section below is automatically generated. Do not edit!
%START-AUTO(\ccCreation)

\ccConstructor{Non_x_monotone_arc_2();}
{
Default constructor.
}
\ccGlue
\ccConstructor{Non_x_monotone_arc_2(const Self& a);}
{
copy constructor
}
\ccGlue
\ccConstructor{Non_x_monotone_arc_2(const Arc_2& arc);}
{
constructs an arc from one x-monotone piece
}
\ccGlue
\ccConstructor{Non_x_monotone_arc_2(InputIterator start, InputIterator end);}
{
constructs an arc from a list x-monotone pieces specified by iterator range [start; end)
template argument type of InputIterator is \ccc{Arc_2}
}
\ccGlue
\begin{description}
\item[Precondition:]the x-monotone arcs must be connected into a single chain
either all x-monotone arcs must be vertical or non-vertical \end{description}
\ccGlue
\ccConstructor{Non_x_monotone_arc_2();}
{
Default constructor.
}
\ccGlue
\ccConstructor{Non_x_monotone_arc_2(const Self& a);}
{
copy constructor
}
\ccGlue
\ccConstructor{Non_x_monotone_arc_2(const Arc_2& arc);}
{
constructs an arc from one x-monotone piece
}
\ccGlue
\ccConstructor{Non_x_monotone_arc_2(InputIterator start, InputIterator end);}
{
constructs an arc from a list x-monotone pieces specified by iterator range [start; end)
template argument type of InputIterator is \ccc{Arc_2}
}
\ccGlue
\begin{description}
\item[Precondition:]the x-monotone arcs must be connected into a single chain
either all x-monotone arcs must be vertical or non-vertical \end{description}
\ccGlue

%END-AUTO(\ccCreation)

\ccOperations

% The section below is automatically generated. Do not edit!
%START-AUTO(\ccOperations)

\ccMethod{int number_of_x_monotone_arcs() const;}
{
returns the number of x-monotone arcs this object consists of
}
\ccGlue
\ccMethod{Arc_const_iterator begin() const;}
{
returns iterator pointing to the first x-monotone arc in the list
}
\ccGlue
\ccMethod{Arc_const_iterator end() const;}
{
returns iterator pointing to beyond the last x-monotone arc in the list
}
\ccGlue
\ccMethod{const Arc_2& x_monotone_arc(int i) const;}
{
returns a distinct ith x-monotone piece of the arc
}
\ccGlue
\ccMethod{Curve_analysis_2 curve() const;}
{
returns the supporting curve
}
\ccGlue
\ccMethod{bool is_vertical() const;}
{
returns true if this arc consists of vertical segments
}
\ccGlue

%END-AUTO(\ccOperations)

\ccHeading{Friends And Related Functions}

% The section below is automatically generated. Do not edit!
%START-AUTO(\ccHeading{Friends And Related Functions})

\ccFunction{std::ostream& operator<<(std::ostream& os, const Non_x_monotone_arc_2< CurvedKernelViaAnalysis_2, Rep_>& arc);}
{
output operator
}
\ccGlue
\ccFunction{Qt_widget& operator<<(Qt_widget& ws, const Non_x_monotone_arc_2< CurvedKernelViaAnalysis_2, Rep_>& arc);}
{
draws an arc to Qt window qt
}
\ccGlue

%END-AUTO(\ccHeading{Friends And Related Functions})

\end{ccRefClass}

% +------------------------------------------------------------------------+
%%RefPage: end of main body, begin of footer
\ccRefPageEnd
% EOF
% +------------------------------------------------------------------------+

