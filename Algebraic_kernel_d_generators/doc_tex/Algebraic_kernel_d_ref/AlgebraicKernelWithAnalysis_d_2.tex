\begin{ccRefConcept}{AlgebraicKernelWithAnalysis_d_2} 

\ccDefinition

The \ccc{AlgebraicKernelWithAnalysis_d_2} concept refines
the \ccc{AlgebraicKernel_d_2} concept by interpreting bivariate polynomials
as real algebraic plane curves. That is, for given bivariate polynomial~$f$,
we consider the curve in the two-dimensional real plane induced by the set of
vanishing points of~$p$: $V_\mathbb{R}(p) := \{(x,y) \in \mathbb{R}^2
\mid p(x,y) = 0 \}$. The kernel provides a way to analyze a single
curves and one to analyze pairs of them. Each such analysis does so in two
steps: First, critical \ccc{x}-coordinates are detected. Second, for each such
coordinate and coordinates contained in open
intervals in between such, status lines are computed. Such a line reflects
the topology of a curve (or a pair of curves) in \ccc{y}-direction
over the coordinate (or interval, respectively). A status line always exists at
a specific \ccc{x}-coordinate (for an interval, a representative might be
chosen) and thus, a status line is also expected to provide access to the
\ccc{y}-coordinates of points of the algebraic curve along the line, where the
topology changes. But mainly, the ``$y$-per-$x$-analyzes'' provide easy
combinatorial access to the topology of curves and pairs of curves. 

If not stated otherwise, a model is required to 
provide the analysis for algebraic curves of general degree $d$ in $\R^2$.

\ccRefines
\ccc{AlgebraicKernel_d_2}

\ccTypes \ccThree{}{+++++++++++++}{++++++++}

A model of \ccc{AlgebraicKernelWithAnalysis_d_2} must provide

\ccNestedType{CurveAnalysis_2}{A model of concept
\ccc{AlgebraicKernelWithAnalysis_d_2::CurveAnalysis_2}.}
\ccGlue
\ccNestedType{CurvePairAnalysis_2}{A model of concept
\ccc{AlgebraicKernelWithAnalysis_d_2::CurvePairAnalysis_2}.}

\ccHasModels

%Algebraic_curve_kernel_2

\ccSeeAlso



\end{ccRefConcept}
