% +------------------------------------------------------------------------+
% | Reference manual page: Lines_through_segments_output_obj.tex (Lines_through_segments_3)
% +------------------------------------------------------------------------+
% | 
% | Package: LTS (Lines_through_segments_3)
% | 
% +------------------------------------------------------------------------+

\ccRefPageBegin

%%RefPage: end of header, begin of main body
% +------------------------------------------------------------------------+

%%%%%%%%%%%%%%%%%%%%%%%%%%%%%%%%%%%%%%%%%%%%%%%%%%%%%
%% Lines_through_segments_output_line_3 class       %
%%%%%%%%%%%%%%%%%%%%%%%%%%%%%%%%%%%%%%%%%%%%%%%%%%%%%
\begin {ccRefClass} {Lines_through_segments_output_line_3}
    
\ccDefinition 

The \ccRefName\ class holds a line tangent to 4 line segments.

\ccInclude{CGAL/Lines_through_segments_output_obj.h}

\ccTypes
\ccNestedType{Line_3}{The line type.}
\ccNestedType{Rational_segment_3}{The input segment type.}

\ccAccessFunctions
\ccCreationVariable{obj}

\ccMethod{Line_3 line() const;}
         {Returns the algebraic line of \ccVar{}.}

\ccSeeAlso
\ccc{CGAL::Lines_through_segments_output_arr_curve_2}\\
\ccc{CGAL::Lines_through_segments_output_arr_arc_2}\\
\ccc{CGAL::Lines_through_segments_output_arr_polygon_2}\\
\ccc{CGAL::Lines_through_segments_output_point_3}\\
\ccc{CGAL::Lines_through_segments_output_overlap_segments_3}\\


\end{ccRefClass} 


%%%%%%%%%%%%%%%%%%%%%%%%%%%%%%%%%%%%%%%%%%%%%%%%%%%%%
%% Lines_through_segments_output_arr_curve_2 class  %
%%%%%%%%%%%%%%%%%%%%%%%%%%%%%%%%%%%%%%%%%%%%%%%%%%%%%
\begin {ccRefClass} {Lines_through_segments_output_arr_curve_2}
    
\ccDefinition 

The \ccRefName\ class holds a curve on the plane arrangement. Each point on the curve represents a line tangent to 4 line segments. The class holds the 2 line segments $S_1$ and $S_2$ that parametrized the arrangement.\\
In order to get a crossing line L from a point (x, y) on the curve, the following should be done:

\begin{enumerate}
\item 
From $x$ we get a point $p_1$ on $S_1$, $S_1 = ({x^1}_0,{y^1}_0,{z^1}_0) + x \cdot ({x^1}_1,{y^1}_1,{z^1}_1)$.
\item
From $y$ we get a point $p_2$ on $S_2$, $S_2 = ({x^2}_0,{y^2}_0,{z^2}_0) + y \cdot ({x^2}_1,{y^2}_1,{z^2}_1)$.
\item
The line that passes through $p_1$ and $p_2$, is tangent to the 4 line segments.
\end{enumerate}
\ccInclude{CGAL/Lines_through_segments_output_obj.h}


\ccTypes
\ccNestedType{Curve_2} {The curve type.}
\ccNestedType{Rational_segment_3}{The input segment type.}

\ccAccessFunctions
\ccCreationVariable{obj}

\ccMethod{Line_3 curve() const;}
         {Returns the curve of \ccVar{}.}

\ccMethod{Line_3 s1() const;}
         {Returns the input rational segment that parametrized the x-axis at the arrangement.}

\ccMethod{Line_3 s2() const;}
         {Returns the input rational segment that parametrized the y-axis at the arrangement.}

\ccSeeAlso
\ccc{CGAL::Lines_through_segments_output_line_3}\\
\ccc{CGAL::Lines_through_segments_output_arr_arc_2}\\
\ccc{CGAL::Lines_through_segments_output_arr_polygon_2}\\
\ccc{CGAL::Lines_through_segments_output_point_3}\\
\ccc{CGAL::Lines_through_segments_output_overlap_segments_3}\\



\end{ccRefClass} 

%%%%%%%%%%%%%%%%%%%%%%%%%%%%%%%%%%%%%%%%%%%%%%%%%%%%%
%% Lines_through_segments_output_arr_arc_2 class    %
%%%%%%%%%%%%%%%%%%%%%%%%%%%%%%%%%%%%%%%%%%%%%%%%%%%%%
\begin {ccRefClass} {Lines_through_segments_output_arr_arc_2}
    
\ccDefinition 

The \ccRefName\ class holds a arc on the sphere arrangement. Each direction on the arc represents a line tangent to 4 line segments. The class holds the 2 line segments $S_1$ and $S_2$ that parametrized the arrangement.\\
In order to get a crossing line L from a direction (dx, dy, dz) on the arc, the following should be done:

\begin{enumerate}
\item 
Get the intersection point, IS = (x, y, z) of $S_1$ and $S_2$ (with the function get\_intersection\_point()).
\item
Each line on the wedge that defined by the intersection point and the direction IS, passes through 4 segments.\\
$L = (x, y, z) + t \cdot (dx, dy, dz)$

\end{enumerate}
\ccInclude{CGAL/Lines_through_segments_output_obj.h}

\ccTypes
\ccNestedType{Arc_2} {The arc type.}
\ccNestedType{Rational_segment_3}{The input segment type.}
\ccNestedType{Point_3}{The intersection point type.}

\ccAccessFunctions
\ccCreationVariable{obj}

\ccMethod{Line_3 arc() const;}
         {Returns the arc of \ccVar{}.}

\ccMethod{Point_3 intersection_point() const;}
         {Returns the intersection point of $S_1$ and $S_2$.}

\ccMethod{Line_3 s1() const;}
         {Returns the input rational segment $S_1$.}

\ccMethod{Line_3 s2() const;}
         {Returns the input rational segment $S_2$.}

\ccSeeAlso
\ccc{CGAL::Lines_through_segments_output_line_3}\\
\ccc{CGAL::Lines_through_segments_output_arr_curve_2}\\
\ccc{CGAL::Lines_through_segments_output_arr_polygon_2}\\
\ccc{CGAL::Lines_through_segments_output_point_3}\\
\ccc{CGAL::Lines_through_segments_output_overlap_segments_3}\\



\end{ccRefClass} 

%%%%%%%%%%%%%%%%%%%%%%%%%%%%%%%%%%%%%%%%%%%%%%%%%%%%%
%% Lines_through_segments_output_arr_polygon_2 class  %
%%%%%%%%%%%%%%%%%%%%%%%%%%%%%%%%%%%%%%%%%%%%%%%%%%%%%
\begin {ccRefClass} {Lines_through_segments_output_arr_polygon_2}
    
\ccDefinition 

The \ccRefName\ class holds a polygon on the plane arrangement. Each point on the polygon represents a line tangent to 4 line segments. The class holds the 2 line segments $S_1$ and $S_2$ that parametrized the arrangement.\\
In order to get a crossing line L from a point (x, y) on the polygon, the following should be done:

\begin{enumerate}
\item 
From $x$ we get a point $p_1$ on $S_1$, $S_1 = ({x^1}_0,{y^1}_0,{z^1}_0) + x \cdot ({x^1}_1,{y^1}_1,{z^1}_1)$.
\item
From $y$ we get a point $p_2$ on $S_2$, $S_2 = ({x^2}_0,{y^2}_0,{z^2}_0) + y \cdot ({x^2}_1,{y^2}_1,{z^2}_1)$.
\item
The line that passes through $p_1$ and $p_2$, passes through 4 line segments.
\end{enumerate}
\ccInclude{CGAL/Lines_through_segments_output_obj.h}

\ccTypes
\ccNestedType{Polygon_2} {The polygon type.}
\ccNestedType{Rational_segment_3}{The input segment type.}

\ccAccessFunctions
\ccCreationVariable{obj}

\ccMethod{Line_3 polygon() const;}
         {Returns the polygon of \ccVar{}.}

\ccMethod{Line_3 s1() const;}
         {Returns the input rational segment that parametrized the x-axis at the arrangement.}

\ccMethod{Line_3 s2() const;}
         {Returns the input rational segment that parametrized the y-axis at the arrangement.}

\ccSeeAlso
\ccc{CGAL::Lines_through_segments_output_line_3}\\
\ccc{CGAL::Lines_through_segments_output_arr_arc_2}\\
\ccc{CGAL::Lines_through_segments_output_arr_curve_2}\\
\ccc{CGAL::Lines_through_segments_output_point_3}\\
\ccc{CGAL::Lines_through_segments_output_overlap_segments_3}\\


\end{ccRefClass} 

%%%%%%%%%%%%%%%%%%%%%%%%%%%%%%%%%%%%%%%%%%%%%%%%%%%%%
%% Lines_through_segments_output_point_3 class       %
%%%%%%%%%%%%%%%%%%%%%%%%%%%%%%%%%%%%%%%%%%%%%%%%%%%%%
\begin {ccRefClass} {Lines_through_segments_output_point_3}
    
\ccDefinition 

The \ccRefName\ class holds an intersection point of four or more concurrent line segments.

\ccInclude{CGAL/Lines_through_segments_output_obj.h}


\ccTypes
\ccNestedType{Rational_point_3}{The point type.}

\ccAccessFunctions
\ccCreationVariable{obj}

\ccMethod{Segment_3 point() const;}
         {Returns the point of \ccVar{}.}

\ccSeeAlso
\ccc{CGAL::Lines_through_segments_output_line_3}\\
\ccc{CGAL::Lines_through_segments_output_arr_curve_2}\\
\ccc{CGAL::Lines_through_segments_output_arr_arc_2}\\
\ccc{CGAL::Lines_through_segments_output_arr_polygon_2}\\
\ccc{CGAL::Lines_through_segments_output_overlap_segments_3}\\


\end{ccRefClass} 

%%%%%%%%%%%%%%%%%%%%%%%%%%%%%%%%%%%%%%%%%%%%%%%%%%%%%
%% Lines_through_segments_output_overlap_segment_3 class  %
%%%%%%%%%%%%%%%%%%%%%%%%%%%%%%%%%%%%%%%%%%%%%%%%%%%%%
\begin {ccRefClass} {Lines_through_segments_output_overlap_segment_3}
    
\ccDefinition 

The \ccRefName\ class holds a common line segment to four or more overlapping line segments.

\ccInclude{CGAL/Lines_through_segments_output_obj.h}

\ccTypes
\ccNestedType{Rational_segment_3}{The line segment type.}

\ccAccessFunctions
\ccCreationVariable{obj}

\ccMethod{Segment_3 segment() const;}
         {Returns the line segment of \ccVar{}.}

\ccSeeAlso
\ccc{CGAL::Lines_through_segments_output_line_3}\\
\ccc{CGAL::Lines_through_segments_output_arr_curve_2}\\
\ccc{CGAL::Lines_through_segments_output_arr_arc_2}\\
\ccc{CGAL::Lines_through_segments_output_arr_polygon_2}\\
\ccc{CGAL::Lines_through_segments_output_point_3}\\


\end{ccRefClass} 

% +------------------------------------------------------------------------+
%%RefPage: end of main body, begin of footer
\ccRefPageEnd
% EOF
% +------------------------------------------------------------------------+
