% +------------------------------------------------------------------------+
% | Reference manual page: Lines_through_segments_traits_3.tex
% +------------------------------------------------------------------------+
% | 12.09.2010   Author Asaf Porat
% | Package: Lines through segments 3
% | 
\RCSdef{\RCSLinesThroughSegmentsTraits_3Rev}{$Id: lts_concept.tex 28489 2010-09-12 10:08:15Z lsaboret $}
\RCSdefDate{\RCSLinesThroughSegmentsTraits_3Date}{$Date: 2010-09-12 12:08:15 +0200 (Tue, 14 Feb 2006) $}
% |
%%RefPage: end of header, begin of main body
% +------------------------------------------------------------------------+

\begin{ccRefConcept}{LinesThroughSegmentsTraits_3}

\ccDefinition
  
The concept \ccRefName\ lists the set of requirements that must be fulfilled by
an instance of the \ccc{Traits} template-parameter of
the class \ccc{CGAL::Lines_through_segments_3<Traits>}.
This concept provides the types of the geometric primitives used in
this class.

\ccTypes
\ccNestedType{Geodesic_traits_2}{models the concept 
\ccc{GeodesicTraits_2}}
\ccNestedType{Hyperbolict_traits_2}{models the concept 
\ccc{HyperbolicTraits_2}}
\ccGlue
\ccNestedType{Point_3}{The point type.}
\ccGlue
\ccNestedType{Segment_3}{The segment type.}
\ccGlue
\ccNestedType{Line_3}{The line type.}

\ccCreation
This concept refines the standard concepts DefaultConstructible, Assignable and
CopyConstructible.




\ccHasModels
\ccc{CGAL::Lines_through_segments_traits_3<Rational_kernel, Algebraic_kernel>}

\ccSeeAlso
\ccc{CGAL::Lines_through_segments_3<Traits>}

\end{ccRefConcept}


\begin{ccRefConcept}{Lines_through_segments_3::GeodesicTraits_2}

\ccDefinition
Refinement of the concept \ccc{ArrangementTraits_2}, used to construct and maintain arrangements of circular arcs embedded on spheres. 

\ccHasModels
\ccc{CGAL::Arr_geodesic_arc_on_sphere_traits_2<Kernel>}

\end{ccRefConcept}

\begin{ccRefConcept}{Lines_through_segments_3::HyperbolicTraits_2}

\ccDefinition
Refinement of the concept \ccc{ArrangementTraits_2}, used to construct and maintain arrangements of bounded segments of algebraic curves of degree 2 at most, also known as conic curves.

\ccHasModels
\ccc{CGAL::Arr_conic_traits_2<Rational_kernel, Alg_kernel, Nt_traits>}

\end{ccRefConcept}

% +------------------------------------------------------------------------+
%%RefPage: end of main body, begin of footer
% EOF
% +------------------------------------------------------------------------+

